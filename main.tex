% This is based on the LLNCS.DEM the demonstration file of
% the LaTeX macro package from Springer-Verlag
% for Lecture Notes in Computer Science,
% version 2.4 for LaTeX2e as of 16. April 2010
%
% See http://www.springer.com/computer/lncs/lncs+authors?SGWID=0-40209-0-0-0
% for the full guidelines.
%
\documentclass[11pt]{article}

\usepackage[utf8]{inputenc}
\usepackage[english]{babel}
\usepackage[margin=1in]{geometry}
\usepackage{amsmath,amsthm,amssymb,graphicx,stmaryrd}
\usepackage{hyperref,thmtools,enumitem,xcolor}

\newcommand{\vph}{\varphi}
\newcommand{\vep}{\varepsilon}
\newcommand{\N}{\mathbb{N}}
\newcommand{\Z}{\mathbb{Z}}
\newcommand{\U}{\mathsf{U}}
\renewcommand{\P}{\mathbb{P}}
\newcommand{\ctx}{\;\mathsf{ctx}}
\newcommand{\scott}[1]{\llbracket #1\rrbracket}
\newcommand{\ttag}[1]{\llparenthesis #1\rrparenthesis}
\newcommand{\elet}[2]{\mathsf{let}\;#1\mathrel{\mathsf{in}}#2}
\newcommand{\rec}{\mathsf{rec}}
\newcommand{\ok}{\ \mathsf{ok}}
\newcommand{\type}{\ \mathsf{type}}
\newcommand{\spec}{\ \mathsf{spec}}
\newcommand{\ctor}{\ \mathsf{ctor}}
\newcommand{\LE}{\ \mathsf{LE}}
\newcommand{\LEctor}{\ \mathsf{LE\;ctor}}
\newcommand{\mk}{\mathsf{mk}}
\newcommand{\lift}{\mathsf{lift}}
\newcommand{\sound}{\mathsf{sound}}
\newcommand{\acc}{\mathsf{acc}}
\newcommand{\intro}{\mathsf{intro}}
\newcommand{\inv}{\mathsf{inv}}
\newcommand{\inl}{\mathsf{inl}\;}
\newcommand{\inr}{\mathsf{inr}\;}
\newcommand{\up}{\mathsf{up}\;}
\newcommand{\dn}{\mathsf{dn}\;}
\newcommand{\ulift}{\mathsf{ulift}}
\renewcommand{\sup}{\mathsf{sup}\;}
\newcommand{\W}{\mathsf{W}}
\newcommand{\nonempty}{\mathsf{nonempty}}
\newcommand{\refl}{\mathsf{refl}}
\newcommand{\lvl}{\operatorname{lvl}}
\newcommand{\sort}{\operatorname{sort}}
\newcommand{\rank}{\operatorname{rank}}
\newcommand{\obj}{\mathsf{obj}}

\DeclareMathOperator{\imax}{imax}
\let\lGamma\Gamma \renewcommand{\Gamma}{\mathrm{\lGamma}}
\let\lDelta\Delta \renewcommand{\Delta}{\mathrm{\lDelta}}
\let\lSigma\Sigma \renewcommand{\Sigma}{\mathrm{\lSigma}}

\makeatletter
\providecommand{\leftsquigarrow}{\mathrel{\mathpalette\reflect@squig\relax}}
\newcommand{\reflect@squig}[2]{\reflectbox{$\m@th#1\rightsquigarrow$}}
\makeatother

\hypersetup{
    colorlinks,
    linkcolor={red!50!black},
    citecolor={blue!50!black},
    urlcolor={blue!80!black}
}

\addto\extrasenglish{
  \let\subsectionautorefname\sectionautorefname
  \let\subsubsectionautorefname\sectionautorefname
  \def\theoremautorefname{theorem}}

\declaretheorem[
name=Theorem,
refname={theorem,theorems},
Refname={Theorem,Theorems},
numberwithin=section]{theorem}
\declaretheorem[
name=Lemma,
refname={lemma,lemmas},
Refname={Lemma,Lemmas},
sibling=theorem]{lemma}
\newtheorem{remark}{Remark}
\declaretheorem[
name=Corollary,
refname={corollary,corollaries},
Refname={Corollary,Corollaries},
sibling=theorem]{corollary}
\newtheorem{definition}{Definition}
\setlist[itemize]{parsep=0pt,topsep=2pt,before=\leavevmode}
\setlist[enumerate]{parsep=0pt,topsep=2pt,before=\leavevmode}

\newlist{thmlist}{enumerate}{1}
\setlist[thmlist]{label=(\arabic{thmlisti}),
  ref=\thelemma.(\arabic{thmlisti}),
  parsep=0pt,topsep=2pt,before=\leavevmode}

\setlength{\parindent}{1em}
\setlength{\parskip}{.3em}

\begin{document}

\title{The Type Theory of Lean}
\author{Mario Carneiro}

\maketitle        % Print title page.
\section*{Abstract}
% \addcontentsline{toc}{chapter}{abstract}

This thesis is a presentation of dependent type theory with inductive types, a hierarchy of universes, with an impredicative universe of propositions, proof irrelevance, and subsingleton elimination, along with axioms for propositional extensionality, quotient types, and the axiom of choice. This theory is notable for being the axiomatic framework of the Lean theorem prover. The axiom system is given here in complete detail, including ``optional'' features of the type system such as $\mathsf{let}$ binders and definitions. We provide a reduction of the theory to a finitely axiomatized fragment utilizing a fixed set of inductive types (the $\W$-type plus a few others), to ease the study of this framework.

The metatheory of this theory (which we will call Lean) is studied. In particular, we prove unique typing of the definitional equality, and use this to construct the expected set-theoretic model, from which we derive consistency of Lean relative to $\mathsf{ZFC}+\{\mbox{there are }n\mbox{ inaccessible cardinals}\mid n<\omega\}$ (a relatively weak large cardinal assumption). As Lean supports models of ZFC with $n$ inaccessible cardinals, this is optimal.

We also show a number of negative results, where the theory is less nice than we would like. In particular, type checking is undecidable, and the type checking as implemented by the Lean theorem prover is a decidable non-transitive underapproximation of the typing judgment. Non-transitivity also leads to lack of subject reduction, and the reduction relation does not satisfy the Church-Rosser property, so reduction to a normal form does not produce a decision procedure for definitional equality. However, a modified reduction relation allows us to restore the Church-Rosser property at the expense of guaranteed termination, so that unique typing is shown to hold.

\tableofcontents  % Print table of contents

\section{Introduction}
\subsection{Type theory in programming languages}
The history of types in mathematical logic dates back to Frege's \emph{Begriffsschrift} \cite{frege}, which establishes a notation system for what amounts to second-order logic with equality. Bertrand Russell discovered a paradox in Frege's system: The predicate $P(A):=\neg A(A)$ leads to a contradiction (or in set-theoretic notation, the set $S=\{x\mid x\notin x\}$ cannot be a set). In reaction, Ernst Zermelo resolved the contradiction by imposing a ``size restriction'' on sets, leading to Zermelo set theory and eventually to Zermelo-Fraenkel set theory (ZFC), which has become the gold standard for axiomatization in modern mathematics. This yields an untyped but stratified view of the universe of mathematical concepts.

Russell's own reaction to Russell's paradox was instead to impose a stratification on the language itself, rejecting the expressions $A(A)$ or $x\in x$ as ``ill-typed''. This line of reasoning says that $A$ is not an object that predicates on objects of the same type as itself, so the notion is prima facie ill-formed. This idea is developed in \emph{Principia Mathematica} \cite{principia} and Quine's New Foundations \cite{quinenf}, but the most relevant application was to the simply typed $\lambda$-calculus \cite{churchstt} by Church (1940).

Somewhat independently, programming languages rediscovered the idea of a type \cite{typesoftypes}. Early programming languages had no explicit notion of type. Lisp used an evaluation model closely related to the untyped $\lambda$-calculus. FORTRAN (1956) had ``modes'' of expressions, either fixed or floating point. Algol 60 (1960) developed expressions and variables of type (\textbf{integer}, \textbf{real}, \textbf{Boolean}), and the extension Algol W by Wirth and Hoare (1966) developed a generative syntax for types including record types and typed references.

The logical and programming traditions are finally explicitly connected in the Curry-Howard isomorphism \cite{curryhoward}, which observed the connection between logical derivations (in the sequent calculus) and lambda terms in the simply typed $\lambda$-calculus. (In the same correspondence, Howard also discusses extensions to first order logic, with lambdas ranging over ``number variables'' $(\lambda x.\,F^\beta)^{\forall x\,\beta}$ separate from typed lambdas $(\lambda X^\alpha.\,F^\beta)^{\alpha\supset\beta}$.) But dependent type theory really begins in earnest with Per Martin-L\"{o}f \cite{martinlof}, who set the foundations for Brouwer's intuitionistic type theory as an outgrowth of the simply typed $\lambda$-calculus with dependent types.

Martin-L\"{o}f describes how constructive type theory can be used in programming languages:
\begin{quote}
By choosing to program in a formal language for constructive mathematics, like the theory of types, one gets access to the whole conceptual apparatus of pure mathematics, neglecting those parts that depend critically on the law of excluded middle, whereas even the best high level programming languages so far designed are wholly inadequate as mathematical languages (and, of course, nobody has claimed them to be so). In fact, I do not think that the search for logically ever more satisfactory high level programming languages can stop short of anything but a language in which (constructive) mathematics can be adequately expressed. \cite{martinlofprog}
\end{quote}

This dream was converted to action by Thierry Coquand, who developed the Calculus of Constructions (CoC) \cite{coquandcoc} and developed an interactive proof assistant \textsf{Coq} to check proofs in this language \cite{coqart}. This type theory was extended with inductive types \cite{dybjer} to form the Calculus of Inductive Constructions (CIC) \cite{paulincic}.

Lean \cite{demoura} is a theorem prover based on CIC as well, with some subtle but important differences. See \autoref{sec:notcoq} for an elaboration of the differences from Coq's axiomatics. The goal of this paper is to demonstrate the consequences of these differences, without leaving anything out. While CIC itself is well-studied \cite{barrassets, barrastypedec, coqincoq}, most papers study subsystems of the actual axiomatic system implemented in Coq, which might be called $\mathsf{CIC^+}$ for its many small extensions added over the years. While we will not analyze $\mathsf{CIC^+}$ in this paper, we will be able to analyze all the extensions that are in Lean CIC, so our proof of consistency is directly applicable to the full Lean kernel.

UNFINISHED: talk about:
\begin{itemize}
\item Werner \cite{setsintypes}: Sets in Types
\item Miquel \cite{notsosimple}: The not so simple proof irrelevant model of CC
\end{itemize}

\section{The axioms}

\subsection{Typing}

The syntax of expressions is given by the following grammar:
\begin{align*}
\ell&::=u\mid 0\mid S\ell\mid \max(\ell,\ell)\mid \imax(\ell,\ell)\\
e&::=x\mid \U_\ell\mid e\;e\mid \lambda x:e.\;e\mid \forall x:e.\;e\\
\Gamma&::=\cdot\mid \Gamma,x:e
\end{align*}
Here $u$ is a universe variable, and $x$ is an expression variable. The typing judgment is defined by the rules:
$$\boxed{\Gamma\vdash e:\alpha}$$
$$\frac{\Gamma\vdash \alpha:\U_\ell\quad \Gamma\vdash e:\beta}{\Gamma,x:\alpha\vdash e:\beta}\qquad
\frac{\Gamma\vdash \alpha:\U_\ell}{\Gamma,x:\alpha\vdash x:\alpha}\qquad
\frac{}{\vdash\U_\ell:\U_{S\ell}}$$
$$\frac{\Gamma\vdash e_1:\forall x:\alpha.\;\beta\quad\Gamma\vdash e_2:\alpha}{\Gamma\vdash e_1\;e_2:\beta[e_2/x]}\qquad
\frac{\Gamma,x:\alpha\vdash e:\beta}{\Gamma\vdash\lambda x:\alpha.\;e:\forall x:\alpha.\;\beta}$$
$$\frac{\Gamma\vdash\alpha:\U_{\ell_1}\quad\Gamma,x:\alpha\vdash \beta:\U_{\ell_2}}{\Gamma\vdash\forall x:\alpha.\;\beta:\U_{\imax(\ell_1,\ell_2)}}\qquad
\frac{\Gamma\vdash e:\alpha\quad \Gamma\vdash\alpha\equiv\beta}{\Gamma\vdash e:\beta}$$
Each constant has a list of universe variables $\bar u$ that may appear in its type; these are substituted for given universe level expressions in $\tau_{\bar u}(c)[\bar\ell/\bar u]$.

For convenience, we will also define the following simple judgments:
$$\boxed{\Gamma\vdash\alpha\type}\qquad
\frac{\Gamma\vdash\alpha:\U_\ell}{\Gamma\vdash\alpha\type}\qquad
\boxed{\vdash\Gamma\ok}\qquad\frac{}{\vdash \cdot\ok}\qquad\frac{\Gamma\vdash\alpha\type}{\vdash \Gamma,x:\alpha\ok}$$
\subsection{Definitional equality}

We will distinguish two notions of definitional equality: the ``ideal'' definitional equality, denoted $\alpha\equiv\beta$, and ``algorithmic'' definitional equality, denoted $\alpha\Leftrightarrow\beta$, which will imply $\alpha\equiv\beta$ and is what is actually checked by Lean.

$$\boxed{\Gamma\vdash e\equiv e'}$$
$$\frac{\Gamma\vdash e:\alpha}{\Gamma\vdash e\equiv e}\qquad
\frac{\Gamma\vdash e\equiv e'}{\Gamma\vdash e'\equiv e}\qquad
\frac{\Gamma\vdash e_1\equiv e_2\quad\Gamma\vdash e_2\equiv e_3}{\Gamma\vdash e_1\equiv e_3}$$
$$\frac{\ell\equiv\ell'}{\vdash \U_\ell\equiv\U_{\ell'}}\qquad
\frac{\Gamma\vdash e_1\equiv e'_1:\forall x:\alpha.\;\beta\quad \Gamma\vdash e_2\equiv e'_2:\alpha}{\Gamma\vdash e_1\;e_2\equiv e'_1\;e'_2}$$
$$\frac{\Gamma\vdash\alpha\equiv \alpha'\quad \Gamma,x:\alpha\vdash e\equiv e'}{\Gamma\vdash\lambda x:\alpha.\;e\equiv \lambda x:\alpha'.\;e'}\qquad
\frac{\Gamma\vdash \alpha\equiv \alpha'\quad \Gamma,x:\alpha\vdash \beta\equiv \beta'}{\Gamma\vdash \forall x:\alpha.\;\beta\equiv \forall x:\alpha'.\;\beta'}$$
$$(\beta)\ \frac{\Gamma,x:\alpha\vdash e:\beta\quad\Gamma\vdash e':\alpha}{\Gamma\vdash (\lambda x:\alpha.\;e)\;e'\equiv e[e'/x]}\qquad
(\eta)\ \frac{\Gamma\vdash e:\forall y:\alpha.\;\beta}{\Gamma\vdash \lambda x:\alpha.\;e\;x\equiv e}$$
$$\frac{\Gamma\vdash p:\P\quad \Gamma\vdash h:p\quad \Gamma\vdash h':p}{\Gamma \vdash h\equiv h'}$$
The notation $\Gamma\vdash e\equiv e':\alpha$ in the application rule abbreviates $\Gamma\vdash e\equiv e'\land\Gamma\vdash e:\alpha\land\Gamma\vdash e':\alpha$. The last rule is called proof irrelevance, which states that any two proofs of a proposition (a type in $\P:=\U_0$) are equal. Equality of levels is defined in terms of an algorithmic inequality judgement $\ell\le\ell'+n$ where $n\in\Z$ (abbreviated to $\ell\le\ell'$ when $n=0$):

$$\boxed{\ell\equiv\ell'}\qquad \frac{\ell\le \ell'\quad \ell'\le\ell}{\ell\equiv\ell'}$$
$$\boxed{\ell\le \ell'+n}$$
$$\frac{n\ge 0}{0\le \ell+n}\qquad
\frac{n\ge 0}{\ell\le \ell+n}$$
$$\frac{\ell\le\ell'+(n-1)}{S\ell\le \ell'+n}\qquad
\frac{\ell\le\ell'+(n+1)}{\ell\le S\ell'+n}\qquad$$
$$\frac{\ell\le \ell_1+n}{\ell\le \max(\ell_1,\ell_2)+n}\qquad
\frac{\ell\le \ell_2+n}{\ell\le \max(\ell_1,\ell_2)+n}\qquad
\frac{\ell_1\le\ell+n\quad \ell_2\le \ell+n}{\max(\ell_1,\ell_2)\le \ell+n}$$
$$\frac{0\le\ell+n}{\imax(\ell_1,0)\le\ell+n}\qquad
\frac{\max(\ell_1,S\ell_2)\le\ell+n}{\imax(\ell_1,S\ell_2)\le\ell+n}$$
$$\frac{\max(\imax(\ell_1,\ell_3),\imax(\ell_2,\ell_3))\le\ell+n}{\imax(\ell_1,\imax(\ell_2,\ell_3))\le\ell+n}\qquad
\frac{\ell\le\max(\imax(\ell_1,\ell_3),\imax(\ell_2,\ell_3))+n}{\ell\le\imax(\ell_1,\imax(\ell_2,\ell_3))+n}$$
$$\frac{\max(\imax(\ell_1,\ell_2),\imax(\ell_1,\ell_3))\le\ell+n}{\imax(\ell_1,\max(\ell_2,\ell_3))\le\ell+n}\qquad
\frac{\ell\le\max(\imax(\ell_1,\ell_2),\imax(\ell_1,\ell_3))+n}{\ell\le\imax(\ell_1,\max(\ell_2,\ell_3))+n}$$
$$\frac{\ell[0/u]\le \ell'[0/u]+n\quad\ell[Su/u]\le \ell'[Su/u]+n}{\ell\le \ell'+n}$$

Although this definition looks complicated, it is most easily understood in terms of its semantics: A level takes values in $\N$, where $\scott{0}=0$, $\scott{S\ell}=\scott{\ell}+1$, $\scott{\max(\ell_1,\ell_2)}=\max(\scott{\ell_1},\scott{\ell_2})$ and $\scott{\imax(\ell_1,\ell_2)}=\imax(\scott{\ell_1},\scott{\ell_2})$, where $\imax(m,n)$ is the function such that $\imax(m,n+1)=\max(m,n+1)$ and $\imax(m,0)=0$. Then a level inequality $\ell\le\ell+n$ holds if for all substitutions $v$ of numerals for the variables in $\ell$ and $\ell'$, $\scott{\ell}_v\le \scott{\ell'}_v+n$. We will return to this in detail in \autoref{sec:soundness}.

\subsection{Reduction}
The algorithmic definitional equivalence relation is defined in terms of a reduction operation on terms:
%
$$\boxed{\Gamma\vdash e\Leftrightarrow e'}$$
$$\frac{}{\Gamma\vdash e\Leftrightarrow e}\qquad
\frac{\Gamma\vdash e\Leftrightarrow e'}{\Gamma\vdash e'\Leftrightarrow e}\qquad
\frac{\ell\equiv\ell'}{\Gamma\vdash\U_\ell\Leftrightarrow\U_\ell'}$$
$$\frac{\Gamma\vdash\alpha\Leftrightarrow\alpha'\quad \Gamma,x:\alpha\vdash e\Leftrightarrow e'}{\Gamma\vdash\lambda x:\alpha.\;e\Leftrightarrow \lambda x:\alpha'.\;e'}\qquad
\frac{\Gamma\vdash\alpha\Leftrightarrow\alpha'\quad \Gamma,x:\alpha\vdash e\Leftrightarrow e'}{\Gamma\vdash\forall x:\alpha.\;e\Leftrightarrow \forall x:\alpha'.\;e'}$$
$$\frac{}{\Gamma\vdash \lambda x:\alpha.\;e\;x\Leftrightarrow e}\qquad
\frac{\Gamma\vdash p:\P\quad \Gamma\vdash h:p\quad \Gamma\vdash h':p'\quad \Gamma \vdash p\Leftrightarrow p'}{\Gamma \vdash h\Leftrightarrow h'}$$
$$\frac{\Gamma\vdash e_1\Leftrightarrow e'_1\quad \Gamma\vdash e_2\Leftrightarrow e'_2}{\Gamma\vdash e_1\;e_2\Leftrightarrow e'_1\;e'_2}\qquad
\frac{e\downarrow k\quad e'\downarrow k'\quad \Gamma\vdash k\Leftrightarrow k'}{\Gamma\vdash e\Leftrightarrow e'}$$

In this judgment the transitivity rule is notably absent. Most of the congruence rules remain except for the $\beta$ rule, and these constitute all the ``easy'' cases of definitional equality. When the other rules fail to make progress, we use the ``weak head normal form'' reduction $e\downarrow k$ to apply the $\beta$ rule as well as the $\delta,\iota,\zeta$ rules which are discussed in their own section.
%
$$
\begin{matrix}
\boxed{e\downarrow k}\\[4mm]
\displaystyle{
\frac{e\rightsquigarrow e'\quad e'\downarrow k}{e\downarrow k}\qquad
\frac{e\not\rightsquigarrow}{e\downarrow e}}
\end{matrix}\qquad
\begin{matrix}
\boxed{e\rightsquigarrow e'}\\[4mm]
\displaystyle{
\frac{e_1\rightsquigarrow e'_1}{e_1\;e_2\rightsquigarrow e'_1\;e_2}\qquad
\frac{}{(\lambda x:\alpha.\;e)\;e'\rightsquigarrow e[e'/x]}}
\end{matrix}
$$

We will add more rules to this list as we introduce new constructs, but this completes the description of the base dependent type theory foundation for Lean.

\subsection{\textsf{let} binders ($\zeta$ reduction)}

The first and simplest extension to this language is to add support for a let binder. We will define the expression $\elet{x:\alpha:=e'}{e}$ to be equivalent to $e[e'/x]$. (The rule asserting this equality is called $\zeta$-reduction.) This differs from the expression $(\lambda x:\alpha.\;e)\;e'$ in that this expression requires $\lambda x:\alpha.\;e$ to be well-typed, while in the let binder we will be able to make use of a definitional equality $x\equiv e'$ while type-checking the body of $e$. One could imagine extending the context with such definitional equalities, but Lean takes a simpler approach and simply zeta expands these binders when necessary for checking.

Adding let binders to the language entails adding the following clauses to the judgments of the previous sections:
%
$$e::=\dots\mid\elet{x:\alpha:=e'}{e}$$
$$\dots\qquad\frac{\Gamma\vdash e':\alpha\quad \Gamma\vdash e[e'/x]:\beta}{\Gamma\vdash\elet{x:\alpha:=e'}{e}:\beta}$$
$$\dots\qquad(\zeta)\ \frac{\Gamma\vdash e':\alpha\quad \Gamma\vdash e[e'/x]:\beta}{\Gamma\vdash\elet{x:\alpha:=e'}{e}\equiv e[e'/x]}$$
$$\dots\qquad\frac{}{\elet{x:\alpha:=e'}{e}\rightsquigarrow e[e'/x]}$$
Note that we don't have any congruence rules for $\mathsf{let}$. We simply unfold it whenever we need to check anything about it. It is easy to see that this is a conservative extension, because we can replace $\elet{x:\alpha:=e'}{e}$ with $e[e'/x]$ and remove any $\zeta$-reduction steps in a whnf derivation to recover the original system.

\subsection{Definitions ($\delta$ reduction)}

There are two kinds of constants in lean: those with definitions and primitive constants. Both have the form $c_{\bar u}$ where $c$ is a new name and $\bar u$ is a list of universe variables, but a definition will also have a reduction step (called $\delta$ reduction) associated with it.

For a constant definition $\mathsf{constant}\;c_{\bar u}:\alpha$ to be admissible, we require that $\vdash\alpha:\U_\ell$, where the universe variables in $\alpha$ and $\ell$ are contained in $\bar u$. Similarly, a definition is specified by a clause $\mathsf{def}\;c_{\bar u}:\alpha:=e$, which is admissible when $\vdash e:\alpha$ and the universe variables in $e$ and $\alpha$ are contained in $\bar u$. Let $\tau_{\bar\ell}(c)=\alpha[\bar\ell/\bar u]$ and $v_{\bar\ell}(c)=e[\bar\ell/\bar u]$ denote the type and value of the definition after substitution for the universe variables. We add the following rules to the system:
$$e::=\dots\mid c_{\bar u}$$
$$\frac{}{\vdash c_{\bar\ell}:\tau_{\bar\ell}(c)}\qquad
\frac{\ell_1\equiv\ell'_1\ \dots\ \ell_n\equiv\ell'_n}{\vdash c_{\bar\ell}\equiv c_{\bar\ell'}}\qquad\frac{\ell_1\equiv\ell'_1\ \dots\ \ell_n\equiv\ell'_n}{\Gamma\vdash c_{\bar\ell}\Leftrightarrow c_{\bar\ell'}}$$
Furthermore, for definitions, we add the following additional rules:
$$(\delta)\ \frac{}{\vdash c_{\bar\ell}\equiv v_{\bar\ell}(c)}\qquad
\frac{}{c_{\bar\ell}\rightsquigarrow v_{\bar\ell}(c)}$$
It is similarly easy to see that a definition is a conservative extension, because we can replace $c_{\bar\ell}$ with $v_{\bar\ell}(c)$ everywhere and remove any $\delta$-reduction steps to get a derivation which doesn't use the definition. This argument of course does not extend to $\mathsf{constant}$, which has no reduction rules and so is simply an axiomatic extension of the system. We will discuss various consistent and conservative extensions by constants, when definitions will not suffice for technical reasons.

\subsection{Inductive types}\label{sec:inductive}

\subsubsection{Inductive specifications}
Inductive types are by far the most complex feature of Lean's axiomatic system, and moreover are very tricky to prove properties about due to their notational complexity. We will define a syntax for defining inductive types, and judgments for showing that they are admissible.
$$K::=0\mid (c:e)+K$$
This is the type of an inductive specification, which is a list of introduction forms with name $c$ and type $e$. We will write $(c:\alpha)$ for the single constructor form $(c:\alpha)+0$, and abbreviate the whole sequence as $\sum_i(c_i:\alpha_i)$.

Let the notation $(x :: \alpha)$, called a ``telescope'', denote a dependent sequence of binders $x_1:\alpha_1, x_2:\alpha_2,\dots, x_n:\alpha_n$. This will be used in contexts, on the left ($\Gamma,x::\alpha\vdash e:\beta$) as well as on the right ($\Gamma\vdash x::\alpha$); this latter expression means that $\Gamma\vdash x_1:\alpha_1$, and $\Gamma,x_1:\alpha_1\vdash x_2:\alpha_2$, and so on up to\\
$\Gamma,x_1:\alpha_1,\dots,x_{n-1}:\alpha_{n-1}\vdash x_n:\alpha_n$. It will also be used to abbreviate sequences of $\lambda$ and $\forall$ as in $\lambda x::\alpha.\;\beta=\lambda x_1:\alpha_1\dots\lambda x_n:\alpha_n.\;\beta$. If $e::\alpha$ and $f:\forall x::\alpha.\;\beta$, then $f\;e:\beta[e/x]$ denotes the sequence of applications $f\;e_1\dots e_n$.

A specification $K$ is typechecked in a context of a variable $t:F$ where $F=\forall x::\alpha.\;\U_\ell$ is a family of sorts (so $t$ is a family of types). The result will be the recursive type $\mu t:F.\;K$, which roughly satisfies the equivalence $\mu t:F.\;K\simeq K[\mu t:F.\;K/t]$. A specification is a sequence of constructors:
$$\boxed{\Gamma;t:F\vdash K\spec} \qquad
\frac{\Gamma\vdash x::\alpha\quad \Gamma;t:\forall x::\alpha.\;\U_\ell\vdash \beta_i\ctor}{\Gamma;t:\forall x::\alpha.\;\U_\ell\vdash \sum_i(c_i:\beta_i)\spec}$$
A constructor is a sequence of arguments ending in an application with head $t$:
$$\boxed{\Gamma;t:F\vdash \alpha\ctor} \qquad
\frac{\Gamma\vdash e::\alpha}{\Gamma;t:\forall x::\alpha.\;\U_\ell\vdash t\;e\ctor}$$
$$\frac{\Gamma\vdash \beta:\U_{\ell'}\quad \ell'\le\ell\quad \Gamma,y:\beta;t:\forall x::\alpha.\;\U_\ell\vdash\tau\ctor}{\Gamma;t:\forall x::\alpha.\;\U_\ell\vdash\forall y:\beta.\;\tau\ctor}$$
$$\frac{\Gamma\vdash \gamma::\U_{\ell'}\quad \Gamma,z::\gamma\vdash e::\alpha \quad
\imax(\ell',\ell)\le\ell\quad \Gamma;t:\forall x::\alpha.\;\U_\ell\vdash\tau\ctor}{\Gamma;t:\forall x::\alpha.\;\U_\ell\vdash(\forall z::\gamma.\;t\;e)\to\tau\ctor}$$
There are two kinds of arguments, represented by the two inductive cases here. The first kind is a nonrecursive argument. The type of this argument must not mention $t$, but it can be used in the types of later arguments. A recursive argument has the type $\forall z::\gamma.\;t\;e$, and cannot be referenced in later arguments.

With the definition of \textsf{spec} in hand, we can finally define the type constructor and introduction operator:
%
$$e::=\dots\mid\mu x:e.\;K\mid c_{\mu x:e.K}\mid\rec_{\mu x:e.K}$$
$$\frac{\Gamma;t:F\vdash K\spec}{\Gamma\vdash \mu t:F.\;K:F}\qquad
\frac{\Gamma;t:F\vdash K\spec\quad (c:\alpha)\in K}{\Gamma\vdash c_{\mu t:F.K}:\alpha[\mu t:F.\;K/t]}$$
In Lean, $\mu t:F.\;K$ and $c_{\mu t:F.K}$ are implemented as additional axiomatic constant symbols (with no free variables, by abstracting over the variables in $\Gamma$). Having them as binders here makes the substitution story more complicated, so we will treat $\mu t:F.\;K$ as simply a nice syntax for $(\lambda x::\Gamma.\;\mu t:F.\;K)\;x$, so that substitutions do not affect $F$ and $K$.

Before we get to the general definition of the eliminator, let us review an example: the natural numbers. The natural numbers are defined in the above format as $\N:=\mu N:\U_1.\;(z:N)+(s:N\to N)$, yielding constructors $z_\N:\N$ (zero) and $s_\N:\N\to\N$ (successor). The eliminator for $\N$ looks like this:
$$\mathsf{rec}_\N:\forall (C:\N\to \U_u).\;C\;z_\N\to(\forall x:\N.\;C\;x\to C\;(s_\N\;x))\to\forall n:\N.\;C\;n$$

There are three components to this definition: the ``motive'' $C$, which will be a type family over the inductive type family just constructed, the ``minor premises'' $C\;z_\N$ and $\forall x:\N.\;C\;x\to C\;(s_\N\;x)$, which asserts that $C$ preserves each constructor, and the ``major premise'' $n:\N$ which then produces an element of the type family $C\;n$. We want to generalize each of these pieces.

\subsubsection{Large elimination}\label{sec:large_elim}
One additional point requires noting in the previous example: The type family $C$ ranges over an arbitrary universe $u$. This is called \emph{large elimination} because it means that one can use recursion over natural numbers to produce functions in large universes. By contrast, the existential quantifier (defined as an inductive predicate) does not have large elimination, meaning that the motive only ranges over $\P$ instead of $\U_u$.

There are two reasons an inductive type can be large eliminating:
\begin{enumerate}
\item The type family $t:\forall x::\alpha.\;\U_\ell$ lives in a universe $1\le\ell$. (This means that $\ell$ is not zero for any values of the parameters.) $\N$ falls into this category.
\item The type family has at most one constructor, and all the non-recursive arguments to the constructor are either propositions or directly appear in the output type. This is called \emph{K-like elimination}, and is relevant for the definition of equality as a large eliminating proposition.
\end{enumerate}
Here it is again with an explicit judgment:

$$\boxed{\Gamma;t:F\vdash K\LE}$$
$$\frac{1\le\ell}{\Gamma;t:\forall x::\alpha.\;\U_\ell\vdash K\LE}\qquad
\frac{}{\Gamma;t:F\vdash 0\LE}\qquad
\frac{\Gamma;t:F\vdash \alpha\LEctor}{\Gamma;t:F\vdash (c:\alpha)\LE}$$
$$\boxed{\Gamma;t:F\vdash \alpha\LEctor}$$
$$\frac{}{\Gamma;t:F\vdash t\;e\LEctor}\qquad
\frac{\Gamma,t:F\vdash \alpha:\P\quad \Gamma,x:\alpha;t:F\vdash\beta\LEctor}{\Gamma;t:F\vdash\forall x:\alpha.\;\beta\LEctor}$$
$$\frac{\Gamma;t:F\vdash\beta\LEctor}{\Gamma;t:F\vdash(\forall z::\gamma.\;t\;e)\to\beta\LEctor}$$
$$\frac{y\in e\quad \Gamma,y:\beta;t:\forall x::\alpha.\;\U_\ell\vdash\forall z::\gamma.\; t\;e\LEctor}{\Gamma;t:\forall x::\alpha.\;\U_\ell\vdash\forall y:\beta.\;\forall z::\gamma.\;t\;e\LEctor}$$

In the final rule, $y\in e$ means that $y$ is one of the elements of the sequence $e::\alpha$. Intuitively, you should think of these rules as ensuring that the inductive type contains at most one element: With multiple constructors or a non-propositional argument, you could inhabit the type with more than one element, unless the argument to the constructor is also a parameter to the type family, in which case each distinct element of the argument type maps to a different member of the inductive type family. The equality type is defined with the following signature:

$$\alpha:\U_\ell,a:\alpha\vdash \mathsf{eq}_a:=\mu t:\alpha\to\P.\;(\mathsf{refl}:t\;a)$$

and although it is a type family over $\P$ (so it fails the first reason to be large eliminating), it has exactly one constructor, with no arguments, so it is large eliminating. Another important large eliminating type is the accessibility relation, which is the source of proof by well-founded recursion:

\begin{align*}
&\alpha:\U_\ell,r:\alpha\to\alpha\to\P\vdash\acc_r:=\mu A:\alpha\to\P.\\
&\qquad(\intro:\forall x:\alpha.\;(\forall y:\alpha.\;r\;y\;x\to A\;y)\to A\;x)
\end{align*}

Here we have K-like elimination because the nonrecursive argument $x:\alpha$ appears in the target type $A\;x$.

\subsubsection{The recursor}
To give a uniform description of the recursor and operations on it, let us label all the parts of an inductive definition $\mu t:F.\;K$.
\begin{align*}
F&=\forall a::\alpha.\;\U_\ell\\
P&=\mu t:F.\;K\\
K&=\textstyle{\sum_c(c:\forall b::\beta.\;t\;p[b])}\\
u::\gamma&\subseteq b::\beta\mbox{ is the subsequence of recursive arguments}\\
\mbox{with}\ \gamma_i&=\forall x::\xi_i.\;P\;\pi_i[b,x].
\end{align*}

Here $\Gamma,b::\beta\vdash p[b]::\alpha$ is a sequence of terms depending on the nonrecursive arguments in $b::\beta$, and $\Gamma,b::\beta,x::\xi_i\vdash \pi_i[b,x]::\alpha$ is also a sequence of terms. Now the type of the recursor is:
$$\frac{\Gamma,t:F\vdash K\spec}{\Gamma\vdash\rec_P:\forall C:\kappa.\;\forall e::\vep.\;\forall a::\alpha.\;\forall z:P\;a.\;C\;a\;z}$$
where:
\begin{itemize}
\item $\kappa=\forall a::\alpha.\;C\;a\to\U_u$ where $u$ is a fresh universe variable if $\Gamma;t:F\vdash K\LE$, otherwise $\kappa=\forall a::\alpha.\;C\;a\to\P$,
\item $\vep$ is a sequence of the same length as $K$, where $\vep_c=\forall b::\beta.\;\forall v::\delta.\;C\;p[b]\;(c\;b)$,
\item $\delta$ is a sequence of the same length as $\gamma$, where $\delta_i=\forall x::\xi_i.\;C\;\pi_i[b,x]\;(u_i\;x)$.
\end{itemize}

\subsubsection{The computation rule ($\iota$ reduction)}

There is one more part to the definition of an inductive type: the so called $\iota$ rule. This states that a recursor evaluated on a constructor gives the corresponding case. For example, for $\N$ we have the rules:
%
\begin{align*}
\mathsf{rec}_\N\;C\;a\;f\;z_\N&\equiv a\\
\mathsf{rec}_\N\;C\;a\;f\;(s_\N\;n)&\equiv f\;n\;(\mathsf{rec}_\N\;C\;a\;f\;n)
\end{align*}
%
In general, using the same names as in the previous section, we have the following computational rule corresponding to $(c:\forall b::\beta.\;t\;p[b])$:
$$\frac{\Gamma,t:F\vdash K\spec}{\Gamma,C:\kappa,e::\vep,b::\beta\vdash\rec_P\;C\;e\;p[b]\;(c\;b)\equiv e_c\;b\;v}$$
where $v::\delta$ is defined as $v_i=\lambda x::\xi_i.\;\rec_P\;C\;e\;\pi_i[b,x]\;(u_i\;x)$. (Technically, the reduction rule is all substitution instances of this rule for all the variables left of the turnstile.) This is also implemented as a reduction rule:
$$\frac{}{\rec_P\;C\;e\;p[b]\;(c\;b)\rightsquigarrow e_c\;b\;v}$$

This rule suffices for the theoretical presentation, but there is a second reduction rule used for K-like eliminators. It can be thought of as a combination of proof irrelevance to change the major premise into a constructor followed by the iota rule.

$$\frac{F=\forall a::\alpha.\;\P}{\rec_P\;C\;e\;p[b]\;h\rightsquigarrow e_c\;b\;v}$$
This rule only applies when all the variables in $b$ are actually on the LHS, which is the reason for the peculiar requirements on K-like eliminators. If $b_i$ appears in the parameters for its type, that means that $p_j[b]=b_i$ for some $j$, and so $b_i$ is on the LHS. 

The foremost example of this is known in the literature as axiom K, which is the reason for the name ``K-like eliminator'', which is this principle applied to the equality type:
$$\mathsf{rec}_{\mathsf{eq}a}\;C\;x\;a\;h\equiv x$$
Here $\mathsf{rec}_{\mathsf{eq}a}\;C:a=b\to C\;a\to C\;b$ is the substitution principle of equality (suppressing the dependence of $C$ on the proof argument), and the computation rule says that ``casting'' $x:C\;a$ over an equality $h:a=a$ produces $x$ again.
%
\subsection{Non-primitive axioms}

All the axioms mentioned thus far are built into Lean so that they are valid even before the first line of code. There are three more axioms that are defined later:

\subsubsection{Quotient types}
Given a type $\alpha:\U_u$ and a relation $R:\alpha\to\alpha\to\P$, the quotient $\alpha/R$ represents the largest type with a surjection $\mk_R:\alpha\to\alpha/R$ such that two elements which are $R$-related are identified in the quotient. Formally, we have the following constants (all of which have two extra arguments for $\alpha$ and $R$):
\begin{align*}
\alpha/R&:\U_u\\
\mk_R&:\alpha\to\alpha/R\\
\sound_R&:\forall x\;y:\alpha.\;R\;x\;y\to \mk_R\;x=\mk_R\;y\\
\lift_R&:\forall\beta:\U_v.\;\forall f:\alpha\to\beta.\;(\forall x\;y:\alpha.\;R\;x\;y\to f\;x=f\;y)\to \alpha/R\to\beta\\
\lift_R&\;\beta\;f\;h\;(\mk_R\;a)\rightsquigarrow f\;a
\end{align*}
Because the last rule is a computational rule, not a constant, and Lean does not support adding computational rules to the kernel, this is a ``semi-builtin'' axiom; one has the option to disable quotient types, or to enable them and get the computational rule. Also, only $\sound_R$ is considered an axiom here, even though all four are undefined constants, because the other constants and the computational rule would all be satisfied with the definitions $\alpha/R:=\alpha$, $\mk_R\;a:=a$, $\lift_R\;f\;h:=f$. As a terminological note, the rule $\lift_R\;f\;h\;(\mk_R\;a)\rightsquigarrow f\;a$ is also referred to as an $\iota$ reduction rule.

\subsubsection{Propositional extensionality}
The axiomatics of the Calculus of Inductive Constructions (CIC) in general leave equality of types in a universe almost completely unspecified, so that most of these statements are left undecided. For example, the notation $\mu t:F.\;K$ defined here for inductive types seems to suggest that the type is determined by $F$ and $K$, but in fact in Lean you can write exactly the same inductive definition twice and get two possibly distinct (but isomorphic) types. (We could repair our construction here by marking a recursive type with an arbitrary name or number $\mu_i t:F.\;K$ so that we can make such ``mirror copy'' types.)

However this sort of agnosticism is quite annoying to work with in practice when dealing with propositions, for which we would like to use the substitution axiom of equality to substitute equivalent propositions. To that end, the propositional extensionality axiom says that propositions that imply each other are equal:
$$\mathsf{propext}:\forall p\;q:\P.\;(p\leftrightarrow q)\to p=q$$

\subsubsection{Axiom of choice}
The axiom of choice in Lean is expressed as a global choice function, and is simply stated by saying that there is a function from proofs that $\alpha$ is nonempty to $\alpha$ itself. We need the definition of $\nonempty$ for this:
\begin{align*}
\nonempty&:=\lambda\alpha:\U_u.\;\mu t:\P.\;(\intro:\alpha\to t)\\
\mathsf{choice}&:\forall\alpha:\U_u.\;\nonempty\;\alpha\to\alpha
\end{align*}
From the axiom of choice, the law of excluded middle is derived (it is not stated as a separate axiom).

\section{Properties of the type system}
A theorem we would like to have of Lean's type system is that it is consistent, and sound with respect to some semantics in a well understood axiom system such as ZFC. Moreover, we want to relate this to Lean's actual typechecker, in the sense that anything Lean verifies as type-correct will be derivable in this axiom system and hence Lean will not certify a contradiction. But first we must understand some aspects of the type system itself, before relating it to other systems.

It is important to note that \emph{Lean's typechecker is not complete.} Obviously Lean can fail on correct theorems due to, say, running out of resources, but the ``algorithmic equality'' relation does not validate all definitional equalities. In fact, we can show that definitional equality as defined here is undecidable.

\subsection{Undecidability of definitional equality}\label{sec:undecidable}
Recall the type $\acc$ from section \ref{sec:large_elim}:
$$\acc_{<}:=\mu A:\alpha\to\P.\ (\intro:\forall x:\alpha.\;(\forall y:\alpha.\;y<x\to A\;y)\to A\;x)$$
(We are fixing a type $\alpha$ and a relation ${<}:\alpha\to\alpha\to\P$ here.) Informally, we would read this as: ``$x$ is $<$-accessible if for all $y<x$, $y$ is $<$-accessible''. Accessibility is then inductively generated by this clause. If every $x:\alpha$ is accessible, then $<$ is a well-founded relation. One interesting fact about $\acc$ is that we can project out the argument given a proof of $\acc\;x$:
\begin{align*}
\inv_x&:\acc\;x\to\forall y:\alpha.\;y<x\to\acc\;y\\
\inv_x&:=\lambda a:\acc\;x.\;\lambda y:\alpha.\;\rec_{\acc}\;(\lambda z.\;y<z\to\acc\;y)\\
&(\lambda z.\;\lambda h:(\forall w.\;w<z\to\acc\;w).\;\lambda \_.\;h\;y)\;x\;a
\end{align*}
Note that the output type of $\inv_x$ is the same as the argument to $\intro\;x$. Thus, we have
$$a\equiv\intro_\acc\;x\;(\inv_x\;a)$$
by proof irrelevance.

Why does this matter? Normally, any proof of $\acc\;x$ could only be unfolded finitely many times by the very nature of inductive proofs, but if we are in an inconsistent context, it is possible to get a proof of wellfoundedness which isn't actually wellfounded, and we can end up unfolding it forever.

To show how to get undecidability from this, suppose $P:\N\to\bf 2$ is a decidable predicate, such as $P\;n:=\;$``Turing machine $M$ runs for at least $n$ steps without halting''. It is not difficult to show that $P\;n$ is decidable but $\forall n.\;P\;n$ is not. Let $>$ be the standard greater-than function on $\N$ (which is not well-founded). We define a function $f:\forall n.\;\acc_{>}\;n\to\bf 1$ as follows:
\begin{align*}
f&:=\rec_{\acc}\;(\lambda\_.\;{\bf 1})\;(\lambda n\;\_\;(g:\forall y.\;y>x\to{\bf 1}).\\
&\qquad\mathsf{if}\;P\;n\;\mathsf{then}\;g\;(n+1)\;(p\;n)\;\mathsf{else}\;()
\end{align*}
where $p\;n$ is a proof of $n<n+1$. Of course this whole function is trivial since the precondition $\acc_{>}n$ is impossible, but definitional equality works in all contexts, including inconsistent ones. This function evaluates as:
$$f\;n\;(\intro_\acc\;n\;h)\rightsquigarrow^*\mathsf{if}\;P\;n\;\mathsf{then}\;f\;(n+1)\;(h\;(n+1)\;(p\;n))\;\mathsf{else}\;()$$
and the \textsf{if} statement evaluates to the left or right branch depending on whether $P\;n\rightsquigarrow^*\mathsf{tt}$ or $P\;n\rightsquigarrow^*\mathsf{ff}$. Now, this is all true of the reduction relation $\rightsquigarrow$, but if we bring in the full power of definitional equivalence we have the ability to work up from a single proof $a:\acc_{>}\;0$:
\begin{align*}
f\;0\;a&\equiv f\;0\;(\intro_\acc\;0\;(\inv_0\;a))\\
&\equiv f\;1\;(\inv_0\;a\;1\;(p\;0))\\
&\equiv f\;1\;(\intro_\acc\;1\;(\inv_1\;(\inv_0\;a\;1\;(p\;0)))\\
&\equiv f\;2\;(\inv_1\;(\inv_0\;a\;1\;(p\;0))\;2\;(p\;1))\\
&\equiv\dots
\end{align*}
where we have shown the case where $P\;0$ and $P\;1$ both evaluate to true. If any $P\;n$ evaluates to false, then we will eventually get an equivalence to $()$, but if $P\;n$ is always true, then $f$ will never reduce to $()$ -- every term definitionally equal to $f\;0\;a$ will contain a subterm def.eq. to $f$. So $a:\acc_{>}\;0\vdash f\;0\;a\equiv()$ holds if and only if $\forall n.\;P\;n$, and hence $\equiv$ is undecidable.

\subsubsection{Algorithmic equality is not transitive}
From the results of the previous section, given that algorithmic equality is implemented by Lean, and hence is obviously decidable, they cannot be equal as relations, so there is some rule of definitional equality that is not respected by algorithmic equality. In the above example, we can typecheck the various parts of the equality chain to see that $\Leftrightarrow$ is not transitive:
\begin{align*}
f\;0\;a&\Leftrightarrow f\;0\;(\intro_\acc\;0\;(\inv_0\;a))\\
&\Leftrightarrow f\;1\;(\inv_0\;a\;1\;(p\;0))\\
&\mbox{but}\\
f\;0\;a&\not\Leftrightarrow f\;1\;(\inv_0\;a\;1\;(p\;0)).
\end{align*}
We can think of the middle step $f\;0\;(\intro_\acc\;0\;(\inv_0\;a))$ as a ``creative'' step, where we pick one of the many possible terms of type $\acc_{>}\;0$ which happens to reduce in the right way. But since the expression $f\;0\;a$ is a normal form, we don't attempt to reduce it, and indeed if we did we would have nontermination problems (since reduction here only makes the term larger).

Note that the fact that we are in an inconsistent context doesn't matter for this: we could have used $a:\acc_{<}\;1$ with the same result.

This instance of non-transitivity can be traced back to the usage of a K-like eliminator via $\acc$. There is another, less known source of non-transitivity: quotients of propositions. While this is not a particularly useful operation, since any proposition is already a subsingleton, so a quotient will not do anything, they can technically be formed, and $\lift$ acts like a K-like eliminator in this case. So for example, if $p:\P$, $R:p\to p\to\P$, $\alpha:\U_1$, $f:p\to\alpha$, $H:\forall x\;y.\;r\;x\;y\to f\;x= f\;y$, $q:p/R$ and $h:p$, then:
\begin{align*}
\lift_R\;\alpha\;f\;H\;q&\Leftrightarrow \lift_R\;\alpha\;f\;H\;(\mk_R\;h)\Leftrightarrow f\;h\\
&\mbox{but}\\
\lift_R\;\alpha\;f\;H\;q&\not\Leftrightarrow f\;h.
\end{align*}

\subsubsection{Failure of subject reduction}
While the type system given here actually satisfies subject reduction (which is to say, if $\Gamma\vdash e:\alpha$ and $e\rightsquigarrow e'$ (or $\Gamma\vdash e\Leftrightarrow e'$, or $\Gamma\vdash e\equiv e'$), then $\Gamma\vdash e':\alpha$), this is because we use the $\equiv$ relation in the conversion rule $\Gamma\vdash e:\alpha$, $\Gamma\vdash \alpha\equiv\beta$ implies $\Gamma\vdash e:\beta$. If we used algorithmic equality instead, to get a variant typing judgment $\Gamma\Vdash e:\alpha$ closer to what one would expect of the Lean typechecker, we find failure of subject reduction, directly from failure of transitivity. If $\Gamma\vdash\alpha\Leftrightarrow\beta$, $\Gamma\vdash\beta\Leftrightarrow\gamma$, $\Gamma\vdash\alpha\not\Leftrightarrow\gamma$, and $\Gamma\Vdash e:\gamma$, then:
\begin{itemize}
\item $\Gamma\Vdash \mbox{id}_\beta\;e:\beta$ because the application forces checking $\Gamma\vdash\beta\Leftrightarrow\gamma$.
\item $\Gamma\Vdash \mbox{id}_\alpha\;(\mbox{id}_\beta\;e):\alpha$ since the application forces checking $\Gamma\vdash\alpha\Leftrightarrow\beta$.
\item But $\Gamma\not\Vdash \mbox{id}_\alpha\;e:\alpha$ because this requires $\Gamma\vdash\alpha\Leftrightarrow\gamma$ which is false.
\end{itemize}
Since we obviously have $\mbox{id}_\beta\;e\rightsquigarrow e$ by the $\beta$ and $\delta$ rules, this is a counterexample to subject reduction.
\subsection{Regularity}
These lemmas are essentially trivial inductions and are true by virtue of the way we set up the type system, so they are recorded here simply to keep track of the invariants.

\begin{lemma}[Regularity]\label{thm:reg}
\begin{enumerate}
\item If $\Gamma\vdash e:\alpha$, then $\vdash\Gamma\;\mathsf{ok}$.
\item If $\Gamma\vdash e:\alpha$, then $FV(e)\cup FV(\alpha)\subseteq\Gamma$.
\item If $\Gamma\vdash\alpha\type$, then $\Gamma\vdash\alpha:\U_\ell$ for some $\ell$.
\item If $\Gamma\vdash e:\alpha$, then $\Gamma\vdash\alpha\type$.
\item\label{item:defeq_reg2} If $\Gamma\vdash e\equiv e'$, then there exists $\alpha,\alpha'$ such that $\Gamma\vdash e:\alpha$ and $\Gamma\vdash e':\alpha'$.
\item If $\Gamma\vdash e:\alpha$ and $e\rightsquigarrow e'$, then $\Gamma\vdash e\equiv e'$.
\item If $\Gamma\vdash e:\alpha$ and $e\downarrow k$, then $\Gamma\vdash e\equiv k$.
\item\label{item:alg_defn} If $\Gamma\vdash e:\alpha$ and $\Gamma\vdash e':\alpha$, and $\Gamma\vdash e\Leftrightarrow e'$, then $\Gamma\vdash e\equiv e'$.
\item If $\Gamma;t:F\vdash K\spec$, then $\Gamma\vdash F\type$ (and more precisely, $F=\forall x::\alpha.\;\U_\ell$ for some $\alpha,\ell$).
\item If $\Gamma;t:F\vdash K\spec$ and $(c:\alpha)\in K$, then $\Gamma;t:F\vdash\alpha\ctor$.
\item If $\Gamma;t:F\vdash \alpha\ctor$, then $\Gamma,t:F\vdash\alpha\type$.
\end{enumerate}
\end{lemma}
\begin{proof}
By induction on the respective judgments (all of the parts may be proven separately).
\end{proof}

\begin{lemma}[Weakening]\label{thm:weak}
\begin{enumerate}
\item If $\Gamma\vdash e:\alpha$ and $\vdash\Gamma,\Delta\ok$, then $\Gamma,\Delta\vdash e:\alpha$.
\item If $\Gamma\vdash e\equiv e'$ and $\vdash\Gamma,\Delta\ok$, then $\Gamma,\Delta\vdash e\equiv e'$.
\item If $\Gamma,\Delta\vdash e:\alpha$ and $FV(e)\subseteq\Gamma$, then $\Gamma\vdash e:\alpha$.
\item If $\Gamma,\Delta\vdash e\equiv e'$ and $FV(e)\cup FV(e')\subseteq\Gamma$, then $\Gamma\vdash e\equiv e'$.
\item $\Gamma\vdash e:\alpha$ implies $\Gamma\vdash' e:\alpha$, and $\Gamma\vdash e\equiv e'$ implies $\Gamma\vdash' e\equiv e'$, where the modified judgment $\vdash'$ eliminates the weakening rules and replaces the variable and universe rules with
$$\frac{(x:\alpha)\in\Gamma}{\Gamma\vdash x:\alpha}\qquad\frac{}{\Gamma\vdash \U_\ell:\U_{S\ell}}\qquad\frac{\ell\equiv\ell'}{\Gamma\vdash \U_\ell\equiv\U_{\ell'}}$$
\end{enumerate}
\end{lemma}
\begin{proof}
(1,2) and (3,4) are each proven by mutual induction on the first hypothesis. For (5), since weakening is provable for the judgment $\vdash'$ it follows that all rules of $\vdash$ are provable in $\vdash'$.
\end{proof}

\begin{lemma}[Properties of substitution]\label{thm:subst}
\begin{enumerate}
\item If $\Gamma,x:\alpha\vdash e_1\equiv e_1'$ and $\Gamma\vdash e_2:\alpha$, then $\Gamma\vdash e_1[e_2/x]\equiv e_1'[e_2/x]$.
\item\label{item:subst_ty} If $\Gamma,x:\alpha\vdash e_1:\beta$ and $\Gamma\vdash e_2:\alpha$, then $\Gamma\vdash e_1[e_2/x]:\beta[e_2/x]$.
\item If $\Gamma,x:\alpha\vdash e_1:\beta$ and $\Gamma\vdash e_2\equiv e_2':\alpha$, then $\Gamma\vdash e_1[e_2/x]\equiv e_1[e_2'/x]$.
\end{enumerate}
\end{lemma}
\begin{proof} (1) and (2) must be proven simultaneously by induction on the first hypotheses. All cases are straightforward. In the proof irrelevance case, we know $\Gamma,x:\alpha\vdash e_1:p$ and $\Gamma,x:\alpha\vdash e_1':p$ for some $p$ with $\Gamma,x:\alpha\vdash p:\P$. By the induction hypothesis, $\Gamma\vdash e_1[e_2/x]:p[e_2/x]$ and $\Gamma\vdash e_1'[e_2/x]:p[e_2/x]$ and $\Gamma\vdash p[e_2/x]:\P[e_2/x]$; but $\P[e_2/x]=\P$ so proof irrelevance applies to show $\Gamma\vdash e_1[e_2/x]=e_1'[e_2/x]$.

(3) is proven by induction on the structure of $e_1$ and applying compatibility lemmas in each case.
\end{proof}

With this theorem we can upgrade lemma \ref{thm:reg}(\ref{item:defeq_reg2}) to:
\begin{lemma}[Regularity continued]
\begin{enumerate}
\item If $\Gamma\vdash e\equiv e'$, then there exists $\alpha$ such that $\Gamma\vdash e\equiv e':\alpha$.
\end{enumerate}
\end{lemma}
\begin{proof}
Straightforward induction on the derivation of $\Gamma\vdash e\equiv e'$. We need lemma \ref{thm:subst}(\ref{item:subst_ty}) to typecheck both sides of the $\beta$ rule. Note that the induction hypothesis is not strong enough for the application rule, except that we explicitly require that both sides have agreeing types in this case.
\end{proof}

\subsection{Minimal derivations}
Let the notation $\Gamma\vdash_0e:\alpha$ mean that $\Gamma\vdash e:\alpha$, and there is no derivation of  $\Gamma\vdash' e:\alpha'$ that has fewer steps than some derivation of $\Gamma\vdash' e:\alpha$ (where the number of steps of a derivation is the sum of the number of steps of the hypotheses to the rule plus one), where the $\vdash'$ derivation is the weakening-free judgment defined in lemma \ref{thm:weak}. (Alternatively, we could count steps in a $\vdash$ derivation but ignore weakening steps.) The step counting does not include embedded derivations of $\Gamma\vdash e\equiv e'$, it only counts the typing steps.

\begin{lemma}[Properties of $\vdash_0$]
\begin{enumerate}
\item If $\Gamma\vdash e:\beta$, then $\Gamma\vdash_0e:\alpha$ for some $\alpha$.
\item The final rule in a derivation of $\Gamma\vdash_0e:\alpha$ cannot be the conversion rule.
\item If $\Gamma\vdash e:\alpha$, then either $\Gamma\vdash_0 e:\alpha$, or there is a minimal derivation whose last step is a conversion $\Gamma\vdash\beta\equiv\alpha$ where $\Gamma\vdash_0e:\beta$.
\item If $\Gamma,\Delta\vdash_0e:\alpha$ and $FV(e)\subseteq\Gamma$, then $\Gamma\vdash_0e:\alpha$.
\item If $\Gamma\vdash_0x:\alpha$, then $(x:\alpha)\in\Gamma$.
\item If $\Gamma\vdash_0\U_\ell:\alpha$, then $\alpha=\U_{S\ell}$.
\item If $\Gamma\vdash_0 e_1\;e_2:\gamma$, then $\gamma=\beta[e_2/x]$, $\Gamma\vdash e_1:\forall x:\alpha.\;\beta$, and $\Gamma\vdash e_2:\alpha$ for some $\alpha,\beta,x$.
\item If $\Gamma\vdash_0 \lambda x:\alpha.\;e:\gamma$, then $\gamma=\forall x:\alpha.\;\beta$ and $\Gamma,x:\alpha\vdash_0 e:\beta$ for some $\alpha,\beta,x$.
\item If $\Gamma\vdash_0 \forall x:\alpha.\;e:\gamma$, then $\gamma=\U_{\imax(\ell_1,\ell_2)}$, $\Gamma\vdash \alpha:\U_{\ell_1}$, and $\Gamma,x:\alpha\vdash \beta:\U_{\ell_2}$ for some $\ell_1,\ell_2,\beta,x$.
\item If $\Gamma\vdash_0 \elet{x:\alpha:=e'}{e}:\beta$, then $\Gamma\vdash e':\alpha$ and $\Gamma\vdash_0 e[e'/x]:\beta$.
\item If $\Gamma\vdash_0 c_{\bar\ell}:\alpha$, then $\alpha=\tau_{\bar\ell}(c)$ (this includes inductive types and defined and axiomatic constants).
\end{enumerate}
\end{lemma}
\begin{proof}\ \\[-1em]
\begin{enumerate}
\item Trivial from the definition.
\item If the last rule was the conversion rule, then the first hypothesis $\Gamma\vdash e:\alpha'$ would have a shorter proof.
\item This follows from the fact that a proof can be weakened by adding $\Delta$ in every step of the proof without increasing the total length, so that a shorter proof of $\Gamma\vdash e:\alpha'$ could be weakened to a short proof of $\Gamma,\Delta\vdash e:\alpha'$.
\item[4-10.] By inversion, we obtain the conclusion in each case without the $\vdash_0$ on the inverted hypotheses. In (7), we know that the derivation $\Gamma,x:\alpha\vdash e:\beta$ is minimal because if $\Gamma,x:\alpha\vdash e:\beta'$ is shorter then we can derive $\Gamma\vdash \lambda x:\alpha.\;e:\forall x:\alpha.\;\beta'$ with a shorter proof, and in (9), if $\Gamma\vdash \elet{x:\alpha:=e'}{e}:\beta'$ is a shorter proof, then $\Gamma\vdash_0 \elet{x:\alpha:=e'}{e}:\beta'$ also has a shorter proof.
\end{enumerate}
\end{proof}

\begin{lemma}[Unique typing]
If $\Gamma\vdash e:\alpha$, then:
\begin{itemize}
\item Either $\Gamma\vdash_0 e:\alpha$, or there is a minimal derivation whose last step is a conversion $\Gamma\vdash\beta\equiv\alpha$ where $\Gamma\vdash_0e:\beta$;
\item If $\Gamma\vdash_0 e:\alpha_1$ and $\Gamma\vdash_0 e:\alpha_2$, then $\Gamma\vdash\alpha_1\equiv\alpha_2$.
\item If $\Gamma\vdash_0 e:\forall x:\beta.\;\alpha_1$ and $\Gamma\vdash_0 e:\forall x:\beta.\;\alpha_2$, then $\Gamma,x:\beta\vdash\alpha_1\equiv\alpha_2$.
\end{itemize}
\end{lemma}
\begin{proof}
By induction on the length of a minimal proof of $\Gamma\vdash e:\alpha$. Note that the two parts of the inductive hypothesis imply that if $\Gamma\vdash_0 e:\alpha'$ then $\Gamma\vdash\alpha'\equiv\alpha$.
\begin{itemize}
\item If the last rule is a base case rule (variable, universe, constant), then it has length 1 and so is minimal; in this case by the inversion lemma the latter part is trivially satisfied.
\item If the last rule is the conversion rule, from hypotheses $\Gamma\vdash e:\beta$ and $\Gamma\vdash\beta\equiv\alpha$, then the inductive hypothesis applies:
\begin{itemize}
\item If $\Gamma\vdash_0 e:\beta$, then the last step is a conversion from $\Gamma\vdash_0 e:\beta$.
\item If there is a minimal derivation of $\Gamma\vdash e:\beta$ via $\Gamma\vdash_0e:\gamma$, $\Gamma\vdash\gamma\equiv\beta$, then we may join the definitional equalities by transitivity to obtain a proof $\Gamma\vdash e:\alpha$ that has the same length as the proof of $\Gamma\vdash e:\beta$ (recall that proof steps for $\Gamma\vdash e\equiv e'$ don't count toward the step count).
\end{itemize}
The second clause is immediate from the IH.
\item If the last step is the application rule, say $\dfrac{\Gamma\vdash e_1:\forall x:\alpha.\;\beta\quad \Gamma\vdash e_2:\alpha}{\Gamma\vdash e_1\;e_2:\beta[e_2/x]}$,
consider also a proof $\dfrac{\Gamma\vdash e_1:\forall x:\alpha'.\;\beta'\quad\Gamma\vdash e_2:\alpha'}{\Gamma\vdash_0 e_1\;e_2:\beta'[e_2/x]}$ (by the inversion lemma). Since all four hypotheses are shorter than the original proof, the inductive hypothesis applies in each case.

We may assume WLOG that $\Gamma\vdash_0 e_2:\alpha'$, because if not it is obtained from conversion on some $\alpha''$, and then since $\forall x:\alpha''.\;\beta'\equiv\forall x:\alpha'.\;\beta'$ we may eliminate the conversion on the right and possibly introduce a conversion on the left for no net increase in the length of the overall proof.

Similarly, since by the inductive hypothesis on $\Gamma\vdash e_2:\alpha$ now $\alpha\equiv\alpha'$, we may arrange it so that the final step of the proof is $\dfrac{\Gamma\vdash e_1:\forall x:\alpha'.\;\beta\quad \Gamma\vdash_0 e_2:\alpha'}{\Gamma\vdash e_1\;e_2:\beta[e_2/x]}$.
By the inductive hypothesis on $\Gamma\vdash e_1:\forall x:\alpha'.\;\beta'$,
\begin{itemize}
\item If both derivations are 
\end{itemize} have $\gamma'\equiv\forall x:\alpha.\;\beta$ and $\alpha'\equiv\alpha$ such that $\Gamma\vdash_0 e_1:\gamma'$ and $\Gamma\vdash_0 e_2:\alpha'$.
\end{itemize}
UNFINISHED
\end{proof}
Now let $\vdash_0\Gamma\ok$ mean that all the types in the context are minimally derived, that is:
$$\frac{}{\vdash_0\cdot\ok}\qquad\frac{\vdash_0\Gamma\ok\quad\Gamma\vdash_0\alpha:\U_\ell}{}\qquad$$
Note that the final rule in such a minimal derivation cannot be the conversion rule, because it has a hypothesis that is a shorter typing of the same expression. Also, if the last rule is weakening, so $\Gamma,x:\beta\vdash_0e:\alpha$ derives from $\Gamma\vdash e:\alpha$, then also $\Gamma\vdash_0e:\alpha$ because any shorter derivation of $\Gamma\vdash e:\alpha'$ could be similarly weakened to a proof of $\Gamma,x:\beta\vdash e:\alpha'$.

UNFINISHED


\subsection{Sort inference}
Our first tool for ensuring that the typing and definitional equality relations are not trivial is to construct a parallel typing and definitional equality judgment that keeps track of a lot more information than the ``official'' one. This will make inductive proofs easier, and then we only have to show that any typing derivation can be ``upgraded'' in this way.
$$\Delta::=\cdot\mid\Delta,x:e:\U_\ell\qquad
\boxed{\Delta\vdash e:\alpha:\U_\ell}$$
$$\frac{\Delta\vdash\alpha:\U_\ell\type\quad\Delta\vdash e:\beta:\U_{\ell'}}{\Delta,x:\alpha:\U_\ell\vdash e:\beta:\U_{\ell'}}\qquad
\frac{\Delta\vdash\alpha:\U_\ell\type}{\Delta,x:\alpha:\U_\ell\vdash x:\alpha:\U_\ell}\qquad
\frac{}{\vdash\U_\ell:\U_{S\ell}\type}$$
$$\frac{\begin{matrix}
\Delta,x:\alpha:\U_{\ell_1}\vdash\beta:\U_{\ell_2}\type\\
\Delta\vdash e_1:\forall x:\alpha.\;\beta:\U_{\ell'}\quad\Delta\vdash e_2:\alpha:\U_{\ell_1}
\end{matrix}}{\Delta\vdash e_1\;e_2:\beta[e_2/x]:\U_{\ell_2}}\qquad
\frac{\Delta,x:\alpha:\U_{\ell_1}\vdash e:\beta:\U_{\ell_2}}{\Delta\vdash\lambda x:\alpha.\;e:\forall x:\alpha.\;\beta:\U_{\imax(\ell_1,\ell_2)}}$$
$$\frac{\Delta,x:\alpha:\U_{\ell_1}\vdash \beta:\U_{\ell_2}\type}{\Delta\vdash\forall x:\alpha.\;\beta:\U_{\imax(\ell_1,\ell_2)}\type}$$
$$\frac{\Delta\vdash e:\alpha:\U_\ell\quad \Delta\vdash \beta:\U_{\ell'}\type\quad \Delta\vdash\alpha\equiv\beta:\U_\ell\quad\ell\equiv\ell'}{\Delta\vdash e:\beta:\U_{\ell'}}$$
In these rules $\Delta\vdash \alpha:\U_\ell\type$ is defined to be $\Delta\vdash \alpha:\U_\ell:\U_{S\ell}$. We also introduce a definitional equality judgment with universe contexts, with a built-in typing of both sides as well:

$$\boxed{\Delta\vdash e\equiv e':\alpha}$$
$$\frac{\Delta\vdash e:\alpha:\U_\ell}{\Delta\vdash e\equiv e:\alpha}\qquad
\frac{\Delta\vdash e\equiv e':\alpha}{\Delta\vdash e'\equiv e:\alpha}\qquad
\frac{\Delta\vdash e_1\equiv e_2:\alpha\quad\Delta\vdash e_2\equiv e_3:\alpha}{\Delta\vdash e_1\equiv e_3:\alpha}$$
$$\frac{\ell\equiv\ell'}{\vdash \U_\ell\equiv\U_{\ell'}:\U_{S\ell}}\qquad
\frac{\Delta\vdash e_1\equiv e'_1:\forall x:\alpha.\;\beta\quad \Delta\vdash e_2\equiv e'_2:\alpha}{\Delta\vdash e_1\;e_2\equiv e'_1\;e'_2:\beta[e_2/x]}$$
$$\frac{\Delta\vdash\alpha\equiv \alpha':\U_{\ell_1}\quad \Delta,x:\alpha:\U_{\ell_1}\vdash e\equiv e':\beta}{\Delta\vdash\lambda x:\alpha.\;e\equiv \lambda x:\alpha'.\;e':\forall x:\alpha.\;\beta}$$
$$\frac{\Delta\vdash \alpha\equiv \alpha':\U_{\ell_1}\quad \Delta,x:\alpha:\U_{\ell_1}\vdash \beta\equiv \beta':\U_{\ell_2}}{\Delta\vdash \forall x:\alpha.\;\beta\equiv \forall x:\alpha'.\;\beta':\U_{\imax(\ell_1,\ell_2)}}$$
$$\frac{\Delta,x:\alpha:\U_\ell\vdash e:\beta:\U_{\ell'}\quad\Delta\vdash e':\alpha:\U_\ell}{\Delta\vdash (\lambda x:\alpha.\;e)\;e'\equiv e[e'/x]:\beta[e'/x]}\qquad
\frac{\Delta\vdash e:\forall y:\alpha.\;\beta:\U_\ell}{\Delta\vdash \lambda x:\alpha.\;e\;x\equiv e:\forall y:\alpha.\;\beta}$$
$$\frac{\Delta\vdash h:p:\P\quad \Delta\vdash h':p:\P}{\Delta \vdash h\equiv h':p}\qquad
\frac{\Delta\vdash e\equiv e':\alpha\quad \Delta\vdash\alpha\equiv\beta:\U_\ell}{\Delta\vdash e\equiv e':\beta}$$

Like $\Gamma\vdash e:\alpha$, we can define a simple context checking judgment:
$$\boxed{\vdash\Delta\ok}\qquad
\frac{}{\vdash\cdot\ok}\qquad
\frac{\Delta\vdash\alpha:\U_\ell\type}{\vdash\Delta,x:\alpha:\U_\ell\ok}$$

Let the notation $[\Delta]$ denotes the context obtained from $\Delta$ by omitting the universes:
$$[\cdot]=\cdot\qquad
[\Delta,x:\alpha:\U_\ell]=[\Delta],x:\alpha$$
And let $\Delta\equiv\Delta'$ mean that the contexts are equivalent except for changing types and levels to definitionally equal ones:
$$\boxed{\Delta\equiv\Delta'}\qquad\frac{}{\cdot\equiv\cdot}\qquad
\frac{\Delta\equiv\Delta'\quad\ell\equiv\ell'\quad\Delta\vdash\alpha\equiv\beta:\U_\ell}{\Delta,x:\alpha:\U_\ell\equiv\Delta',x:\beta:\U_{\ell'}}$$

\begin{lemma}[Substitution and regularity for sort inference]
\begin{enumerate}
\item If $\Delta\vdash e:\alpha:\U_\ell$, then $\vdash \Delta\ok$.
\item If $\Delta\vdash \alpha:\U_\ell\type$, then $\vdash \Delta\ok$.
\item If $\Delta,x:\alpha:\U_\ell\vdash e_1:\beta:\U_{\ell'}$ and $\Delta\vdash e_2:\alpha:\U_\ell$,\\
then $\Delta\vdash e_1[e_2/x]:\beta[e_2/x]:\U_{\ell'}$.
\item If $\Delta,x:\alpha:\U_\ell\vdash e_1\equiv e_1':\beta$ and $\Delta\vdash e_2:\alpha:\U_\ell$,\\
then $\Delta\vdash e_1[e_2/x]\equiv e_1'[e_2/x]:\beta[e_2/x]$.
\item If $\Delta,x:\alpha:\U_\ell\vdash e_1:\beta:\U_\ell$ and $\Delta\vdash e_2\equiv e_2':\alpha$,\\
then $\Delta\vdash e_1[e_2/x]\equiv e_1[e_2'/x]:\beta[e_2/x]$.
\item If $\Delta,x:\alpha:\U_\ell\vdash \beta:\U_{\ell'}\type$ and $\Delta\vdash e_2:\alpha:\U_\ell$, then $\Delta\vdash \beta[e_2/x]:\U_{\ell'}\type$.
\item If $\Delta\vdash e:\alpha:\U_\ell$, then $\Delta\vdash \alpha:\U_\ell\type$.
\item If $\Delta\equiv\Delta'$ then $\vdash\Delta\ok$.
\item $\Delta\equiv\Delta'$ is an equivalence relation.
\item If $\Delta\equiv\Delta'$ and $\Delta\vdash e:\alpha:\U_\ell$ then $\Delta'\vdash e:\alpha:\U_\ell$.
\item If $\Delta\equiv\Delta'$ and $\Delta\vdash e\equiv e':\alpha$ then $\Delta'\vdash e\equiv e':\alpha$.
\item If $\Delta\vdash e:\alpha:\U_\ell$, then $[\Delta]\vdash e:\alpha$.
\item If $\Delta\vdash e:\alpha:\U_\ell$, then $[\Delta]\vdash \alpha:\U_\ell$.
\item If $\Delta\vdash \alpha:\U_\ell\type$, then $[\Delta]\vdash \alpha:\U_\ell$.
\item If $\vdash\Delta\ok$, then $\vdash[\Delta]\ok$.
\item If $\Delta\equiv\Delta'$ then $[\Delta]\equiv[\Delta']$.
\end{enumerate}
\end{lemma}
\begin{proof}\ \\[-7mm]
\begin{itemize}
\item Part (1) an easy induction. Part (2) is a special case of (1).
\item (3) is by induction on the first premise, using the substitution property and (3) in the conversion rule.
\item (8) is a special case of (7).
\item (9) is an easy induction, using (8) in the application case.
\item (10) is by cases on $\Delta$.
\item UNFINISHED
\end{itemize}
\end{proof}

\begin{lemma}[Properties of the context truncation]
\begin{enumerate}
\item If $\Delta\vdash e:\alpha:\U_\ell$, then $[\Delta]\vdash e:\alpha$.
\item If $\Delta\vdash e:\alpha:\U_\ell$, then $[\Delta]\vdash \alpha:\U_\ell$.
\item If $\Delta\vdash \alpha:\U_\ell\type$, then $[\Delta]\vdash \alpha:\U_\ell$.
\item If $\vdash\Delta\ok$, then $\vdash[\Delta]\ok$.
\item If $\Delta\equiv\Delta'$ then $[\Delta]\equiv[\Delta']$.
\end{enumerate}
\end{lemma}
\begin{proof}\ \\[-7mm]
\begin{itemize}
\item (1) and (2) are proven together by induction. The substitution property is used in the application case.
\item (3) is a special case of (1), (4) is by cases using (3).
\end{itemize}
\end{proof}


\begin{lemma}[Sort inference]
\begin{enumerate}
\item If $\Gamma\vdash e:\alpha$, then $\Delta\vdash e:\alpha:\U_\ell$ for some $\Delta,\ell$ such that $[\Delta]=\Gamma$.
\item If $\Gamma\vdash \alpha:\U_\ell$, then $\Delta\vdash \alpha:\U_\ell\type$ for some $\Delta$ such that $[\Delta]=\Gamma$.
\item If $\Gamma\vdash e\equiv e'$, then $\Delta\vdash e\equiv e':\alpha$ for some $\Delta,\alpha$ such that $[\Delta]=\Gamma$.
\end{enumerate}
\end{lemma}
\begin{proof}
All parts are proven by mutual induction.
\begin{itemize}
\item UNFINISHED
\end{itemize}
\end{proof}
The key to why this judgment is relevant is the observation that the conversion rule cannot do anything interesting at the universe level. We want to use this to prove that $\Gamma\vdash\U_\ell\equiv\U_{\ell'}$ implies $\ell\equiv\ell'$.

UNFINISHED

\subsection{Unique typing}\label{sec:unique}
There are a large number of ``natural'' properties about the typing and definitional equality judgments we will want to be true in order to reason that certain judgments are not derivable for ``obvious'' reasons, for example that it is not possible to prove $\vdash\P:\P$ (which is a necessary condition for soundness).

\begin{theorem}[Unique typing]\label{thm:unique}
If $\Gamma\vdash e:\alpha$ and $\Gamma\vdash e:\beta$, then $\Gamma\vdash\alpha\equiv\beta$.
\end{theorem}

Unfortunately, we cannot yet prove this theorem. The critical step is the Church-Rosser theorem, which we will develop in the next section. However, we can set up the induction, which is necessary now since the Church-Rosser theorem will require that this theorem is true, and we will be caught in a circularity unless we are careful about the claims.

We will prove this theorem by induction on the number of alternations between the judgments $\Gamma\vdash e:\alpha$ and $\Gamma\vdash\alpha\equiv\beta$ (which are mutually recursive). Define $\Gamma\vdash_n e:\alpha$ and $\Gamma\vdash_n\alpha\equiv\beta$ by induction on $n\in\N$ as follows:
\begin{itemize}
\item $\Gamma\vdash_0\alpha\equiv\beta$ iff $\alpha=\beta$.
\item $\Gamma\vdash_{n+1}\alpha\equiv\beta$ iff there is a proof of $\Gamma\vdash\alpha\equiv\beta$ using  only $\Gamma\vdash_n e:\alpha$ typing judgments.
\item Assuming $\Gamma\vdash_m\alpha\equiv\beta$ is defined for $m\le n$, $\Gamma\vdash_n e:\alpha$ means that there is a proof of $\Gamma\vdash e:\alpha$ in which all appeals to the conversion rule use $\Gamma\vdash_m\alpha\equiv\beta$ for $m\le n$.
\end{itemize}
So if $\Gamma\vdash_0 e:\alpha$, then there is a proof that does not use the conversion rule at all; if $\Gamma\vdash_1\alpha\equiv\beta$ then there is a proof whose typing judgments do not use the conversion rule; if $\Gamma\vdash_1 e:\alpha$ then there is a proof using only the $1$-provable conversion rule; and so on. We will prove theorem \ref{thm:unique} by induction on this $n$.

\begin{lemma}[$n$-provability basics]
\begin{enumerate}
\item If $m\le n$ then $\Gamma\vdash_m e:\alpha$ implies $\Gamma\vdash_n e:\alpha$.
\item If $m\le n$ then $\Gamma\vdash_m\alpha\equiv\beta$ implies $\Gamma\vdash_n \alpha\equiv\beta$.
\item If $\Gamma\vdash e:\alpha$ then $\Gamma\vdash_n e:\alpha$ for some $n\in\N$.
\item If $\Gamma\vdash\alpha\equiv\beta$ then $\Gamma\vdash_n \alpha\equiv\beta$ for some $n\in\N$.
\end{enumerate}
\end{lemma}
\begin{proof}
(1) is immediate from the definition, (2) follows from (1). (3,4) are proven by a mutual induction on the typing judgment.
\end{proof}
\begin{definition}
Say that $\vdash_n$ has \emph{definitional inversion} if the following properties hold:
\begin{enumerate}
\item If $\Gamma\vdash_n x\equiv y$, then either $x=y$ or $\Gamma\vdash x,y:p$ for some $\Gamma\vdash p:\P$.
\item If $\Gamma\vdash_n \U_\ell\equiv\U_{\ell'}$, then $\ell\equiv\ell'$.
\item If $\Gamma\vdash_n \forall x:\alpha.\;e\equiv\forall x:\alpha'.\;e'$, then $\Gamma\vdash_n \alpha\equiv\alpha'$ and $\Gamma,x:\alpha\vdash_n e\equiv e'$.
\item If $\Gamma\vdash_n c_{\ell}\equiv c_{\ell'}$ (where $c$ is an axiomatic constant like $\mathsf{choice}$ or an inductive type or constructor, not a definition), then $\ell\equiv\ell'$, unless $\Gamma\vdash_n c_{\ell}:p$ for some $\Gamma\vdash_n p:\P$.
\item If $\Gamma\vdash_n \lambda x:\alpha.\;e\equiv\lambda x:\alpha'.\;e'$, then $\Gamma\vdash_n \alpha\equiv\alpha'$ and $\Gamma,x:\alpha\vdash_n e\equiv e'$.
\end{enumerate}
(We will also use the term \emph{unique typing} for this property given lemma \ref{thm:utype}.)
\end{definition}

\begin{theorem}[Unique typing]\label{thm:utype}
If $\vdash_n$ has definitional inversion, and $\Gamma\vdash_n e:\alpha$ and $\Gamma\vdash_n e:\beta$, then $\Gamma\vdash_n\alpha\equiv\beta$.
\end{theorem}
\begin{proof}
By the weakening lemma, we can use instead the judgment $\vdash'_n$ which has no weakening rule.

By induction on the proof of $\Gamma\vdash'_n e:\alpha$ with a secondary induction on $\Gamma\vdash'_n e:\beta$.
\begin{enumerate}
\item If $\Gamma\vdash'_n e:\alpha$ from the conversion rule on $\Gamma\vdash_n\alpha'\equiv\alpha$, $\Gamma\vdash_n e:\alpha'$, then $\Gamma\vdash_n\alpha'\equiv\beta$ by the IH, so $\Gamma\vdash_n\alpha\equiv\beta$ by transitivity. (Similarly if the conversion rule applies on $\Gamma\vdash'_n e:\beta$.)
\item Otherwise, the same typing rule applies in both derivations. The variable, universe, lambda, let, and constant cases are trivial.
\item In the forall case, we have $\Gamma\vdash_n\forall x:\alpha.\;\beta:\U_{\imax(\ell_1,\ell_2)},\U_{\imax(\ell_1',\ell_2')}$ from $\Gamma\vdash\alpha:\U_{\ell_1},\U_{\ell_1'}$ and $\Gamma\vdash\beta:\U_{\ell_2},\U_{\ell_2'}$, and from the inductive hypothesis $\Gamma\vdash_n\U_{\ell_1}\equiv\U_{\ell_1'}$. From definitional inversion, $\ell_1\equiv \ell_1'$ and $\ell_2\equiv \ell_2'$, so $\Gamma\vdash_n\U_{\imax(\ell_1,\ell_2)}\equiv\U_{\imax(\ell_1',\ell_2')}$.
\item In the application case, we have $\Gamma\vdash_n e_1\;e_2:\beta[e_2/x],\beta'[e_2/x]$ from $\Gamma\vdash_n e_1:\forall x:\alpha.\;\beta,\forall x:\alpha'.\;\beta'$ and $\Gamma\vdash_n e_2:\alpha,\alpha'$, and from the inductive hypothesis $\Gamma\vdash_n\forall x:\alpha.\;\beta\equiv\forall x:\alpha'.\;\beta'$. From definitional inversion, $\Gamma\vdash_n\alpha\equiv\alpha'$ and $\Gamma,x:\alpha\vdash_n\beta\equiv\beta'$, so $\Gamma\vdash_n\beta[e_2/x]\equiv\beta'[e_2/x]$.
\end{enumerate}
\end{proof}

Thus, it suffices to prove that $\vdash_n$ has definitional inversion for every $n$ to establish theorem \ref{thm:unique}. We can show the base case:
\begin{lemma}\label{thm:0dinv}
$\vdash_0$ has definitional inversion.
\end{lemma}
\begin{proof}
Since $\Gamma\vdash_0 e\equiv e'$ means $e=e'$, all cases are trivial by inversion on the construction of the term.
\end{proof}

\subsection{The Church-Rosser theorem}\label{sec:church_rosser}
\emph{Note: In this section, we will omit the indices from the provability relation, but we will focus on characterizing the $\equiv$ relation at a particular level. So read $\Gamma\vdash \alpha\equiv\beta$ as $\Gamma\vdash_{n+1} \alpha\equiv\beta$, and $\Gamma\vdash e:\alpha$ as $\Gamma\vdash_n e:\alpha$. Also (and importantly) we will assume that $\vdash_n$ has unique typing, which will prevent the appearance of certain pathologies.}

The standard formulation of the Church-Rosser theorem, when applied to the $\rightsquigarrow$ reduction relation, is not true; under reasonable definitions of reduction, Lean will not have unique normal forms, because of proof irrelevance. (We already saw how this plays out in section \ref{sec:undecidable}). All other substantive reduction rules act on terms the same way regardless of their types. To analyze this, we will split the definitional equality judgment into two parts: A $\beta\eta\delta\zeta\iota$-reduction relation (henceforth abbreviated $\kappa$ reduction), and a relation that does proof irrelevance. The idea is that $\kappa$ reduction satisfies a modified version of the Church-Rosser theorem, while proof irrelevance picks up the pieces, quantifying exactly how non-unique the normal form is.

The $\eta$ reduction relation can sometimes fight against the $\iota$ reduction in the sense that it is possible for a K-like eliminator to reduce in two ways, where the $\eta$ reduced form cannot reduce, for example with the following reductions, using $\rec_{a=}:\forall C.\;C\;a\to \forall b.\;a=b\to C\;b$:
\begin{align*}
&\lambda h:a=a.\;\rec_{a=}\;C\;e\;a\;h\rightsquigarrow_\eta\rec_{a=}\;C\;e\;a\\
&\lambda h:a=a.\;\rec_{a=}\;C\;e\;a\;h\rightsquigarrow_\iota\lambda h:a=a.\;e
\end{align*}

To resolve this, we will require that $\rec$ and $\lift$ always have their required number of parameters. To accomplish this, we will slightly modify the language to remove $\rec_P$ the function and replace it with a fully applied form $\rec_P(C,e,p,x)$. (We will use the notation $f(x)$ instead of $f\;x$ for these ``atomic applications''.) The transformation is as follows:
\begin{itemize}
\item If $e$ is a list of terms of length $n$ and $\rec_P$ has $m\ge n$ arguments, then $\overline{\rec_P\;e}=\lambda x::\alpha.\;\rec_P(e,x)$ where $x$ is the remaining $n-m$ arguments, with type $\alpha$ according to the specification of $P$.
\item If $e$ is a list of terms of length $n\le 6$ (note that $\lift$ has 6 arguments), then $\overline{\lift\;e}=\lambda x::\alpha.\;\lift(e,x)$ where $x$ is the remaining $6-n$ arguments.
\item Otherwise, the transformation is recursive in subterms: $\overline x=x$, $\overline{\lambda x:\alpha.\;e}=\lambda x:\overline{\alpha}.\;\overline{e}$, etc. 
\end{itemize}
The reverse translation maps $\underline{\rec_P(C,e,p,x)}=\rec_P\;C\;e\;p\;x$ and $\underline{\lift(\alpha,R,\beta,f,h,q)}=\lift\;\alpha\;R\;\beta\;f\;h\;q$ and leaves all other forms unchanged. The typing rules are extended in the obvious way to the new forms. The relevant properties of this transformation are:

\begin{lemma}[Properties of the $\rec$-normal form]
\begin{itemize}
\item If $\Gamma\vdash e:\alpha$ in the modified language, then $\underline{\Gamma}\vdash \underline{e}:\underline{\alpha}$ in the original language and $\overline{(\underline{e})}=e$.
\item If $\Gamma\vdash e:\alpha$ in the original language, then $\overline{\Gamma}\vdash \overline{e}:\overline{\alpha}$ and $\Gamma\vdash \underline{(\overline{e})}\equiv e$. (In fact, the proof of equivalence uses only $\eta$.)
\item If $\Gamma\vdash e\equiv e'$ in the modified language, then $\underline{\Gamma}\vdash \underline{e}\equiv\underline{e'}$.
\item If $\Gamma\vdash e\equiv e'$ in the original language, then $\overline{\Gamma}\vdash \overline{e}\equiv\overline{e'}$.
\end{itemize}
\end{lemma}

With this new language, the problem above is rejected on syntax grounds:
\begin{align*}
&\lambda h:a=a.\;\rec_{a=}(C,e,a,h)\rightsquigarrow_\iota\lambda h:a=a.\;e\\
&\lambda h:a=a.\;\rec_{a=}(C,e,a,h)\not\rightsquigarrow_\eta\rec_{a=}(C,e,a)
\end{align*}
where the second reduction doesn't work because $\rec_{a=}(C,e,a,h)$ is not an application, so the $\eta$ rule doesn't apply (and $\rec_{a=}(C,e,a)$ is not syntactically correct because it has the wrong number of arguments).

\begin{remark}
Rather than modifying the language, we could have instead modified the eta rule to not apply when the argument is a recursor with the correct number of arguments, but this approach of using application-like forms that are not applications seems clearer to the author.
\end{remark}

The $\kappa$ reduction relation is defined on the modified language, with compatibility rules such as these for every syntax operator (including $\rec_P(e)$ and $\lift(e)$):
$$\boxed{\Gamma\vdash e\rightsquigarrow_\kappa e'}\qquad
\frac{\Gamma\vdash e_1 \rightsquigarrow_\kappa e_1'}{\Gamma\vdash e_1\;e_2\rightsquigarrow_\kappa e_1'\;e_2}\qquad
\frac{\Gamma\vdash e_2 \rightsquigarrow_\kappa e_2'}{\Gamma\vdash e_1\;e_2\rightsquigarrow_\kappa e_1\;e_2'}$$
$$\frac{\Gamma\vdash \alpha \rightsquigarrow_\kappa\alpha'}{\Gamma\vdash \lambda x:\alpha.\;e\rightsquigarrow_\kappa \lambda x:\alpha'.\;e}\qquad
\frac{\Gamma,x:\alpha\vdash e \rightsquigarrow_\kappa e'}{\Gamma\vdash \lambda x:\alpha.\;e\rightsquigarrow_\kappa \lambda x:\alpha.\;e'}\qquad\dots$$
The substantive rules are:
$$(\beta)\ \frac{}{\Gamma\vdash (\lambda x:\alpha.\;e)\;e'\rightsquigarrow_\kappa e[e'/x]}\qquad
(\eta)\ \frac{\Gamma,x:\alpha\vdash h\equiv_p x}{\Gamma\vdash \lambda x:\alpha.\;e\;h\rightsquigarrow_\kappa e}$$
$$(\delta)\ \frac{\mathsf{def}\;c:\alpha:=e}{\Gamma\vdash c\rightsquigarrow_\kappa e}\qquad
(\zeta)\ \frac{}{\Gamma\vdash \elet{x:\alpha:=e'}{e}\rightsquigarrow_\kappa e[e'/x]}$$
$$(\iota)\ \frac{P\mbox{ is non-K inductive with ctor }c}{\Gamma\vdash \rec_P(C,e,p,c\;b)\rightsquigarrow_\kappa e_c\;b\;v}\qquad
(\iota_q)\ \frac{}{\Gamma\vdash \lift(R,f,h,\mk_R\;a)\rightsquigarrow_\kappa f\;a}$$
$$(K)\ \frac{P\mbox{ is K-like inductive}}{\Gamma\vdash \rec_P(C,e,p,h)\rightsquigarrow_\kappa e\;\inv[p,h]\;v}\qquad
(\iota_K)\ \frac{P\mbox{ is K-like inductive}}{\Gamma\vdash\inv_i\;(\intro\;b)\rightsquigarrow_\kappa b_i}$$

See section \ref{sec:inductive} for the variable names and types used in the regular $\iota$ rule.

We have an alternate $\iota$ rule for K-like inductives, where $\inv[p,h]$ is a sequence of terms such that $\intro\;\inv[p,h]\equiv h$ (by proof irrelevance) and $\inv_i[p,\intro\;b]\equiv b_i$. By the definition of a K-like inductive, every argument to the $\intro$ constructor is either propositional, or appears as one of the parameters $p_i$ to the inductive family. We define $\inv_i[p,h]:=p_j$ when the $i$th constructor argument is non-propositional and appears at position $j$ in the output type, and $\inv_i[p,h]=\inv_i\;h$ for the propositions, where $\inv_i$ is an atomic projection function. These $\inv_i$ projection operators can be defined in the original language using the recursor, like we demonstrated for $\acc$ in section \ref{sec:undecidable}, but we will treat them as constants of the modified language.

The proof irrelevance relation deals with all the ways that normal forms can fail to be unique. Specifically, this relation is responsible for changing universe levels and changing proofs.
$$\boxed{\Gamma\vdash e\equiv_p e'}$$
$$\frac{\Gamma\vdash e:\alpha}{\Gamma\vdash e\equiv_p e}\qquad
\frac{\Gamma\vdash\alpha\equiv_p\alpha'\quad \Gamma,x:\alpha\vdash e\equiv_p e'}{\Gamma\vdash \forall x:\alpha.\;e\equiv_p \forall x:\alpha'.\;e'}\qquad
\frac{\Gamma\vdash e_1\equiv_p e_1'\quad \Gamma\vdash e_2\equiv_p e_2'}{\Gamma\vdash e_1\;e_2\equiv_p e_1'\;e_2'}\qquad\dots$$
$$\frac{\ell\equiv\ell'}{\vdash \U_\ell\equiv_p\U_{\ell'}}\qquad
\frac{\Gamma\vdash\alpha\equiv\alpha'\quad \Gamma,x:\alpha\vdash e\equiv_p e'}{\Gamma\vdash \lambda x:\alpha.\;e\equiv_p \lambda x:\alpha'.\;e'}\qquad
\frac{\Gamma\vdash p:\P\quad \Gamma\vdash h:p\quad \Gamma\vdash h':p}{\Gamma \vdash h\equiv_p h'}$$
Note also that this relation allows full definitional equality in the domain of a lambda, that is, it requires that $\Gamma\vdash \alpha\equiv\alpha'$ instead of $\Gamma\vdash \alpha\equiv_p\alpha'$ like the other rules. To give a hint on why this is relevant, consider the following reduction sequence:

\begin{align*}
&\lambda x:\alpha.\;(\lambda y:\beta.\;e)\;x\rightsquigarrow_\eta\lambda y:\beta.\;e\\
&\lambda x:\alpha.\;(\lambda y:\beta.\;e)\;x\rightsquigarrow_\beta\lambda x:\alpha.\;e[x/y]=\lambda y:\alpha.\;e
\end{align*}
(We view $\alpha$-equivalent terms as syntactically equal.) By reducing the expression in two ways, we obtain the ``same'' expression $\lambda y.\;e$, but the type annotation is different, and although we know from the application rule that $\alpha\equiv\beta$, this equivalence may involve the full strength of the type system to prove. These two terms are, however, $\equiv_p$-equivalent, and this is close enough to equal for us to prove the main theorems.

\begin{lemma}[Regularity of reductions]
\begin{enumerate}
\item If $\Gamma\vdash e:\alpha$ and $\Gamma\vdash e\rightsquigarrow_\kappa e'$, then $\Gamma\vdash e\equiv e':\alpha$.
\item $\equiv_p$ is an equivalence relation.
\item If $\Gamma\vdash e\equiv_p e'$, then $\Gamma\vdash e\equiv e'$.
\item\label{item:p_subst} If $\Gamma,x:\alpha\vdash e_1\equiv_p e_1'$ and $\Gamma\vdash e_2\equiv_p e_2'$ then $\Gamma\vdash e_1[e_2/x]\equiv_p e_1'[e_2'/x]$.
\end{enumerate}
\end{lemma}
\begin{proof}
All parts are easy inductions.
\end{proof}
Note that the first part implies subject reduction for $\rightsquigarrow_\kappa$.

\begin{theorem}[Church-Rosser property]\label{thm:church_rosser}
If $\Gamma\vdash e:\alpha$, and $e\rightsquigarrow_\kappa^* e_1$ and $e\rightsquigarrow_\kappa^* e_2$, then there exists $e_1'$ and $e_2'$ such that $\Gamma\vdash e_1'\equiv_p e_2'$, and $e_1\rightsquigarrow_\kappa^* e_1'$ and $e_2\rightsquigarrow_\kappa^* e_2'$.
\end{theorem}
The proof follows the Tait--Martin-L\"{o}f method, extended to all the $\kappa$ rules. Define the parallel reduction $\gg_\kappa$ by the following rules:
$$\frac{}{\Gamma\vdash x\gg_\kappa x}\qquad
\frac{\Gamma\vdash \alpha\gg_\kappa\alpha'\quad \Gamma,x:\alpha\vdash e\gg_\kappa e'}{\Gamma\vdash \lambda x:\alpha.\;e\gg_\kappa \lambda x:\alpha'.\;e'}\qquad
\frac{\Gamma\vdash e_1\gg_\kappa e_1'\quad \Gamma\vdash e_2\gg_\kappa e_2'}{\Gamma\vdash e_1\;e_2\gg_\kappa e_1'\;e_2'}\qquad\dots$$
$$\frac{\Gamma,x:\alpha\vdash e_1\gg_\kappa e_1'\quad \Gamma\vdash e_2\gg_\kappa e_2'}{\Gamma\vdash (\lambda x:\alpha.\;e_1)\;e_2\gg_\kappa e_1'[e_2'/x]}\qquad
\frac{x\notin FV(e)\quad \Gamma,x:\alpha\vdash h\equiv_p x\quad \Gamma\vdash e\gg_\kappa e'}{\Gamma\vdash \lambda x:\alpha.\;e\;h\gg_\kappa e'}$$
$$\frac{\Gamma\vdash e_2[e_1/x]\gg_\kappa e'}{\Gamma\vdash \elet{x:\alpha:=e_1}{e_2}\gg_\kappa e'}\qquad
\frac{\mathsf{def}\;c:\alpha:=e\quad \Gamma\vdash e\gg_\kappa e'}{\Gamma\vdash c\gg_\kappa e'}$$
$$\frac{\Gamma\vdash f\gg_\kappa f'\quad \Gamma\vdash a\gg_\kappa a'}{\Gamma\vdash \lift(R,f,h,\mk_R\;a)\gg_\kappa f'\;a'}\qquad
\frac{\begin{matrix}
P\mbox{ is non-K inductive with ctor }c\\
\Gamma\vdash C,e,b,p\gg_\kappa C',e',b',p'
\end{matrix}}{\Gamma\vdash \rec_P(C,e,p,c\;b)\gg_\kappa e'\;b'\;v'}$$
$$\frac{\begin{matrix}
P\mbox{ is K-like inductive}\\
\Gamma\vdash C,e,p,h\gg_\kappa C',e',p',h'
\end{matrix}}{\Gamma\vdash \rec_P(C,e,p,h)\gg_\kappa e_c'\;\inv[p',h']\;v'}\qquad
\frac{\begin{matrix}
P\mbox{ is K-like inductive}\\
\Gamma\vdash b_i\gg_\kappa b_i'
\end{matrix}}{\Gamma\vdash \inv_i\;(\intro\;b)\gg_\kappa b_i'}$$

The ellipsis on the first line abbreviates compatibility rules for all the term constructors, recursing into all subterms like in the examples for lambda and application. All the substantive rules also follow a similar pattern: for each substantive rule in $\rightsquigarrow_\kappa$, there is a corresponding rule where after applying the $\rightsquigarrow_\kappa$ rule all variables on the RHS are $\gg_\kappa$ evaluated to the primed versions, and these are what end up in the target expression. (Note that in the $\iota$ rule, $v$ is a term that mentions $e$ and $p$; these are replaced by the primed versions in $v'$.)

In addition, we define the following ``complete reduction'' $\Gamma\vdash e\ggg_\kappa e'$ by exactly the same rules as $\gg_\kappa$, except that the compatibility rules only apply if none of the substantive rules are applicable. This makes $\ggg_\kappa$ almost deterministic (producing a unique $e'$ given $e$), except that the $\equiv_p$ hypothesis in the $\iota$ rule allows some freedom of choice of the parameters $b$.

It is easy to prove the following properties by induction:
\begin{lemma}[Properties of $\gg_\kappa$]\label{thm:gg_prop}
\begin{enumerate}
\item If $\Gamma\vdash e:\alpha$, then $\Gamma\vdash e\gg_\kappa e$.
\item\label{item:red_gg} If $\Gamma\vdash e\rightsquigarrow_\kappa e'$ then $\Gamma\vdash e\gg_\kappa e'$.
\item\label{item:gg_red} If $\Gamma\vdash e\gg_\kappa e'$ then $\Gamma\vdash e\rightsquigarrow_\kappa^* e'$.
\item\label{item:gg_subst} If $\Gamma,x:\alpha\vdash e_1\gg_\kappa e_1'$ and $\Gamma\vdash e_2\gg_\kappa e_2'$ (where $\Gamma\vdash e_2:\alpha$) then\\ $\Gamma\vdash e_1[e_2/x]\gg_\kappa e_1'[e_2'/x]$.
\item\label{item:ggg_gg} If $\Gamma\vdash e\ggg_\kappa e'$, then $\Gamma\vdash e\gg_\kappa e'$.
\item\label{item:ggg_ex} If $\Gamma\vdash e:\alpha$, then $\Gamma\vdash e\ggg_\kappa e'$ for some $e'$.
\end{enumerate}
\end{lemma}

\begin{lemma}[Compatibility of $\gg_\kappa$ with $\equiv_p$]\label{thm:gg_compat}
If $\Gamma\vdash e_1\equiv_p e_3\gg_\kappa e_2$, then there exists $e_4$ such that $\Gamma\vdash e_1\gg_\kappa e_4\equiv_p e_2$.
\end{lemma}
\begin{proof}
By induction on $e_1\equiv_p e_3$ and inversion on $e_3\gg_\kappa e_2$. (We will omit the contexts from the relations.)
\begin{itemize}
\item If $e_1\equiv_p e_3=e_1$ by the reflexivity rule, then $e_1\gg_\kappa e_2\equiv_p e_2$.
\item If $e_1\equiv_p e_3$ by the proof irrelevance rule, then $e_3:p:\P$, so $e_2:p:\P$ as well and hence $e_1\gg_\kappa e_1\equiv_p e_2$.
\item If $e_1\equiv_p e_3$ and $e_3\gg_\kappa e_2$ both use the same compatibility rule, then it is immediate from the induction hypothesis.
\item If $e_1:p:\P$ is a proof, then $e_1\gg_\kappa e_1\equiv_p e_2$. (We will thus assume that $e_1$ is not a proof in later cases.)
\item If $\lambda x:\alpha_1.\;e_1\equiv_p\lambda x:\alpha_3.\;e_3\gg_\kappa\lambda x:\alpha_2.\;e_2$ by the lambda compatibility rule, then $\alpha_1\equiv\alpha_3\gg_\kappa\alpha_2$ and $e_1\equiv_p e_3\gg_\kappa e_2$, and by the IH we have $e_1\gg_\kappa e_4\equiv_p e_2$, so $\lambda x:\alpha_1.\;e_1\gg_\kappa\lambda x:\alpha_3.\;e_4\equiv_p\lambda x:\alpha_2.\;e_2$.
\item If $(\lambda x:\alpha_1.\;e_1)\;e_1'\equiv_p(\lambda x:\alpha_3.\;e_3)\;e_3'\gg_\kappa e_2[e_2'/x]$ where $e_1\equiv_p e_2\gg_\kappa e_3$, $e_1'\equiv_p e_2'\gg_\kappa e_3'$ and $\alpha_1\equiv\alpha_3$, then $(\lambda x:\alpha_1.\;e_1)\;e_1'\gg_\kappa e_1[e_1'/x]\equiv_p e_2[e_2'/x]$. (Other cases are similar, when the $\equiv_p$ is proven by compatibility rules and the $\gg_\kappa$ is a substantive rule.)
\item If $\lambda x:\alpha_1.\;e_1\;h_1\equiv_p\lambda x:\alpha_3.\;e_3\;h_3\gg_\kappa e_2$ where $e_1\equiv_p e_3\gg_\kappa e_2$ and $h_1\equiv_p h_3\equiv_p x$, then by the IH $e_1\gg_\kappa e_4\equiv_p e_2$ so that $\lambda x:\alpha_1.\;e_1\;h_1\gg_\kappa e_4\equiv_p e_2$.
\item If $\lift(R_1,\beta_1,f_1,h_1,q_1)\equiv_p\lift(R_3,\beta_3,f_3,h_3,\mk_R\;a_3)$ where $q_1\equiv_p \mk_R\;a_3$ by proof irrelevance, then $\beta:\P$ so $e_1:\beta$ is a proof. (Note: we are using that $\vdash_n$ has unique typing here.)
\item If $\rec_P(C_1,e_1,p_1,h_1)\equiv_p\rec_P(C_3,e_3,p_3,c\;b_3)\gg_\kappa (e_2)_c\;b_2\;v_2$ where $P$ is non-K inductive and $h_1\equiv_p c\;b_3$ by proof irrelevance, it is a small eliminator, so $\rec_P(C_1,e_1,p_1,h_1)$ is a proof.
\item If $e_1\equiv_p\inv_i\;(\intro\;b_3)$, then $e_1$ is a proof, since the $\inv_i$ function is only defined at propositional arguments.
\end{itemize}
\end{proof}
\begin{lemma}[Triangle lemma]\label{thm:tri}
If $\Gamma\vdash e:\alpha$, $e\gg_\kappa e'$, and $e\ggg_\kappa e^\bullet$, then there exists $e^\circ$ such that $\Gamma\vdash e'\gg_\kappa e^\circ\equiv_p e^\bullet$.
\end{lemma}
\begin{proof}
By induction on $e\ggg_\kappa e^\bullet$ and inversion on $e\gg_\kappa e'$.
\begin{itemize}
\item If $x\lll_\kappa x\gg_\kappa x$, then $x\gg_\kappa x\equiv_p x$.
\item If $e\ggg_\kappa e^\bullet$ by the beta rule:
\begin{itemize}
\item If $e_1^\bullet[e_2^\bullet/x]\lll_\kappa(\lambda x:\alpha.\;e_1)\;e_2\gg_\kappa e_1'[e_2'/x]$ by the beta rule, then $e_1'\gg_\kappa e_1^\circ$ and $e_2'\gg_\kappa e_2^\circ$ by the inductive hypothesis, so $e_1'[e_2'/x]\gg_\kappa e_1^\circ[e_2^\circ/x]\equiv_p e_1^\bullet[e_2^\bullet/x]$ by the substitution property.
\item If $e_1^\bullet[e_2^\bullet/x]\lll_\kappa(\lambda x:\alpha.\;e_1)\;e_2\gg_\kappa (\lambda x:\alpha.\;e_1')\;e_2'$ by the application rule and lambda rule, then $(\lambda x:\alpha.\;e_1')\;e_2'\gg_\kappa e_1^\circ[e_2^\circ/x]\equiv_p e_1^\bullet[e_2^\bullet/x]$ by the beta rule for $\gg_\kappa$ and the IH.
\item If $e_1^\bullet\;h[e_2^\bullet/x]\lll_\kappa(\lambda x:\alpha.\;e_1\;h)\;e_2\gg_\kappa e_1'\;e_2'$ by the application rule and eta rule, then $e_1'\;e_2'\gg_\kappa e_1^\circ\;e_2^\circ\equiv_p e_1^\bullet\;h[e_2^\bullet/x]$ because $e_2^\circ\equiv_p e_2^\bullet=x[e_2^\bullet/x]\equiv_p h[e_2^\bullet/x]$.
\end{itemize}
\item If $e\ggg_\kappa e^\bullet$ by the eta rule:
\begin{itemize}
\item If $e^\bullet\lll_\kappa\lambda x:\alpha.\;e\;h\gg_\kappa e'$ by the eta rule, then $e'\gg_\kappa e^\circ\equiv_p e^\bullet$.
\item If $e^\bullet\lll_\kappa\lambda x:\alpha.\;e\;h\gg_\kappa \lambda x:\alpha'.\;e'\;h'$ by the lambda rule and application rule, then $\lambda x:\alpha'.\;e'\;h'\gg_\kappa e^\circ\equiv_p e^\bullet$.
\item If $\lambda y:\beta^\bullet.\;e^\bullet\lll_\kappa\lambda x:\alpha.\;(\lambda y:\beta.\;e)\;h\gg_\kappa \lambda x:\alpha'.\;e'[h/y]$ by the lambda rule and beta rule, then $\lambda x:\alpha'.\;e'[h/y]\gg_\kappa \lambda x:\alpha^\circ.\;e^\circ[h/y]\equiv_p$\\$\lambda x:\beta^\bullet.\;e^\bullet[x/y]=\lambda y:\beta^\bullet.\;e^\bullet$. (Here we have used the $\lambda$ rule for $\equiv_p$ to equate $\alpha^\circ\equiv\beta^\bullet$, which are def.eq. because $\alpha$ and $\beta$ are.)
\end{itemize}
\item If $e^\bullet\lll_\kappa c\gg_\kappa e'$ by the delta rule, then $e'\gg_\kappa e^\circ\equiv_p e^\bullet$.
\item If $e\ggg_\kappa e^\bullet$ by the zeta rule:
\begin{itemize}
\item If $e_1^\bullet[e_2^\bullet/x]\lll_\kappa\elet{x:\alpha:=e_1}{e_2}\gg_\kappa e_2'[e_1'/x]$ by the zeta rule, then $e_1'[e_2'/x]\gg_\kappa e_1^\circ[e_2^\circ/x]\equiv_p e_1^\bullet[e_2^\bullet/x]$.
\item If $e_1^\bullet[e_2^\bullet/x]\lll_\kappa\elet{x:\alpha:=e_1}{e_2}\gg_\kappa \elet{x:\alpha':=e_1'}{e_2'}$ by the let compatibility rule, then $\elet{x:\alpha':=e_1'}{e_2'}\gg_\kappa e_1^\circ[e_2^\circ/x]\equiv_p e_1^\bullet[e_2^\bullet/x]$ by the zeta rule.
\end{itemize}
\item If $e\ggg_\kappa e^\bullet$ by the non-K inductive iota rule:
\begin{itemize}
\item If $e_c^\bullet\;b^\bullet\;v^\bullet\lll_\kappa \rec_P(C,e,p,c\;b)\gg_\kappa e_c'\;b'\;v'$ by the iota rule, then $e_c'\;b'\;v'\gg_\kappa e_c^\circ\;b^\circ\;v^\circ\equiv_p e_c^\bullet\;b^\bullet\;v^\bullet$.
\item If $e_c^\bullet\;b^\bullet\;v^\bullet\lll_\kappa \rec_P(C,e,p,c\;b)\gg_\kappa \rec_P(C',e',p',c\;b')$ by the $\rec_P$ compatibility rule, then $\rec_P(C',e',p',c\;b')\gg_\kappa e_c^\circ\;b^\circ\;v^\circ\equiv_p e_c^\bullet\;b^\bullet\;v^\bullet$ by the iota rule.
\end{itemize}
\item If $e\ggg_\kappa e^\bullet$ by the quotient iota rule:
\begin{itemize}
\item If $f^\bullet\;a^\bullet\lll_\kappa \lift(R,f,h,\mk_R\;a)\gg_\kappa f'\;a'$ by the iota rule, then $f'\;a'\gg_\kappa f^\circ\;a^\circ\equiv_p f^\bullet\;a^\bullet$.
\item If $f^\bullet\;a^\bullet\lll_\kappa \lift(R,f,h,\mk_R\;a)\gg_\kappa\lift(R',f',h',\mk_{R'}\;a')$ by the $\lift$ compatibility rule, then we have $q'\gg_\kappa q^\circ\equiv_p q^\bullet$ for $q\in\{C,e,p,h\}$, and $\lift(R',f',h',\mk_{R'}\;a')\gg_\kappa f^\circ\;a^\circ\equiv_p f^\bullet\;a^\bullet$.
\end{itemize}
\item If $e\ggg_\kappa e^\bullet$ by the K rule:
\begin{itemize}
\item If $e_c^\bullet\;\inv[p^\bullet,h^\bullet]\;v^\bullet\lll_\kappa \rec_P(C,e,p,h)\gg_\kappa e_c'\;\inv[p',h']\;v'$ by the iota rule, then $e_c'\;\inv[p',h']\;v'\gg_\kappa e_c^\circ\;\inv[p^\circ,h^\circ]\;v^\circ\equiv_p e_c^\bullet\;\inv[p^\bullet,h^\bullet]\;v^\bullet$.
\item If $e_c^\bullet\;\inv[p^\bullet,h^\bullet]\;v^\bullet\lll_\kappa \rec_P(C,e,p,h)\gg_\kappa \rec_P(C',e',p',h')$ by the $\rec_P$ compatibility rule, then $\rec_P(C',e',p',h')\gg_\kappa e_c^\circ\;\inv[p^\circ,h^\circ]\;v^\circ\equiv_p e_c^\bullet\;\inv[p^\bullet,h^\bullet]\;v^\bullet$ by the iota rule.
\end{itemize}
\item If $e\ggg_\kappa e^\bullet$ by the K-iota rule:
\begin{itemize}
\item If $b_i^\bullet\lll_\kappa \inv_i\;(c\;b)\gg_\kappa b_i'$ by the iota rule, then $b_i'\gg_\kappa b_i^\circ\equiv_p b_i^\bullet$ by the IH.
\item If $b_i^\bullet\lll_\kappa \inv_i\;(c\;b)\gg_\kappa \inv_i\;(c\;b')$ by compatibility rules, then $\inv_i\;(c\;b')\gg_\kappa b_i^\circ\equiv_p b_i^\bullet$ by the K-iota rule.
\end{itemize}
\item If $e\ggg_\kappa e^\bullet$ by a compatibility rule:
\begin{itemize}
\item If $e_1^\bullet\;e_2^\bullet\lll_\kappa e_1\;e_2\gg_\kappa e_1'\;e_2'$ by the application rule, then $e_1'\;e_2'\gg_\kappa e_1^\circ\;e_2^\circ\equiv_p e_1^\bullet\;e_2^\bullet$.
\item If $\forall x:\alpha^\bullet.\;e^\bullet\lll_\kappa \forall x:\alpha.\;e\gg_\kappa \forall x:\alpha'.\;e'$ by the forall rule, then $\forall x:\alpha'.\;e'\gg_\kappa \forall x:\alpha^\circ.\;e^\circ\equiv_p \forall x:\alpha^\bullet.\;e^\bullet$.
\item Other compatibility rules follow the same pattern.
\end{itemize}
\end{itemize}
\end{proof}
The main proof of Church-Rosser is a corollary of lemma \ref{thm:tri}, and does not differ substantially from the usual proof putting diamonds together, because the additional complication of having $\equiv_p$ at the bottom of the diamond commutes with all the other reductions.
\begin{proof}[of theorem \ref{thm:church_rosser}]
We prove in succession the following theorems:
\begin{enumerate}
\item If $\Gamma\vdash e:\alpha$, and $e_1\ll_\kappa e\gg_\kappa e_2$, then $\exists e_1'\;e_2'.\; e_1\gg_\kappa e_1'\equiv_p e_2'\ll_\kappa e_2$.
\item If $\Gamma\vdash e:\alpha$, and $e_1\ll_\kappa^* e\gg_\kappa e_2$, then $\exists e_1'\;e_2'.\; e_1\gg_\kappa e_1'\equiv_p e_2'\ll_\kappa^* e_2$.
\item If $\Gamma\vdash e:\alpha$, and $e_1\ll_\kappa^* e\gg_\kappa^* e_2$, then $\exists e_1'\;e_2'.\; e_1\gg_\kappa^* e_1'\equiv_p e_2'\ll_\kappa^* e_2$.
\item If $\Gamma\vdash e:\alpha$, and $e_1\leftsquigarrow_\kappa^*e\rightsquigarrow_\kappa^* e_2$, then $\exists e_1'\;e_2'.\; e_1\rightsquigarrow_\kappa^* e_1'\equiv_p e_2'\leftsquigarrow_\kappa^* e_2$.
\end{enumerate}
(4) is the theorem we want.
\begin{enumerate}
\item If $e\gg_\kappa e_1,e_2$, then by lemma \ref{thm:tri} there exists $e_1',e_2'$ such that $e_i\gg_\kappa e_i'$ and $e_1'\equiv_p e^\bullet\equiv_p e_2'$. But then $e_1'\equiv_p e_2'$ are as desired. 
\item By induction on $e\gg_\kappa^* e_1$. If $e\gg_\kappa^* e_1\gg_\kappa e_3$ and we have inductively that $e_1\gg_\kappa e_1'\equiv_p e_2'\ll_\kappa^* e_2$, then by applying (1) to $e_3\ll_\kappa e_1\gg_\kappa e_1'$ we obtain $e_3\gg_\kappa e_3'\equiv_p e_1''\ll_\kappa e_1'$, and by lemma \ref{thm:gg_compat} applied to $e_1''\ll_\kappa e_1'\equiv_p e_2'$ we obtain $e_1''\equiv_p e_2''\ll_\kappa e_2'$ so that  $e_3\gg_\kappa e_3'\equiv_p e_1''\equiv_p e_2''\ll_\kappa e_2'\ll_\kappa^* e_2$.
\item By induction on $e\gg_\kappa^* e_2$. The proof is the same as (2), replacing lemma \ref{thm:gg_compat} for the analogous statement for $\gg_\kappa^*$, i.e. if $e_1\gg_\kappa^*e_2$ and $e_1\equiv_p e_1'$ then there exists $e_2'$ such that $e_1'\gg_\kappa^*e_2'\equiv_p e_1'$. This follows by induction on lemma \ref{thm:gg_compat}.
\item  The equivalence of (3) and (4) comes from properties \ref{item:red_gg} and \ref{item:gg_red} of lemma \ref{thm:gg_prop}.
\end{enumerate}
\end{proof}

Now say that $\Gamma\vdash e_1\equiv_\kappa e_2$ if $\Gamma\vdash e_1,e_2:\alpha$ for some $\alpha$, and there exists $e_1',e_2'$ such that $\Gamma\vdash e_1\rightsquigarrow_\kappa^* e_1'\equiv_p e_2'\leftsquigarrow_\kappa^* e_2$. This relation is obviously reflexive and symmetric and implies $\Gamma\vdash e_1\equiv e_2$, and the Church-Rosser property implies it is also transitive.

\begin{theorem}[Completeness of the $\kappa$ reduction]\label{thm:ckappa}
$\Gamma\vdash e\equiv e'$ if and only if $\Gamma\vdash e\equiv_\kappa e'$ (in the modified language).
\end{theorem}
\begin{proof}
The reverse direction follows from regularity lemmas observed above. The forward direction is by induction on $\equiv$.
\begin{itemize}
\item The equivalence relation rules are immediate since $\equiv_\kappa$ is an equivalence relation (by the Church-Rosser property).
\item For the compatibility rules, since both $\equiv_p$ and $\rightsquigarrow_\kappa$ have compatibility rules, this property passes to $\equiv_\kappa$. Thus, for example in the lambda case, we have $\Gamma\vdash\lambda x:\alpha.\;e\equiv_\kappa\lambda x:\alpha.\;e'$ since $\Gamma,x:\alpha\vdash e\equiv_\kappa e'$ from the IH, and similarly $\Gamma\vdash\lambda x:\alpha.\;e'\equiv_\kappa\lambda x:\alpha'.\;e'$, so by transitivity $\Gamma\vdash\lambda x:\alpha.\;e\equiv_\kappa\lambda x:\alpha'.\;e'$.
\item The universe changing rules (for constants and $\U_\ell$) are in $\equiv_p$.
\item The $\beta$ and $\eta$ rules are in $\rightsquigarrow_\kappa$, and the proof irrelevance rule is in $\equiv_p$. All the other equivalence rules are also introduced in $\rightsquigarrow_\kappa$.
\item For K-like eliminators, we must show $\rec_P(C,e,p,\intro\;b)\equiv_\kappa e\;b\;v$. From the K rule we have $\rec_P(C,e,p,\intro\;b)\equiv_\kappa e\;\inv[p,c\;b]\;v$ so it suffices to show $\inv_i[p,\intro\;b]\equiv_\kappa b_i$ for each $i$. If $b_i$ is propositional then this is by proof irrelevance, otherwise $\inv_i[p,\intro\;b]=p_j$, and the well-typedness of $\rec_P(C,e,p,\intro\;b)$ implies that $\Gamma\vdash_n b_i\equiv p_j$. Thus by completeness of the $\kappa$ reduction at $\vdash_n$, $\Gamma\vdash_n b_i\equiv_\kappa p_j$ and hence $\Gamma\vdash_{n+1} b_i\equiv_\kappa p_j$.
\end{itemize}
\end{proof}

Now we can finally finish the inductive step of the proof of theorem \ref{thm:unique}:

\begin{theorem}[Definitional inversion]\label{thm:1dinv}
$\vdash_{n+1}$ has definitional inversion.
\end{theorem}
\begin{proof}
In each case we apply theorem \ref{thm:ckappa} on the assumptions.
\begin{enumerate}
\item\emph{If $\Gamma\vdash_{n+1} x\equiv y$, then $x=y$, unless $x$ is a proof.}

There are no $\rightsquigarrow_\kappa$ reductions from a variable, so if $\Gamma\vdash_{n+1} x\equiv_\kappa y$ then $\Gamma\vdash_{n+1} x\equiv_p y$. By inversion, either the variable compatibility rule is used, so $x=y$, or $\Gamma\vdash_{n+1} x\equiv_p y$ by proof irrelevance, so that $\Gamma\vdash x,y:p$ for some $\Gamma\vdash p:\P$.

\item \emph{If $\Gamma\vdash_{n+1} \U_\ell\equiv\U_{\ell'}$, then $\ell\equiv\ell'$.}

Again, there are no $\rightsquigarrow_\kappa$ reductions from $\U_\ell$, so $\Gamma\vdash_{n+1} \U_\ell\equiv_p\U_{\ell'}$, and if the compatibility rule is used then $\ell\equiv\ell'$. If proof irrelevance is used, then $\Gamma\vdash_n\U_\ell,\U_{\ell'}:p$ for some $\Gamma\vdash_n p:\P$. Since $\Gamma\vdash_n\U_\ell:\U_{S\ell}:\U_{SS\ell}$ as well, by unique typing at $n$, $\Gamma\vdash_n \P\equiv \U_{SS\ell}$, so by definitional inversion $0\equiv SS\ell$, a contradiction.

\item \emph{If $\Gamma\vdash_{n+1} \forall x:\alpha.\;e\equiv\forall x:\alpha'.\;e'$, then $\Gamma\vdash_{n+1} \alpha\equiv\alpha'$ and $\Gamma,x:\alpha\vdash_{n+1} e\equiv e'$.}

In this case, there are no $\rightsquigarrow_\kappa$ reductions except the compatibility rules, so $\forall x:\alpha.\;e\rightsquigarrow_\kappa^*\forall x:\alpha_1.\;e_1$ for some $\alpha\rightsquigarrow_\kappa^*\alpha_1$ and $e\rightsquigarrow_\kappa^*e_1$, and similarly $\alpha'\rightsquigarrow_\kappa^*\alpha'_1$ and $e'\rightsquigarrow_\kappa^*e'_1$, and if these are $\equiv_p$ equivalent using the compatibility rule then we are done.

If $\Gamma\vdash_{n+1}\forall x:\alpha_1.\;e_1\equiv_p\forall x:\alpha'_1.\;e'_1$ by proof irrelevance, then $\Gamma\vdash_n\forall x:\alpha_1.\;e_1,\forall x:\alpha'_1.\;e'_1:p:\P$. But $\Gamma\vdash_n\forall x:\alpha_1.\;e_1:\U_{\imax(\ell_1,\ell_2)}$ for some $\ell_1,\ell_2$ since $\alpha_1$ and $e_1$ are well-typed, so by unique typing at $n$ $p\equiv\U_{\imax(\ell_1,\ell_2)}$ and $0\equiv S\imax(\ell_1,\ell_2)$, a contradiction.

\item \emph{If $\Gamma\vdash_{n+1} c_{\ell}\equiv c_{\ell'}$, where $c$ is an axiomatic constant, then $\ell\equiv\ell'$, unless $c_\ell$ is a proof.}

Same as the variable case. There are no reductions from an axiomatic constant, so $\Gamma\vdash_{n+1} c_{\ell}\equiv_p c_{\ell'}$; then if the compatibility rule is used we have $\ell\equiv\ell'$, and if proof irrelevance is used then $c_{\ell}$ is a proof.

\item \emph{If $\Gamma\vdash_{n+1} \lambda x:\alpha.\;e\equiv\lambda x:\alpha'.\;e'$, then $\Gamma\vdash_{n+1} \alpha\equiv\alpha'$ and $\Gamma,x:\alpha\vdash_{n+1} e\equiv e'$.}

Since $\Gamma\vdash_n\lambda x:\alpha.\;e:\forall x:\alpha.\;\beta$ and $\Gamma\vdash_n\lambda x:\alpha'.\;e':\forall x:\alpha'.\;\beta'$, by unique typing and definitional inversion at $n$ we have $\Gamma\vdash_n\alpha\equiv\alpha'$.

Thus, $\Gamma,x:\alpha\vdash_{n+1} e\equiv(\lambda x:\alpha.\;e)\;x\equiv(\lambda x:\alpha'.\;e')\;x\equiv e'$ by the beta rule and application rule.
\end{enumerate}
\end{proof}

We've already described the structure of this theorem in earlier parts, but now we are finally ready to put all the parts together:

\begin{proof}[of theorem \ref{thm:unique}]
We prove by induction on $n$ that $\vdash_n$ has definitional inversion (and hence unique typing, by theorem \ref{thm:utype}), and also that it satisfies the conclusion of theorem \ref{thm:ckappa}.
\begin{itemize}
\item For $n=0$, $\vdash_0$ has definitional inversion by lemma \ref{thm:0dinv}, and theorem \ref{thm:ckappa} is trivial (where both $\Gamma\vdash e\equiv_\kappa e'$ and $\Gamma\vdash e\equiv e'$ mean $e=e'$).
\item For $n+1$, suppose $\vdash_n$ has definitional inversion and satisfies theorem \ref{thm:ckappa}. Then all the results of section \ref{sec:church_rosser} follow, including theorem \ref{thm:ckappa}. Then definitional inversion at $n+1$ is theorem \ref{thm:1dinv}.
\end{itemize}
\end{proof}

%\section{Reduction of inductive types to $\W$-types}

Given the complicated structure involved in simply stating the axioms of inductive types, one may wonder if there is an easier way. In fact there is; we can replace the whole structure of inductive types with a few simple inductive type constructors.

\subsection{The menagerie}
The most well known general form of our kind of inductive type is the $\W$-type, defined when $\Gamma\vdash A:\U_\ell$ and $\Gamma,x:A\vdash B:\U_\ell$:
$$\W x:A.\;B:=\mu w:\U_\ell.\;(\mathsf{sup}:\forall x:A.\;(B\to w)\to w)$$
This carries most of the ``power'' of inductive types, but we still need some glue to be able to reduce everything else to this. First, note that most of the telescopes $x::\alpha$ in an inductive type can be replaced by $\Sigma(x::\alpha)$, where $\Sigma ():=\bf 1$ and $\Sigma (x:\alpha,y::\beta):=\Sigma x:\alpha,\Sigma(y::\beta)$. This just packs up all the types in the telescope into one dependent tuple. Similarly, we want the types $\bf 0$ and $\alpha+\beta$ to pack up all the constructors into one.

To localize the universe management we will have a ``universe lift'' function $\ulift_u^v:\U_u\to\U_v$, defined when $u\le v$, as well as the $\nonempty$ operation (also known as the propositional truncation $\|\alpha\|$) to construct small eliminators. All the other type operators above will have the smallest possible universe level.

Finally, to handle inductive families and subsingleton eliminators, we will need the equality and $\acc$ types discussed previously. Here are the rules for these types:
\begin{align*}
e::=\dots&\mid\bot\mid\Sigma x:e.\;e\mid e+e\mid \ulift_{\ell}^{\ell}\;e\mid \|e\|\mid \mathsf{W} x:e.\;e\mid e=e\mid \acc_e\\
&\mid\rec_\bot\mid(e,e)\mid\pi_1\;e\mid\pi_2\;e\mid \inl e\mid \inr e\mid \rec_+\;e\;e\mid {\uparrow}e\mid {\downarrow}e\\
&\mid |e|\mid\rec_{||}\;e\mid\sup e\;e\mid\rec_\W\;e\mid \refl\;e\mid \rec_=\;e\;e\mid \intro_\acc\;e\;e\mid\rec_\acc\;e
\end{align*}
$$\frac{}{\vdash \bot:\P}\qquad
\frac{1\le\ell\quad \Gamma\vdash C:\U_\ell}{\Gamma\vdash \rec_\bot:\bot\to C}$$
$$\frac{\Gamma\vdash \alpha:\U_\ell\quad \Gamma,x:\alpha\vdash \beta:\U_{\ell'}}
{\Gamma\vdash \Sigma x:\alpha.\;\beta:\U_{\max(\ell,\ell',1)}}\qquad
\frac{\Gamma\vdash \alpha:\U_\ell\quad \Gamma\vdash \beta:\U_{\ell'}}
{\Gamma\vdash \alpha+\beta:\U_{\max(\ell,\ell',1)}}$$
$$\frac{\Gamma\vdash e_1:\alpha\quad \Gamma\vdash e_2:\beta\;e_1}
{\Gamma\vdash(e_1,e_2):\Sigma x:\alpha.\;\beta}\qquad
\frac{\Gamma\vdash p:\Sigma x:\alpha.\;\beta}{\Gamma\vdash \pi_1\;p:\alpha}\qquad
\frac{\Gamma\vdash p:\Sigma x:\alpha.\;\beta}
{\Gamma\vdash \pi_2\;p:\beta[\pi_1\;p/x]}$$
$$\frac{\Gamma\vdash \beta\type\quad \Gamma\vdash e:\alpha}{\Gamma\vdash \inl e:\alpha+\beta}\qquad
\frac{\Gamma\vdash \alpha\type\quad \Gamma\vdash e:\beta}{\Gamma\vdash \inr e:\alpha+\beta}$$
$$\frac{1\le\ell\quad \Gamma\vdash C:\alpha+\beta\to\U_\ell\quad \Gamma\vdash a:\forall x:\alpha.\;C\;(\inl x)\quad \Gamma\vdash b:\forall x:\beta.\;C\;(\inr x)}{\Gamma\vdash \rec_+\;a\;b:\forall p:\alpha+\beta.\;C\;p}$$
$$\frac{\Gamma\vdash \alpha:\U_\ell\quad \max(1,\ell)\le\ell'}
{\Gamma\vdash\ulift_{\ell}^{\ell'}\;\alpha:\U_{\ell'}}\qquad
\frac{\Gamma\vdash\ulift_{\ell}^{\ell'}\;\alpha:\U_{\ell'}\quad \Gamma\vdash e:\alpha}{\Gamma\vdash {\uparrow}e:\ulift_{\ell}^{\ell'}\;\alpha}\qquad
\frac{\Gamma\vdash e:\ulift_{\ell}^{\ell'}\;\alpha}{\Gamma\vdash {\downarrow}e:\alpha}$$
$$\frac{\Gamma\vdash \alpha\type}
{\Gamma\vdash\|\alpha\|:\P}\qquad
\frac{\Gamma\vdash e:\alpha}{\Gamma\vdash |e|:\|\alpha\|}\qquad
\frac{\Gamma\vdash C:\P\quad \Gamma\vdash f:\alpha\to C}{\Gamma\vdash \rec_{||}\;f:\|\alpha\|\to C}$$
$$\frac{\Gamma\vdash \alpha:\U_\ell\quad \Gamma,x:\alpha\vdash\beta:\U_{\ell'}}
{\Gamma\vdash\W x:\alpha.\;\beta:\U_{\max(\ell,\ell',1)}}\qquad
\frac{\Gamma\vdash a:\alpha\quad\Gamma\vdash f:\beta[a/x]\to \W x:\alpha.\;\beta}{\Gamma\vdash\sup a\;f:\W x:\alpha.\;\beta}$$
$$\frac{\begin{matrix}
1\le\ell\quad \Gamma\vdash C:(\W x:\alpha.\;\beta)\to\U_\ell\\
\Gamma\vdash e:\forall (a:\alpha)\;(f:\beta[a/x]\to\W x:\alpha.\;\beta).\;(\forall b:\beta[a/x].\;C\;(f\;b))\to C\;(\sup a\;f)
\end{matrix}}{\Gamma\vdash \rec_\W\;e:\forall w:(\W x:\alpha.\;\beta).\;C\;w}$$
$$\frac{\Gamma\vdash a:\alpha\quad\Gamma\vdash b:\alpha}
{\Gamma\vdash a=b:\P}\qquad
\frac{\Gamma\vdash a:\alpha}{\Gamma\vdash \refl\;a:a=a}$$
$$\frac{\Gamma\vdash a:\alpha\quad 1\le\ell\quad \Gamma\vdash C:\alpha\to\U_\ell\quad\Gamma\vdash e:C\;a}{\Gamma\vdash \rec_=\;e:\forall b:\alpha.\;a=b\to C\;b}$$
$$\frac{\Gamma\vdash r:\alpha\to\alpha\to\P}{\Gamma\vdash \acc_r:\alpha\to\P}\qquad
\frac{\Gamma\vdash x:\alpha\quad\Gamma\vdash f:\forall y:\alpha.\;r\;y\;x\to\acc_r\;y}{\Gamma\vdash \intro_\acc\;x\;f:\acc_r\;x}$$
$$\frac{\begin{matrix}
1\le\ell\quad \Gamma\vdash C:\alpha\to\U_\ell\\
\Gamma\vdash e:\forall x:\alpha.\; (\forall y:\alpha.\;r\;y\;x\to\acc_r\;y)\to (\forall y:\alpha.\;r\;y\;x\to C\;y)\to C\;x
\end{matrix}}{\Gamma\vdash \rec_\acc\;e:\forall x:\alpha.\;\acc_r\;x\to C\;x}$$
All of these could have been defined as inductive types in the sense of \autoref{sec:inductive}:
\begin{align*}
\bot&:=\mu t:\P.\;0\\
\Sigma x:\alpha.\;\beta&:=\mu t:\U_{\max(\ell,\ell',1)}.\;(\mathsf{pair}:\forall x:\alpha.\;\beta\to t)\\
\alpha+\beta&:=\mu t:\U_{\max(\ell,\ell',1)}.\;(\mathsf{inl}:\alpha\to t)+(\mathsf{inr}:\beta\to t)\\
\ulift_{\ell}^{\ell'}\;\alpha&:=\mu t:\U_{\ell'}.\;(\mathsf{up}:\alpha\to t)\\
\|\alpha\|&:=\mu t:\P.\;(\intro:\alpha\to t)\\
\W x:\alpha.\;\beta&:=\mu t:\U_{\max(\ell,\ell',1)}.\;(\mathsf{sup}:\forall x:\alpha.\;(\beta\to t)\to t)\\
a=b&:=(\mu t:\alpha\to\P.\;(\refl:t\;a))\;b\\
\acc_r&:=\mu t:\alpha\to\P.\;(\intro:\forall x:\alpha.\;(\forall y:\alpha.\;r\;y\;x\to t\;y)\to t\;x)
\end{align*}
However, we are interested in taking them as primitive in this section and deriving general inductive types. All of the new operators have compatibility rules for $\equiv$ and $\Leftrightarrow$; we will not belabor this as they all look roughly the same: when all the parts are equivalent, so is the whole. For example:
$$\frac{\Gamma\vdash\alpha\equiv\alpha'\quad\Gamma,x:\alpha\vdash\beta\equiv\beta'}{\Gamma\vdash\Sigma x:\alpha.\;\beta\equiv\Sigma x:\alpha'.\;\beta'}$$

Since we will need to handle $\P$ specially in the proof of soundness, we have simplified all the large eliminating recursors to require $1\le\ell$. The general recursor can be constructed from this by using $C':=\lambda x:P.\;\ulift_{\ell}^{\max(1,\ell)}\;(C\;x)$ (for each such inductive type $P$).

In a few of the constructors, additional parameters are elided, such as $C$ in $\rec_\bot$; one should imagine that each constructor is sufficiently annotated to ensure unique typing. Following their interpretation as inductive types, they also come with the following $\iota$ rules:
\begin{align*}
\pi_1(a,b)&\equiv a\\
\pi_2(a,b)&\equiv b\\
\rec_+\;a\;b\;(\inl x)&\equiv a\;x\\
\rec_+\;a\;b\;(\inr x)&\equiv b\;x\\
{\downarrow\uparrow} x&\equiv x\\
\rec_\W\;e\;(\sup a\;f)&\equiv e\;a\;f\;(\lambda b:\beta[a/x].\;\rec_\W\;e\;(f\;b))\\
\rec_=\;e\;a\;h&\equiv e\\
\rec_\acc\;e\;x\;(\intro_\acc\;x\;f)&\equiv e\;x\;f\;(\lambda (y:\alpha)\;(h:r\;y\;x).\;\rec_\acc\;e\;y\;(f\;y\;h))
\end{align*}
which are valid in any context that typechecks everything on the LHS.

Here are a few additional type operators that can be defined from the ones given:
$${\bf 0}_\ell:=\ulift^\ell\;\bot\qquad\top:=\bot\to\bot\qquad{\bf 1}_\ell:=\ulift^\ell\;\top\qquad \alpha\times \beta:=\Sigma\_:\alpha.\;\beta$$
$$p\land q:=\|p\times q\|\qquad p\lor q:=\|p+q\|$$
$$\{x:\alpha\mid p\}:=\Sigma x:\alpha.\;p\qquad
\exists x:\alpha.\; p:=\|\{x:\alpha\mid p\}\|$$

The following additional ``$\eta$ rules'' are needed for the reduction, which are provable but not definitional equalities in Lean. Since we are going for soundness only, we will help ourselves to this modest strengthening of the system; moreover this is only for convenience -- without such $\eta$ rules we would only be able to go as far as indexed $\W$-types, which are more complex. (These rules are also required for this axiomatization since we've omitted the recursors in favor of projections for $\Sigma$ and $\ulift$.)
$${\uparrow\downarrow} x\equiv x\qquad (\pi_1\;x,\pi_2\;x)\equiv x$$
The results of \autoref{sec:unique} apply straightforwardly to this setting, with these two rules added as $\rightsquigarrow_\kappa$ reduction rules along with all the $\iota$ rules mentioned above.
%
\subsection{Translating type families}
Let us first suppose that the inductive family lives in a universe $1\le\ell$. In this case we don't have to worry about $\P$ and small elimination. The idea is to eliminate families by first erasing the indices to get a ``skeleton'' type $S$ that mixes all the different members of the family together, and then separately define a predicate $\mathsf{good}:S\to\forall x::\alpha.\P$ that carves out the members that actually belong to index $x$. The final result will be the type $\lambda x::\alpha.\;\{s:S\mid\mathsf{good}\;s\;x\}$. For example, the type
$$X=\mu t:\N\to\U_1.\;(\mathsf{one}:t\;1)+(\mathsf{double}:\forall n:\N.\;t\;n\to t\;(2n))$$
has the indices erased to get
$$S=\mu t:\U_1.\;(\mathsf{one}:t)+(\mathsf{double}:\forall n:\N.\;t\to t),$$
and then the predicate is defined by recursion on $S$:
\begin{align*}
\mathsf{good}\;\mathsf{one}\;m&:=m=1\\
\mathsf{good}\;(\mathsf{double}\;n\;x)\;m&:=m=2n\land \mathsf{good}\;x\;n
\end{align*}
Now $S$ will also be reduced to the $\W$-type:
$$S'=\W x:{\bf 1}+\N.\;\rec_+\;(\lambda\_.\;{\bf 0})\;(\lambda n.\;{\bf 1})$$
because there are two branches, one with no non-recursive arguments and one with a non-recursive argument of type $\N$ (hence ${\bf 1}+\N$), and first branch has no recursive arguments and the second has one.

So the general translation will take the form
$$P\;x\simeq\{s:\W p:A.\;B\;p\mid \rec_\W\;(\lambda (p:A)\; \_.\;G\;p)\;s\;x\},$$\vspace{-7mm}
\begin{align*}
\mbox{where}\qquad\Gamma&\vdash A:\U_\ell\\
\Gamma&\vdash B:A\to\U_\ell\\
\Gamma&\vdash G:\forall p:A.\;(B\;p\to\forall x::\alpha.\;\P)\to\forall x::\alpha.\;\P.
\end{align*}

We will construct these three terms recursively based on the derivation of the $\spec$ judgment.
$$\boxed{\Gamma;t:F\vdash K\spec\Rightarrow A;B;G}$$
$$\frac{1\le\ell\quad \Gamma\vdash x::\alpha}{\Gamma;t:\forall x::\alpha.\;\U_\ell\vdash 0\spec\Rightarrow {\bf 0};\rec_0;\rec_0}$$
$$\frac{\Gamma;t:F\vdash \beta\ctor\Rightarrow A_1;p.B_1;pgx.G_1\quad \Gamma;t:F\vdash K\spec\Rightarrow A;B;G}{\Gamma;t:F\vdash (c:\beta)+K\spec\Rightarrow A_1+A;\rec_+\;(\lambda p.B_1)\;B;\rec_+\;(\lambda pg\;(x::\alpha).\;G_1)\;G}$$
$$\boxed{\Gamma;t:F\vdash \beta\ctor\Rightarrow A;p.B;pgx.G}$$
$$\frac{\Gamma\vdash e::\alpha}{\Gamma;t:\forall x::\alpha.\;\U_\ell\vdash t\;e\ctor\Rightarrow {\bf 1}_\ell;\ p.\;{\bf 0}_\ell;\ pgx.\;x=e}$$
$$\frac{\Gamma\vdash \beta:\U_{\ell'}\quad \ell'\le\ell\quad \Gamma,y:\beta;t:\forall x::\alpha.\;\U_\ell\vdash\tau\ctor\Rightarrow A;p.B;pgx.G}{
\begin{array}{l}
\Gamma;t:\forall x::\alpha.\;\U_\ell\vdash\forall y:\beta.\;\tau\ctor\Rightarrow\Sigma y':\beta.\;A[y'/y];\\
\qquad p'.B[\pi_1\;p'/y][\pi_2\;p'/p];\ p'gx.G[\pi_1\;p'/y][\pi_2\;p'/p]
\end{array}}$$
$$\frac{\begin{matrix}
\Gamma\vdash \gamma::\U_{\ell'}\quad \Gamma,z::\gamma\vdash e::\alpha\quad \ell'_i\le\ell\\
\Gamma;t:\forall x::\alpha.\;\U_\ell\vdash\tau\ctor\Rightarrow A;p.B;pg'x.G
\end{matrix}}{
\begin{array}{l}
\Gamma;t:\forall x::\alpha.\;\U_\ell\vdash(\forall z::\gamma.\;t\;e)\to\tau\ctor\Rightarrow A;\ p.\;\Sigma(z::\gamma)+B;\\
\qquad pgx.\;G[\lambda b.\;g\;(\inr b)/g']\land \forall z::\gamma.\;g\;(\inl (z))\;e
\end{array}}$$
In the final rule, the notation $(z)$ where $z::\gamma$ means the tuple of elements of $z$ of type $\Sigma(z::\gamma)$: explicitly, $(z_1,\dots,z_n)=(z_1,(z_2,\dots,(x_n,()))):\Sigma(z::\gamma)$.
Note that in the base case of $\mathsf{ctor}$, we have $x=e$ where $x$ and $e$ are telescopes; this can be defined as $(x)=(e)$, or using heterogeneous equality $x_1=e_1\land x_2==e_2\land \dots\land x_n==e_n$, or using the equality recursor $\exists (h_1:x_1=e_1)\;(h_2:\rec_=\;x_2\;x_1\;h_1=e_2)\dots$. We will use $(x)=(e)$ since it is the least notationally burdensome of these options.

The final result is given by the following translation:
$$\frac{\Gamma;t:F\vdash K\spec\Rightarrow A;B;G}
{\Gamma\vdash\scott{\mu t:F.\;K}=\lambda x::\alpha.\;\{s:\W p:A.\;B\;p\mid \rec_\W\;(\lambda (p:A)\; \_.\;G\;p)\;s\;x\}}$$
In the case of a small eliminator, we just artificially lift the target universe above 1, translate it, and then propositionally truncate the resulting type and lift if back to the original universe $\ell$:
$$\frac{\Gamma;t:F\vdash K\spec\quad \neg(\Gamma;t:\forall x::\alpha.\;\U_\ell\vdash K\LE)}
{\Gamma\vdash\scott{\mu t:\forall x::\alpha.\;\U_\ell.\;K}=\lambda x::\alpha.\;\ulift^\ell\|\scott{\mu t:\forall x::\alpha.\;\U_{\ell'}.\;K}\;x\|},$$
where $\ell'$ is the maximum of $1$ and all the constructor arguments. The idea here is that since we have a small eliminator, it's impossible to tell that members of the inductive type are distinct, so we lose nothing in the propositional truncation.

\subsection{Translating subsingleton eliminators}
The hard case is when we have a subsingleton eliminator. In this case we must abandon $\W$-types entirely, since we have to produce a subsingleton family from the start -- propositional truncation will destroy the large elimination property, so we have to use $\acc$ instead. The zero case is easy:
$$\frac{\Gamma\vdash x::\alpha}
{\Gamma\vdash\scott{\mu t:\forall x::\alpha.\;\U_\ell.\;0}=\lambda x::\alpha.\;{\bf 0}_\ell}$$

For our purposes it will be easier to work with the following variant on $\acc$:
\begin{align*}
&\alpha:\U_\ell,\vph:\alpha\to\P,r:\alpha\to\alpha\to\P\vdash\acc^\vph_r=\mu t:\alpha\to\P.\\
&\qquad\;(\intro:\forall x:\alpha.\;\vph\;x\to(\forall y:\alpha.\;r\;y\;x\to t\;y)\to t\;x)
\end{align*}

This is just the same as $\acc_r$ but for the additional parameter $\vph$ that restricts the satisfying instances. This can be built from plain $\acc$ in our existing axiomatization as follows:

\begin{align*}
\acc^\vph_r\;x&:=\exists h:\vph\;x.\;\acc_{r'}\;(x,h)\\
\mbox{where}\qquad r'&:=\lambda x\;x':\{x:\alpha\mid \vph\;x\}.\;r\;(\pi_1\;x)\;(\pi_1\;x')
\end{align*}
Large elimination for $\acc^\vph_r$ is derivable because $\exists h:p.\;q$ has projections when $p$ is a proposition.

In the translation, we must pack up the family into a single type and then use $\acc$ for the recursive instances. Let us run an example first:
$$P=\mu t:\N\to\N\to\P.\;(\intro:\forall n:\N.\;n>2\to (\forall m.\;m<n\to t\;n\;m)\to t\;0\;n)$$
This is a large eliminating type because of the constructor's three arguments, one appears in the result ($t\;0\;n$), one is a proposition ($n>2$), and one is recursive ($\forall m.\;m<n\to t\;n\;m$).

First we pack the domain into a sigma type, in this case $\N\times\N$, and the propositional constraints go into $\vph$. The recursive arguments become the edge relation for $\acc$. Here, $(a,b)$ is accessible when there exists an $n$ such that $(a,b)=(0,n)$, $n>2$ and for all $m<n$, $(n,m)$ is accessible, so we translate this to $\vph(a,b)$ iff there exists $n$ such that $(a,b)=(0,n)$ and $n>2$, and $r\;(a',b')\;(a,b)$ iff there exists $m,n$ such that $(a,b)=(0,n)$ and $m<n$ and $(a',b')=(n,m)$.

In both clauses we introduce a variable $n$ equal to $b$ or $b'$, and this variable can be eliminated. This is true generally because of the restriction on large eliminators: every non-propositional nonrecursive argument, like $n$ here, must appear in the output type, yielding a variable-variable equality $n=b$ which can be used to eliminate $n$. However, due to potential dependencies on earlier arguments, we will delay this elimination to the recursor. So in this translation we have:
\begin{align*}
P\;x\simeq\acc_r^\vph\;(x)\qquad\mbox{where}\qquad\Gamma&\vdash \vph:=\lambda p:\Sigma(x::\alpha).\; B[p/(x)]\\
\Gamma&\vdash r:=\lambda p\;q:\Sigma(x::\alpha).\;R[p/(x')][q/(x)]\\
\Gamma,x::\alpha&\vdash B:\P\\
\Gamma,x'::\alpha,x::\alpha&\vdash R:\P
\end{align*}
where we must specify the definition of $B$ and $R$ inductively with the displayed free variables. Here the notation $B[p/(x)]$ means to replace each $x_i$ with the appropriate projection $\pi_1(\pi_2^i\;p)$ in $B$. We will also accumulate an auxiliary $\Gamma,x'::\alpha,x::\alpha\vdash S:\P$ for constructing the disjunctions in $R$.

$$\boxed{\Gamma;t:F\vdash \tau\LEctor\Rightarrow x.B;x'x.[S;R]}$$
$$\frac{}{\Gamma;t:F\vdash t\;e\LEctor\Rightarrow x.\;x=e;\ x'x.\;[x=e;\bot]}$$
$$\frac{\Gamma,t:F\vdash \beta:\U_\ell\quad \Gamma,y:\beta;t:F\vdash\tau\LEctor\Rightarrow x.B;x'x.[S;R]}
{\Gamma;t:F\vdash\forall y:\beta.\;\tau\LEctor\Rightarrow x.\exists y:\beta.\;B;x'x.[\exists y:\beta.\;S;\exists y:\beta.\;R]}$$
$$\frac{\Gamma;t:F\vdash\beta\LEctor\Rightarrow x.B;x'x.[S;R]}{\Gamma;t:F\vdash(\forall z::\gamma.\;t\;e)\to\beta\LEctor\Rightarrow x.B;x'x.[S;(S\land\exists z::\gamma.\;x'=e)\lor R]}$$
Intuitively, $S$ collects the facts that are true about the main instance argument $x$, so that in each recursive constructor we push a conjunction of $S$ with the fact $\exists z::\gamma.\;x'=e$ we need to hold for $x'$. Since we do the same thing for propositional and index arguments (just existentially generalize everything), we have collapsed both into one rule. Once we have constructed the term, we have the following rule:

$$\frac{\Gamma,t:\forall x::\alpha.\;\U_\ell\vdash \beta\LEctor\Rightarrow x.B;x'x.[S;R]}
{\Gamma\vdash\scott{\mu t:\forall x::\alpha.\;\U_\ell.\;(c:\beta)}=\lambda x::\alpha.\;\ulift^\ell(\acc^\vph_r(x))}$$
\vspace{-5mm}
\begin{align*}
\mbox{where, as before, }\vph&:=\lambda p:\Sigma(x::\alpha).\; B[p/(x)]\\
\mbox{and }r&:=\lambda p\;q:\Sigma(x::\alpha).\;R[p/(x')][q/(x)].
\end{align*}

\subsection{The remainder}
We have described the translation of a recursive type in great detail, but it still remains to define the introduction rules and the recursor, and show that the iota rule holds definitionally with these definitions. As these are more or less uniquely determined by the translated type of the inductive type itself, and it is yet more cumbersome than what has been thus far written, this will be left as future work, probably as part of a formalization of all of this. For now, we will proceed with the understanding that the eight inductive types $\bot,\Sigma,+,\ulift,\|\cdot\|,\W,=,\acc$ are indeed sufficient to cover all Lean-definable inductive types, and leave all this horrible induction behind.

\section{Unique typing}\label{sec:unique}
There are a large number of ``natural'' properties about the typing and definitional equality judgments we will want to be true in order to reason that certain judgments are not derivable for ``obvious'' reasons, for example that it is not possible to prove $\vdash\P:\P$ (which is a necessary condition for soundness).

\begin{theorem}[Unique typing]\label{thm:unique}
If $\Gamma\vdash e:\alpha$ and $\Gamma\vdash e:\beta$, then $\Gamma\vdash\alpha\equiv\beta$.
\end{theorem}

Unfortunately, we cannot yet prove this theorem. The critical step is the Church-Rosser theorem, which we will develop in the next section. However, we can set up the induction, which is necessary now since the Church-Rosser theorem will require that this theorem is true, and we will be caught in a circularity unless we are careful about the claims.

We will prove this theorem by induction on the number of alternations between the judgments $\Gamma\vdash e:\alpha$ and $\Gamma\vdash\alpha\equiv\beta$ (which are mutually recursive). Define $\Gamma\vdash_n e:\alpha$ and $\Gamma\vdash_n\alpha\equiv\beta$ by induction on $n\in\N$ as follows:
\begin{itemize}
\item $\Gamma\vdash_0\alpha\equiv\beta$ iff $\alpha=\beta$.
\item $\Gamma\vdash_{n+1}\alpha\equiv\beta$ iff there is a proof of $\Gamma\vdash\alpha\equiv\beta$ using  only $\Gamma\vdash_n e:\alpha$ typing judgments.
\item Assuming $\Gamma\vdash_m\alpha\equiv\beta$ is defined for $m\le n$, $\Gamma\vdash_n e:\alpha$ means that there is a proof of $\Gamma\vdash e:\alpha$ in which all appeals to the conversion rule use $\Gamma\vdash_m\alpha\equiv\beta$ for $m\le n$.
\end{itemize}
So if $\Gamma\vdash_0 e:\alpha$, then there is a proof that does not use the conversion rule at all; if $\Gamma\vdash_1\alpha\equiv\beta$ then there is a proof whose typing judgments do not use the conversion rule; if $\Gamma\vdash_1 e:\alpha$ then there is a proof using only the $1$-provable conversion rule; and so on. We will prove \autoref{thm:unique} by induction on this $n$.

\begin{lemma}[$n$-provability basics]
\begin{thmlist}
\item If $m\le n$ then $\Gamma\vdash_m e:\alpha$ implies $\Gamma\vdash_n e:\alpha$.
\item If $m\le n$ then $\Gamma\vdash_m\alpha\equiv\beta$ implies $\Gamma\vdash_n \alpha\equiv\beta$.
\item If $\Gamma\vdash e:\alpha$ then $\Gamma\vdash_n e:\alpha$ for some $n\in\N$.
\item If $\Gamma\vdash\alpha\equiv\beta$ then $\Gamma\vdash_n \alpha\equiv\beta$ for some $n\in\N$.
\end{thmlist}
\end{lemma}
\begin{proof}
(1) is immediate from the definition, (2) follows from (1). (3,4) are proven by a mutual induction on the typing judgment.
\end{proof}
\begin{definition}\label{def:dinv}
Say that $\vdash_n$ has \emph{definitional inversion} if the following properties hold:
\begin{enumerate}
\item If $\Gamma\vdash_n \U_\ell\equiv\U_{\ell'}$, then $\ell\equiv\ell'$.
\item If $\Gamma\vdash_n \forall x:\alpha.\;\beta\equiv\forall x:\alpha'.\;\beta'$, then $\Gamma\vdash_n \alpha\equiv\alpha'$ and $\Gamma,x:\alpha\vdash_n \beta\equiv \beta'$.
\item $\Gamma\vdash_n \U_\ell\not\equiv\forall x:\alpha.\;\beta$.
\end{enumerate}
(We will also use the term \emph{unique typing} for this property given \autoref{thm:utype}.)
\end{definition}
There are other inversions along these lines, but distinguishing universes and foralls is the most important part and it is what we need for the induction.

\begin{theorem}[Unique typing]\label{thm:utype}
If $\vdash_n$ has definitional inversion, and $\Gamma\vdash_n e:\alpha$ and $\Gamma\vdash_n e:\beta$, then $\Gamma\vdash_n\alpha\equiv\beta$.
\end{theorem}
\begin{proof}
By the weakening lemma, we can use instead the judgment $\vdash'_n$ which has no weakening rule.

By induction on the proof of $\Gamma\vdash'_n e:\alpha$ with a secondary induction on $\Gamma\vdash'_n e:\beta$.
\begin{enumerate}
\item If $\Gamma\vdash'_n e:\alpha$ from the conversion rule on $\Gamma\vdash_n\alpha'\equiv\alpha$, $\Gamma\vdash_n e:\alpha'$, then $\Gamma\vdash_n\alpha'\equiv\beta$ by the IH, so $\Gamma\vdash_n\alpha\equiv\beta$ by transitivity. (Similarly if the conversion rule applies on $\Gamma\vdash'_n e:\beta$.)
\item Otherwise, the same typing rule applies in both derivations. The variable, universe, lambda, let, and constant cases are trivial.
\item In the forall case, we have $\Gamma\vdash_n\forall x:\alpha.\;\beta:\U_{\imax(\ell_1,\ell_2)},\U_{\imax(\ell_1',\ell_2')}$ from $\Gamma\vdash\alpha:\U_{\ell_1},\U_{\ell_1'}$ and $\Gamma\vdash\beta:\U_{\ell_2},\U_{\ell_2'}$, and from the inductive hypothesis $\Gamma\vdash_n\U_{\ell_1}\equiv\U_{\ell_1'}$. From definitional inversion, $\ell_1\equiv \ell_1'$ and $\ell_2\equiv \ell_2'$, so $\Gamma\vdash_n\U_{\imax(\ell_1,\ell_2)}\equiv\U_{\imax(\ell_1',\ell_2')}$.
\item In the application case, we have $\Gamma\vdash_n e_1\;e_2:\beta[e_2/x],\beta'[e_2/x]$ from $\Gamma\vdash_n e_1:\forall x:\alpha.\;\beta,\forall x:\alpha'.\;\beta'$ and $\Gamma\vdash_n e_2:\alpha,\alpha'$, and from the inductive hypothesis $\Gamma\vdash_n\forall x:\alpha.\;\beta\equiv\forall x:\alpha'.\;\beta'$. From definitional inversion, $\Gamma\vdash_n\alpha\equiv\alpha'$ and $\Gamma,x:\alpha\vdash_n\beta\equiv\beta'$, so $\Gamma\vdash_n\beta[e_2/x]\equiv\beta'[e_2/x]$.
\end{enumerate}
\end{proof}

Thus, it suffices to prove that $\vdash_n$ has definitional inversion for every $n$ to establish \autoref{thm:unique}. We can show the base case:
\begin{lemma}\label{thm:0dinv}
$\vdash_0$ has definitional inversion.
\end{lemma}
\begin{proof}
Since $\Gamma\vdash_0 e\equiv e'$ means $e=e'$, all cases are trivial by inversion on the construction of the term.
\end{proof}

\subsection{The $\kappa$ reduction}\label{sec:kappa}
\emph{Note: In this section, we will omit the indices from the provability relation, but we will focus on characterizing the $\equiv$ relation at a particular level. So read $\Gamma\vdash \alpha\equiv\beta$ as $\Gamma\vdash_{n+1} \alpha\equiv\beta$, and $\Gamma\vdash e:\alpha$ as $\Gamma\vdash_n e:\alpha$. Also (and importantly) we will assume that $\vdash_n$ has unique typing, which will prevent the appearance of certain pathologies.}

The standard formulation of the Church-Rosser theorem, when applied to the $\rightsquigarrow$ reduction relation, is not true; under reasonable definitions of reduction, Lean will not have unique normal forms, because of proof irrelevance. (We already saw how this plays out in \autoref{sec:undecidable}). All other substantive reduction rules act on terms the same way regardless of their types. To analyze this, we will split the definitional equality judgment into two parts: A $\beta\delta\zeta\iota$-reduction relation (henceforth abbreviated $\kappa$ reduction), and a relation that does proof irrelevance. The idea is that $\kappa$ reduction satisfies a modified version of the Church-Rosser theorem, while proof irrelevance picks up the pieces, quantifying exactly how non-unique the normal form is.

The $\eta$ rule can sometimes fight against the $\iota$ reduction in the sense that it is possible for a subsingleton eliminator to reduce in two ways, where the $\eta$ reduced form cannot reduce, for example with the following reductions, using $\rec_{a=}:\forall C.\;C\;a\to \forall b.\;a=b\to C\;b$:
\begin{align*}
&\lambda h:a=a.\;\rec_{a=}\;C\;e\;a\;h\rightsquigarrow_\eta\rec_{a=}\;C\;e\;a\\
&\lambda h:a=a.\;\rec_{a=}\;C\;e\;a\;h\rightsquigarrow_\iota\lambda h:a=a.\;e
\end{align*}

To resolve this, we will require that $\rec$ and $\lift$ always have their required number of parameters. To accomplish this, we define an $\eta$-expansion map as a preprocessing stage on terms before reduction. The transformation is as follows:
\begin{itemize}
\item If $e$ is a list of terms of length $n$ and $\rec_P$ has $m\ge n$ arguments, then $\overline{\rec_P\;e}=\lambda x::\alpha.\;\rec_P\;e\;x$ where $x$ is the remaining $n-m$ arguments, with type $\alpha$ according to the specification of $P$.
\item If $e$ is a list of terms of length $n\le 6$ (note that $\lift$ has 6 arguments), then $\overline{\lift\;e}=\lambda x::\alpha.\;\lift\;e\;x$ where $x$ is the remaining $6-n$ arguments.
\item Otherwise, the transformation is recursive in subterms: $\overline x=x$, $\overline{\lambda x:\alpha.\;e}=\lambda x:\overline{\alpha}.\;\overline{e}$, etc.
\end{itemize}
A term is said to be in $\rec$-normal form if every $\rec_P$ and $\lift$ subterm is followed by a sequence of applications of the appropriate length.

\begin{lemma}[Properties of the $\rec$-normal form]
\begin{itemize}
\item A term $e$ is in $\rec$-normal form iff $\bar e=e$.
\item $\bar e$ is always in $\rec$-normal form.
\item If $\Gamma\vdash e:\alpha$, then $\Gamma\vdash \overline{e}\equiv e$. (In fact, the proof of equivalence uses only $\eta$.)
\item If $e_1,e_2$ are in $\rec$-normal form, then so is $e_1[e_2/x]$.
\end{itemize}
\end{lemma}

The $\kappa$ reduction relation is defined on terms in $\rec$-normal form, with compatibility rules such as these for every syntax operator (including $\rec_P\;e$ and $\lift\;e$):
$$\boxed{\Gamma\vdash e\rightsquigarrow_\kappa e'}\qquad
\frac{\Gamma\vdash e_1 \rightsquigarrow_\kappa e_1'}{\Gamma\vdash e_1\;e_2\rightsquigarrow_\kappa e_1'\;e_2}\qquad
\frac{\Gamma\vdash e_2 \rightsquigarrow_\kappa e_2'}{\Gamma\vdash e_1\;e_2\rightsquigarrow_\kappa e_1\;e_2'}$$
$$\frac{\Gamma\vdash \alpha \rightsquigarrow_\kappa\alpha'}{\Gamma\vdash \lambda x:\alpha.\;e\rightsquigarrow_\kappa \lambda x:\alpha'.\;e}\qquad
\frac{\Gamma,x:\alpha\vdash e \rightsquigarrow_\kappa e'}{\Gamma\vdash \lambda x:\alpha.\;e\rightsquigarrow_\kappa \lambda x:\alpha.\;e'}\qquad\dots$$
The substantive rules are:
$$(\beta)\ \frac{}{\Gamma\vdash (\lambda x:\alpha.\;e)\;e'\rightsquigarrow_\kappa e[e'/x]}\qquad
(\delta)\ \frac{\mathsf{def}\;c:\alpha:=e}{\Gamma\vdash c\rightsquigarrow_\kappa e}\qquad
(\zeta)\ \frac{}{\Gamma\vdash \elet{x:\alpha:=e'}{e}\rightsquigarrow_\kappa e[e'/x]}$$
$$(\iota)\ \frac{P\mbox{ is non-SS inductive with ctor }c}{\Gamma\vdash \rec_P\;C\;e\;p\;(c\;b)\rightsquigarrow_\kappa e_c\;b\;v}\qquad
(\iota_q)\ \frac{}{\Gamma\vdash \lift\;R\;f\;h\;(\mk_R\;a)\rightsquigarrow_\kappa f\;a}$$
$$(K^+)\ \frac{P\mbox{ is SS inductive}\quad\Gamma\vdash \intro\;\inv[p,h]:\alpha}{\Gamma\vdash \rec_P\;C\;e\;p\;h\rightsquigarrow_\kappa e\;\inv[p,h]\;v}$$

See \autoref{sec:inductive} for the variable names and types used in the $\iota$ rules; recall in particular that $v$ in the RHS of the rule is a sequence of lambdas $v_i=\lambda x::\xi_i.\;\rec_P\;C\;e\;\pi_i[b,x]\;(u_i\;x)$ dictated by the definition of the inductive type.

We have an alternate $\iota$ rule for SS inductives, where $\inv[p,h]$ is a sequence of terms such that $\intro\;\inv[p,h]\equiv h$ (by proof irrelevance) and $\inv_i[p,\intro\;b]\equiv b_i$, which we call $K^+$ because it is a souped-up version of the K-like reduction rule in \autoref{sec:iota}. It applies only when $\intro\;\inv[p,h]$ is well-typed (and is the reason why $\rightsquigarrow_\kappa$ needs a context), which can also be written as a collection of $\equiv$ judgments at $\vdash_n$.

By the definition of a subsingleton inductive, every argument to the $\intro$ constructor is either propositional, or appears as one of the parameters $p_i$ to the inductive family. We define $\inv_i[p,h]:=p_j$ when the $i$th constructor argument is non-propositional and appears at position $j$ in the output type, and $\inv_i[p,h]=\inv_i\;h$ for the propositions, where $\inv_i$ is an atomic projection function. These $\inv_i$ projection operators can be defined using the recursor, like we demonstrated for $\acc$ in \autoref{sec:undecidable}. It doesn't really matter if these terms reduce or not (i.e. they could be constants or defined via the recursor), since they are proofs and are thus going to be pushed into the proof irrelevance relation.

The proof irrelevance relation deals with all the ways that normal forms can fail to be unique. Specifically, this relation is responsible for changing universe levels and changing proofs, as well as the $\eta$ rule.
$$\boxed{\Gamma\vdash e\equiv_p e'}$$
$$\frac{\Gamma\vdash e:\alpha}{\Gamma\vdash e\equiv_p e}\qquad
\frac{\Gamma\vdash\alpha\equiv_p\alpha'\quad \Gamma,x:\alpha\vdash e\equiv_p e'}{\Gamma\vdash \lambda x:\alpha.\;e\equiv_p \lambda x:\alpha'.\;e'}\qquad
\frac{\Gamma\vdash e_1\equiv_p e_1'\quad \Gamma\vdash e_2\equiv_p e_2'}{\Gamma\vdash e_1\;e_2\equiv_p e_1'\;e_2'}\qquad\dots$$
$$\frac{\Gamma,x:\alpha\vdash e\equiv_p e'\;x}{\Gamma\vdash \lambda x:\alpha.\;e\equiv_p e'}\qquad
\frac{\Gamma,x:\alpha\vdash e\;x\equiv_p e'}{\Gamma\vdash e\equiv_p \lambda x:\alpha.\;e'}\qquad
\frac{\Gamma\vdash p:\P\quad \Gamma\vdash h:p\quad \Gamma\vdash h':p}{\Gamma \vdash h\equiv_p h'}$$
The ellipsis abbreviates all the compatibility rules. The $\eta$ rule here is split in two because $\equiv_p$ lacks a symmetry rule, but it is also tightly syntax-constrained -- it is essentially only useful for proving $\lambda x::\alpha.\;e\;x\equiv_p e$. In particular it does not apply at all on proving that two variables of function type are equivalent, or proving that two universes or non-function things are equivalent.

\begin{lemma}[Regularity of reductions]
\begin{thmlist}
\item If $\Gamma\vdash e:\alpha$ and $\Gamma\vdash e\rightsquigarrow_\kappa e'$, then $\Gamma\vdash e\equiv e':\alpha$.
\item $\equiv_p$ is an equivalence relation.
\item If $\Gamma\vdash e\equiv_p e'$, then $\Gamma\vdash e\equiv e'$.
\item\label{item:p_subst} If $\Gamma,x:\alpha\vdash e_1\equiv_p e_1'$ and $\Gamma\vdash e_2\equiv_p e_2'$ then $\Gamma\vdash e_1[e_2/x]\equiv_p e_1'[e_2'/x]$.
\end{thmlist}
\end{lemma}
\begin{proof}
All parts are easy inductions.
\end{proof}
Note that the first part implies subject reduction for $\rightsquigarrow_\kappa$.

\subsection{The Church-Rosser theorem}\label{sec:church_rosser}
\begin{theorem}[Church-Rosser property]\label{thm:church_rosser}
If $\Gamma\vdash e:\alpha$, and $e\rightsquigarrow_\kappa^* e_1$ and $e\rightsquigarrow_\kappa^* e_2$, then there exists $e_1'$ and $e_2'$ such that $\Gamma\vdash e_1'\equiv_p e_2'$, and $e_1\rightsquigarrow_\kappa^* e_1'$ and $e_2\rightsquigarrow_\kappa^* e_2'$.
\end{theorem}
The proof follows the Tait--Martin-L\"{o}f method, extended to all the $\kappa$ rules. Define the parallel reduction $\gg_\kappa$ by the following rules:
$$\frac{}{\Gamma\vdash x\gg_\kappa x}\qquad
\frac{\Gamma\vdash \alpha\gg_\kappa\alpha'\quad \Gamma,x:\alpha\vdash e\gg_\kappa e'}{\Gamma\vdash \lambda x:\alpha.\;e\gg_\kappa \lambda x:\alpha'.\;e'}\qquad
\frac{\Gamma\vdash e_1\gg_\kappa e_1'\quad \Gamma\vdash e_2\gg_\kappa e_2'}{\Gamma\vdash e_1\;e_2\gg_\kappa e_1'\;e_2'}\qquad\dots$$
$$\frac{\Gamma,x:\alpha\vdash e_1\gg_\kappa e_1'\quad \Gamma\vdash e_2\gg_\kappa e_2'}{\Gamma\vdash (\lambda x:\alpha.\;e_1)\;e_2\gg_\kappa e_1'[e_2'/x]}$$
$$\frac{\Gamma\vdash e_2[e_1/x]\gg_\kappa e'}{\Gamma\vdash \elet{x:\alpha:=e_1}{e_2}\gg_\kappa e'}\qquad
\frac{\mathsf{def}\;c:\alpha:=e\quad \Gamma\vdash e\gg_\kappa e'}{\Gamma\vdash c\gg_\kappa e'}$$
$$\frac{\Gamma\vdash f\gg_\kappa f'\quad \Gamma\vdash a\gg_\kappa a'}{\Gamma\vdash \lift\;R\;f\;h\;(\mk_R\;a)\gg_\kappa f'\;a'}\qquad
\frac{\begin{matrix}
P\mbox{ is non-SS inductive with ctor }c\\
\Gamma\vdash C,e,b,p\gg_\kappa C',e',b',p'
\end{matrix}}{\Gamma\vdash \rec_P\;C\;e\;p\;(c\;b)\gg_\kappa e'\;b'\;v'}$$
$$\frac{P\mbox{ is SS inductive}\quad\Gamma\vdash \intro\;\inv[p,h]:\alpha\quad\Gamma\vdash C,e,p,h\gg_\kappa C',e',p',h'}{\Gamma\vdash \rec_P\;C\;e\;p\;h\gg_\kappa e_c'\;\inv[p',h']\;v'}$$

The ellipsis on the first line abbreviates compatibility rules for all the term constructors, recursing into all subterms like in the examples for lambda and application. All the substantive rules also follow a similar pattern: for each substantive rule in $\rightsquigarrow_\kappa$, there is a corresponding rule where after applying the $\rightsquigarrow_\kappa$ rule all variables on the RHS are $\gg_\kappa$ evaluated to the primed versions, and these are what end up in the target expression. (Note that in the $\iota$ rule, $v$ is a term that mentions $e$ and $p$; these are replaced by the primed versions in $v'$.)

In addition, we define the following ``complete reduction'' $\Gamma\vdash e\ggg_\kappa e'$ by exactly the same rules as $\gg_\kappa$, except that the compatibility rules only apply if none of the substantive rules are applicable. This makes $\ggg_\kappa$ almost deterministic (producing a unique $e'$ given $e$), except that the $\equiv_p$ hypothesis in the $\iota$ rule allows some freedom of choice of the parameters $b$.

It is easy to prove the following properties by induction:
\begin{lemma}[Properties of $\gg_\kappa$]\label{thm:gg_prop}
\begin{thmlist}
\item If $\Gamma\vdash e:\alpha$, then $\Gamma\vdash e\gg_\kappa e$.
\item\label{item:red_gg} If $\Gamma\vdash e\rightsquigarrow_\kappa e'$ then $\Gamma\vdash e\gg_\kappa e'$.
\item\label{item:gg_red} If $\Gamma\vdash e\gg_\kappa e'$ then $\Gamma\vdash e\rightsquigarrow_\kappa^* e'$.
\item\label{item:gg_subst} If $\Gamma,x:\alpha\vdash e_1\gg_\kappa e_1'$ and $\Gamma\vdash e_2\gg_\kappa e_2'$ (where $\Gamma\vdash e_2:\alpha$) then\\ $\Gamma\vdash e_1[e_2/x]\gg_\kappa e_1'[e_2'/x]$.
\item\label{item:ggg_gg} If $\Gamma\vdash e\ggg_\kappa e'$, then $\Gamma\vdash e\gg_\kappa e'$.
\item\label{item:ggg_ex} If $\Gamma\vdash e:\alpha$, then $\Gamma\vdash e\ggg_\kappa e'$ for some $e'$.
\end{thmlist}
\end{lemma}

\begin{lemma}[Compatibility of $\gg_\kappa$ with $\equiv_p$]\label{thm:gg_compat}
If $\Gamma\vdash e_1\equiv_p e_3\gg_\kappa e_2$, then there exists $e_4$ such that $\Gamma\vdash e_1\gg_\kappa e_4\equiv_p e_2$.
\end{lemma}
\begin{proof}
By induction on $e_1\equiv_p e_3$ and inversion on $e_3\gg_\kappa e_2$. (We will omit the contexts from the relations.)
\begin{itemize}
\item If $e_1\equiv_p e_3=e_1$ by the reflexivity rule, then $e_1\gg_\kappa e_2\equiv_p e_2$.
\item If $e_1\equiv_p e_3$ by the proof irrelevance rule, then $e_3:p:\P$, so $e_2:p:\P$ as well and hence $e_1\gg_\kappa e_1\equiv_p e_2$.
\item If $e_1\equiv_p e_3$ and $e_3\gg_\kappa e_2$ both use the same compatibility rule, then it is immediate from the induction hypothesis.
\item If $e_1:p:\P$ is a proof, then $e_1\gg_\kappa e_1\equiv_p e_2$. (We will thus assume that $e_1$ is not a proof in later cases.)
\item If $(\lambda x:\alpha_1.\;e_1)\;e_1'\equiv_p(\lambda x:\alpha_3.\;e_3)\;e_3'\gg_\kappa e_2[e_2'/x]$ where $e_1\equiv_p e_3\gg_\kappa e_2$, $e_1'\equiv_p e_3'\gg_\kappa e_2'$ and $\alpha_1\equiv\alpha_3$, then $(\lambda x:\alpha_1.\;e_1)\;e_1'\gg_\kappa e_1[e_1'/x]\equiv_p e_2[e_2'/x]$. (Other cases are similar, when the $\equiv_p$ is proven by compatibility rules and the $\gg_\kappa$ is a substantive rule.)
\item If $e_1\;e_1'\equiv_p(\lambda x:\alpha_3.\;e_3)\;e_3'\gg_\kappa e_2[e_2'/x]$ where $e_1\;x\equiv_p e_2\gg_\kappa e_3$ and $e_1'\equiv_p e_2'\gg_\kappa e_3'$, then $e_1\;e_1'=(e_1\;x)[e_1'/x]\equiv_p e_2[e_2'/x]$.
\item If $\lift\;R_1\;\beta_1\;f_1\;h_1\;q_1\equiv_p\lift\;R_3\;\beta_3\;f_3\;h_3\;(\mk_R\;a_3)$ where $q_1\equiv_p \mk_R\;a_3$ by proof irrelevance, then $\beta:\P$ so $e_1:\beta$ is a proof. (Note: we are using that $\vdash_n$ has unique typing here.)
\item If $\rec_P\;C_1\;e_1\;p_1\;h_1\equiv_p\rec_P\;C_3\;e_3\;p_3\;(c\;b_3)\gg_\kappa (e_2)_c\;b_2\;v_2$ where $P$ is non-SS inductive and $h_1\equiv_p c\;b_3$ by proof irrelevance, it is a small eliminator, so $\rec_P\;C_1\;e_1\;p_1\;h_1$ is a proof.
\end{itemize}
\end{proof}
\begin{lemma}[Triangle lemma]\label{thm:tri}
If $\Gamma\vdash e:\alpha$, $e\gg_\kappa e'$, and $e\ggg_\kappa e^\bullet$, then there exists $e^\circ$ such that $\Gamma\vdash e'\gg_\kappa e^\circ\equiv_p e^\bullet$.
\end{lemma}
\begin{proof}
By induction on $e\ggg_\kappa e^\bullet$ and inversion on $e\gg_\kappa e'$.
\begin{itemize}
\item If $x\lll_\kappa x\gg_\kappa x$, then $x\gg_\kappa x\equiv_p x$.
\item If $e\ggg_\kappa e^\bullet$ by the beta rule:
\begin{itemize}
\item If $e_1^\bullet[e_2^\bullet/x]\lll_\kappa(\lambda x:\alpha.\;e_1)\;e_2\gg_\kappa e_1'[e_2'/x]$ by the beta rule, then $e_1'\gg_\kappa e_1^\circ$ and $e_2'\gg_\kappa e_2^\circ$ by the inductive hypothesis, so $e_1'[e_2'/x]\gg_\kappa e_1^\circ[e_2^\circ/x]\equiv_p e_1^\bullet[e_2^\bullet/x]$ by the substitution property.
\item If $e_1^\bullet[e_2^\bullet/x]\lll_\kappa(\lambda x:\alpha.\;e_1)\;e_2\gg_\kappa (\lambda x:\alpha.\;e_1')\;e_2'$ by the application rule and lambda rule, then $(\lambda x:\alpha.\;e_1')\;e_2'\gg_\kappa e_1^\circ[e_2^\circ/x]\equiv_p e_1^\bullet[e_2^\bullet/x]$ by the beta rule for $\gg_\kappa$ and the IH.
\end{itemize}
\item If $e^\bullet\lll_\kappa c\gg_\kappa e'$ by the delta rule, then $e'\gg_\kappa e^\circ\equiv_p e^\bullet$.
\item If $e_1^\bullet[e_2^\bullet/x]\lll_\kappa\elet{x:\alpha:=e_1}{e_2}\gg_\kappa e_2'[e_1'/x]$ by the zeta rule, then $e_1'[e_2'/x]\gg_\kappa e_1^\circ[e_2^\circ/x]\equiv_p e_1^\bullet[e_2^\bullet/x]$.
\item If $e\ggg_\kappa e^\bullet$ by the non-SS inductive iota rule:
\begin{itemize}
\item If $e_c^\bullet\;b^\bullet\;v^\bullet\lll_\kappa \rec_P\;C\;e\;p\;(c\;b)\gg_\kappa e_c'\;b'\;v'$ by the iota rule, then $e_c'\;b'\;v'\gg_\kappa e_c^\circ\;b^\circ\;v^\circ\equiv_p e_c^\bullet\;b^\bullet\;v^\bullet$.
\item If $e_c^\bullet\;b^\bullet\;v^\bullet\lll_\kappa \rec_P\;C\;e\;p\;(c\;b)\gg_\kappa \rec_P\;C'\;e'\;p'\;(c\;b')$ by the $\rec_P$ compatibility rule, then $\rec_P\;C'\;e'\;p'\;(c\;b')\gg_\kappa e_c^\circ\;b^\circ\;v^\circ\equiv_p e_c^\bullet\;b^\bullet\;v^\bullet$ by the iota rule.
\end{itemize}
\item If $e\ggg_\kappa e^\bullet$ by the quotient iota rule:
\begin{itemize}
\item If $f^\bullet\;a^\bullet\lll_\kappa \lift\;R\;f\;h\;(\mk_R\;a)\gg_\kappa f'\;a'$ by the iota rule, then $f'\;a'\gg_\kappa f^\circ\;a^\circ\equiv_p f^\bullet\;a^\bullet$.
\item If $f^\bullet\;a^\bullet\lll_\kappa \lift\;R\;f\;h\;(\mk_R\;a)\gg_\kappa\lift\;R'\;f'\;h'\;(\mk_{R'}\;a')$ by the $\lift$ compatibility rule, then we have $q'\gg_\kappa q^\circ\equiv_p q^\bullet$ for $q\in\{C,e,p,h\}$, and $\lift\;R'\;f'\;h'\;(\mk_{R'}\;a')\gg_\kappa f^\circ\;a^\circ\equiv_p f^\bullet\;a^\bullet$.
\end{itemize}
\item If $e\ggg_\kappa e^\bullet$ by the $K^+$ rule:
\begin{itemize}
\item If $e_c^\bullet\;\inv[p^\bullet,h^\bullet]\;v^\bullet\lll_\kappa \rec_P\;C\;e\;p\;h\gg_\kappa e_c'\;\inv[p',h']\;v'$ by the iota rule, then\\
$e_c'\;\inv[p',h']\;v'\gg_\kappa e_c^\circ\;\inv[p^\circ,h^\circ]\;v^\circ\equiv_p e_c^\bullet\;\inv[p^\bullet,h^\bullet]\;v^\bullet$.
\item If $e_c^\bullet\;\inv[p^\bullet,h^\bullet]\;v^\bullet\lll_\kappa \rec_P\;C\;e\;p\;h\gg_\kappa \rec_P\;C'\;e'\;p'\;h'$ by the $\rec_P$ compatibility rule, then $\rec_P\;C'\;e'\;p'\;h'\gg_\kappa e_c^\circ\;\inv[p^\circ,h^\circ]\;v^\circ\equiv_p e_c^\bullet\;\inv[p^\bullet,h^\bullet]\;v^\bullet$ by the iota rule.
\end{itemize}
\item If $e\ggg_\kappa e^\bullet$ by a compatibility rule:
\begin{itemize}
\item If $e_1^\bullet\;e_2^\bullet\lll_\kappa e_1\;e_2\gg_\kappa e_1'\;e_2'$ by the application rule, then $e_1'\;e_2'\gg_\kappa e_1^\circ\;e_2^\circ\equiv_p e_1^\bullet\;e_2^\bullet$.
\item If $\forall x:\alpha^\bullet.\;e^\bullet\lll_\kappa \forall x:\alpha.\;e\gg_\kappa \forall x:\alpha'.\;e'$ by the forall rule, then $\forall x:\alpha'.\;e'\gg_\kappa \forall x:\alpha^\circ.\;e^\circ\equiv_p \forall x:\alpha^\bullet.\;e^\bullet$.
\item Other compatibility rules follow the same pattern.
\end{itemize}
\end{itemize}\vspace{-6mm}
\end{proof}
The main proof of Church-Rosser is a corollary of \autoref{thm:tri}, and does not differ substantially from the usual proof putting diamonds together, because the additional complication of having $\equiv_p$ at the bottom of the diamond commutes with all the other reductions.
\begin{proof}[Proof of \autoref{thm:church_rosser}]
We prove in succession the following theorems:
\begin{enumerate}
\item If $\Gamma\vdash e:\alpha$, and $e_1\ll_\kappa e\gg_\kappa e_2$, then $\exists e_1'\;e_2'.\; e_1\gg_\kappa e_1'\equiv_p e_2'\ll_\kappa e_2$.
\item If $\Gamma\vdash e:\alpha$, and $e_1\ll_\kappa^* e\gg_\kappa e_2$, then $\exists e_1'\;e_2'.\; e_1\gg_\kappa e_1'\equiv_p e_2'\ll_\kappa^* e_2$.
\item If $\Gamma\vdash e:\alpha$, and $e_1\ll_\kappa^* e\gg_\kappa^* e_2$, then $\exists e_1'\;e_2'.\; e_1\gg_\kappa^* e_1'\equiv_p e_2'\ll_\kappa^* e_2$.
\item If $\Gamma\vdash e:\alpha$, and $e_1\leftsquigarrow_\kappa^*e\rightsquigarrow_\kappa^* e_2$, then $\exists e_1'\;e_2'.\; e_1\rightsquigarrow_\kappa^* e_1'\equiv_p e_2'\leftsquigarrow_\kappa^* e_2$.
\end{enumerate}
(4) is the theorem we want.
\begin{enumerate}
\item If $e\gg_\kappa e_1,e_2$, then by \autoref{thm:tri} there exists $e_1',e_2'$ such that $e_i\gg_\kappa e_i'$ and $e_1'\equiv_p e^\bullet\equiv_p e_2'$. But then $e_1'\equiv_p e_2'$ are as desired.
\item By induction on $e\gg_\kappa^* e_1$. If $e\gg_\kappa^* e_1\gg_\kappa e_3$ and we have inductively that $e_1\gg_\kappa e_1'\equiv_p e_2'\ll_\kappa^* e_2$, then by applying (1) to $e_3\ll_\kappa e_1\gg_\kappa e_1'$ we obtain $e_3\gg_\kappa e_3'\equiv_p e_1''\ll_\kappa e_1'$, and by \autoref{thm:gg_compat} applied to $e_1''\ll_\kappa e_1'\equiv_p e_2'$ we obtain $e_1''\equiv_p e_2''\ll_\kappa e_2'$ so that  $e_3\gg_\kappa e_3'\equiv_p e_1''\equiv_p e_2''\ll_\kappa e_2'\ll_\kappa^* e_2$.
\item By induction on $e\gg_\kappa^* e_2$. The proof is the same as (2), replacing \autoref{thm:gg_compat} for the analogous statement for $\gg_\kappa^*$, i.e. if $e_1\gg_\kappa^*e_2$ and $e_1\equiv_p e_1'$ then there exists $e_2'$ such that $e_1'\gg_\kappa^*e_2'\equiv_p e_1'$. This follows by induction on \autoref{thm:gg_compat}.
\item  The equivalence of (3) and (4) comes from properties \ref{item:red_gg} and \ref{item:gg_red}.
\end{enumerate}
\end{proof}

Now say that $\Gamma\vdash e_1\equiv_\kappa e_2$ if $\Gamma\vdash e_1,e_2:\alpha$ for some $\alpha$, and there exists $e_1',e_2'$ such that $\Gamma\vdash e_1\rightsquigarrow_\kappa^* e_1'\equiv_p e_2'\leftsquigarrow_\kappa^* e_2$. This relation is obviously reflexive and symmetric and implies $\Gamma\vdash e_1\equiv e_2$, and the Church-Rosser property implies it is also transitive.

\begin{theorem}[Completeness of the $\kappa$ reduction]\label{thm:ckappa}
$\Gamma\vdash e\equiv e'$ if and only if $\Gamma\vdash e\equiv_\kappa e'$.
\end{theorem}
\begin{proof}
The reverse direction follows from regularity lemmas observed above. The forward direction is by induction on $\equiv$.
\begin{itemize}
\item The equivalence relation rules are immediate since $\equiv_\kappa$ is an equivalence relation (by the Church-Rosser property).
\item For the compatibility rules, since both $\equiv_p$ and $\rightsquigarrow_\kappa$ have compatibility rules, this property passes to $\equiv_\kappa$. Thus, for example in the lambda case, we have $\Gamma\vdash\lambda x:\alpha.\;e\equiv_\kappa\lambda x:\alpha.\;e'$ since $\Gamma,x:\alpha\vdash e\equiv_\kappa e'$ from the IH, and similarly $\Gamma\vdash\lambda x:\alpha.\;e'\equiv_\kappa\lambda x:\alpha'.\;e'$, so by transitivity $\Gamma\vdash\lambda x:\alpha.\;e\equiv_\kappa\lambda x:\alpha'.\;e'$.
\item The universe changing rules (for constants and $\U_\ell$) are in $\equiv_p$.
\item The $\beta$ and $\eta$ rules are in $\rightsquigarrow_\kappa$, and the proof irrelevance rule is in $\equiv_p$. All the other equivalence rules are also introduced in $\rightsquigarrow_\kappa$.
\item For subsingleton eliminators, we must show $\rec_P(C,e,p,\intro\;b)\equiv_\kappa e\;b\;v$. From the $K^+$ rule we have $\rec_P(C,e,p,\intro\;b)\equiv_\kappa e\;\inv[p,c\;b]\;v$ so it suffices to show $\inv_i[p,\intro\;b]\equiv_\kappa b_i$ for each $i$. If $b_i$ is propositional then this is by proof irrelevance, otherwise $\inv_i[p,\intro\;b]=p_j$, and the well-typedness of $\rec_P(C,e,p,\intro\;b)$ implies that $\Gamma\vdash_n b_i\equiv p_j$. Thus by completeness of the $\kappa$ reduction at $\vdash_n$, $\Gamma\vdash_n b_i\equiv_\kappa p_j$ and hence $\Gamma\vdash_{n+1} b_i\equiv_\kappa p_j$.
\end{itemize}
\end{proof}

Now we can finally finish the inductive step of the proof of \autoref{thm:unique}:

\begin{theorem}[Definitional inversion]\label{thm:1dinv}
$\vdash_{n+1}$ has definitional inversion.
\end{theorem}
\begin{proof}
In each case we apply \autoref{thm:ckappa} on the assumptions.
\begin{enumerate}
\item \emph{If $\Gamma\vdash_{n+1} \U_\ell\equiv\U_{\ell'}$, then $\ell\equiv\ell'$.}

Again, there are no $\rightsquigarrow_\kappa$ reductions from $\U_\ell$, so $\Gamma\vdash_{n+1} \U_\ell\equiv_p\U_{\ell'}$, and if the compatibility rule is used then $\ell\equiv\ell'$. If proof irrelevance is used, then $\Gamma\vdash_n\U_\ell,\U_{\ell'}:p$ for some $\Gamma\vdash_n p:\P$. Since $\Gamma\vdash_n\U_\ell:\U_{S\ell}:\U_{SS\ell}$ as well, by unique typing at $n$, $\Gamma\vdash_n \P\equiv \U_{SS\ell}$, so by definitional inversion $0\equiv SS\ell$, a contradiction.

\item \emph{If $\Gamma\vdash_{n+1} \forall x:\alpha.\;\beta\equiv\forall x:\alpha'.\;\beta'$, then $\Gamma\vdash_{n+1} \alpha\equiv\alpha'$ and $\Gamma,x:\alpha\vdash_{n+1} \beta\equiv \beta'$.}

In this case, there are no $\rightsquigarrow_\kappa$ reductions except the compatibility rules, so $\forall x:\alpha.\;\beta\rightsquigarrow_\kappa^*\forall x:\alpha_1.\;\beta_1$ for some $\alpha\rightsquigarrow_\kappa^*\alpha_1$ and $\beta\rightsquigarrow_\kappa^*\beta_1$, and similarly $\alpha'\rightsquigarrow_\kappa^*\alpha'_1$ and $\beta'\rightsquigarrow_\kappa^*\beta'_1$, and if these are $\equiv_p$ equivalent using the compatibility rule then we are done.

If $\Gamma\vdash_{n+1}\forall x:\alpha_1.\;\beta_1\equiv_p\forall x:\alpha'_1.\;\beta'_1$ by proof irrelevance, then $\Gamma\vdash_n\forall x:\alpha_1.\;\beta_1,\forall x:\alpha'_1.\;\beta'_1:p:\P$. But $\Gamma\vdash_n\forall x:\alpha_1.\;\beta_1:\U_{\imax(\ell_1,\ell_2)}$ for some $\ell_1,\ell_2$ since $\alpha_1$ and $\beta_1$ are well-typed, so by unique typing at $n$, $p\equiv\U_{\imax(\ell_1,\ell_2)}$ and $0\equiv S\imax(\ell_1,\ell_2)$, a contradiction.

\item \emph{$\Gamma\vdash_n \U_\ell\not\equiv\forall x:\alpha.\;\beta$.}

Suppose not. Similarly to previous parts, as there are no reductions from $\U_\ell$ and no reductions except the compatibility rule for $\forall$, we obtain $\Gamma\vdash_n\U_\ell\equiv_p\forall x:\alpha'.\;\beta'$, and now there is no applicable rule except proof irrelevance, but this implies $\U_\ell:p:\P$ and hence $0\equiv SS\ell$, a contradiction.
\end{enumerate}
\end{proof}

We've already described the structure of this theorem in earlier parts, but now we are finally ready to put all the parts together:

\begin{proof}[Proof of \autoref{thm:unique}]
We prove by induction on $n$ that $\vdash_n$ has definitional inversion (and hence unique typing, by \autoref{thm:utype}), and also that it satisfies the conclusion of \autoref{thm:ckappa}.
\begin{itemize}
\item For $n=0$, $\vdash_0$ has definitional inversion by \autoref{thm:0dinv}, and \autoref{thm:ckappa} is trivial (where both $\Gamma\vdash e\equiv_\kappa e'$ and $\Gamma\vdash e\equiv e'$ mean $e=e'$).
\item For $n+1$, suppose $\vdash_n$ has definitional inversion and satisfies \autoref{thm:ckappa}. Then all the results of \autoref{sec:church_rosser} follow, including \autoref{thm:ckappa}. Then definitional inversion at $n+1$ is \autoref{thm:1dinv}.
\end{itemize}
\end{proof}

\section{Reduction of inductive types to $\W$-types}

Given the complicated structure involved in simply stating the axioms of inductive types, one may wonder if there is an easier way. In fact there is; we can replace the whole structure of inductive types with a few simple inductive type constructors.

\subsection{The menagerie}
The most well known general form of our kind of inductive type is the $\W$-type, defined when $\Gamma\vdash A:\U_\ell$ and $\Gamma,x:A\vdash B:\U_\ell$:
$$\W x:A.\;B:=\mu w:\U_\ell.\;(\mathsf{sup}:\forall x:A.\;(B\to w)\to w)$$
This carries most of the ``power'' of inductive types, but we still need some glue to be able to reduce everything else to this. First, note that most of the telescopes $x::\alpha$ in an inductive type can be replaced by $\Sigma(x::\alpha)$, where $\Sigma ():=\bf 1$ and $\Sigma (x:\alpha,y::\beta):=\Sigma x:\alpha,\Sigma(y::\beta)$. This just packs up all the types in the telescope into one dependent tuple. Similarly, we want the types $\bf 0$ and $\alpha+\beta$ to pack up all the constructors into one.

To localize the universe management we will have a ``universe lift'' function $\ulift_u^v:\U_u\to\U_v$, defined when $u\le v$, as well as the $\nonempty$ operation (also known as the propositional truncation $\|\alpha\|$) to construct small eliminators. All the other type operators above will have the smallest possible universe level.

Finally, to handle inductive families and subsingleton eliminators, we will need the equality and $\acc$ types discussed previously. Here are the rules for these types:
\begin{align*}
e::=\dots&\mid\bot\mid\Sigma x:e.\;e\mid e+e\mid \ulift_{\ell}^{\ell}\;e\mid \|e\|\mid \mathsf{W} x:e.\;e\mid e=e\mid \acc_e\\
&\mid\rec_\bot\mid(e,e)\mid\pi_1\;e\mid\pi_2\;e\mid \inl e\mid \inr e\mid \rec_+\;e\;e\mid {\uparrow}e\mid {\downarrow}e\\
&\mid |e|\mid\rec_{||}\;e\mid\sup e\;e\mid\rec_\W\;e\mid \refl\;e\mid \rec_=\;e\;e\mid \intro_\acc\;e\;e\mid\rec_\acc\;e
\end{align*}
$$\frac{}{\vdash \bot:\P}\qquad
\frac{1\le\ell\quad \Gamma\vdash C:\U_\ell}{\Gamma\vdash \rec_\bot:\bot\to C}$$
$$\frac{\Gamma\vdash \alpha:\U_\ell\quad \Gamma,x:\alpha\vdash \beta:\U_{\ell'}}
{\Gamma\vdash \Sigma x:\alpha.\;\beta:\U_{\max(\ell,\ell',1)}}\qquad
\frac{\Gamma\vdash \alpha:\U_\ell\quad \Gamma\vdash \beta:\U_{\ell'}}
{\Gamma\vdash \alpha+\beta:\U_{\max(\ell,\ell',1)}}$$
$$\frac{\Gamma\vdash e_1:\alpha\quad \Gamma\vdash e_2:\beta\;e_1}
{\Gamma\vdash(e_1,e_2):\Sigma x:\alpha.\;\beta}\qquad
\frac{\Gamma\vdash p:\Sigma x:\alpha.\;\beta}{\Gamma\vdash \pi_1\;p:\alpha}\qquad
\frac{\Gamma\vdash p:\Sigma x:\alpha.\;\beta}
{\Gamma\vdash \pi_2\;p:\beta[\pi_1\;p/x]}$$
$$\frac{\Gamma\vdash \beta\type\quad \Gamma\vdash e:\alpha}{\Gamma\vdash \inl e:\alpha+\beta}\qquad
\frac{\Gamma\vdash \alpha\type\quad \Gamma\vdash e:\beta}{\Gamma\vdash \inr e:\alpha+\beta}$$
$$\frac{1\le\ell\quad \Gamma\vdash C:\alpha+\beta\to\U_\ell\quad \Gamma\vdash a:\forall x:\alpha.\;C\;(\inl x)\quad \Gamma\vdash b:\forall x:\beta.\;C\;(\inr x)}{\Gamma\vdash \rec_+\;a\;b:\forall p:\alpha+\beta.\;C\;p}$$
$$\frac{\Gamma\vdash \alpha:\U_\ell\quad \max(1,\ell)\le\ell'}
{\Gamma\vdash\ulift_{\ell}^{\ell'}\;\alpha:\U_{\ell'}}\qquad
\frac{\Gamma\vdash\ulift_{\ell}^{\ell'}\;\alpha:\U_{\ell'}\quad \Gamma\vdash e:\alpha}{\Gamma\vdash {\uparrow}e:\ulift_{\ell}^{\ell'}\;\alpha}\qquad
\frac{\Gamma\vdash e:\ulift_{\ell}^{\ell'}\;\alpha}{\Gamma\vdash {\downarrow}e:\alpha}$$
$$\frac{\Gamma\vdash \alpha\type}
{\Gamma\vdash\|\alpha\|:\P}\qquad
\frac{\Gamma\vdash e:\alpha}{\Gamma\vdash |e|:\|\alpha\|}\qquad
\frac{\Gamma\vdash C:\P\quad \Gamma\vdash f:\alpha\to C}{\Gamma\vdash \rec_{||}\;f:\|\alpha\|\to C}$$
$$\frac{\Gamma\vdash \alpha:\U_\ell\quad \Gamma,x:\alpha\vdash\beta:\U_{\ell'}}
{\Gamma\vdash\W x:\alpha.\;\beta:\U_{\max(\ell,\ell',1)}}\qquad
\frac{\Gamma\vdash a:\alpha\quad\Gamma\vdash f:\beta[a/x]\to \W x:\alpha.\;\beta}{\Gamma\vdash\sup a\;f:\W x:\alpha.\;\beta}$$
$$\frac{\begin{matrix}
1\le\ell\quad \Gamma\vdash C:(\W x:\alpha.\;\beta)\to\U_\ell\\
\Gamma\vdash e:\forall (a:\alpha)\;(f:\beta[a/x]\to\W x:\alpha.\;\beta).\;(\forall b:\beta[a/x].\;C\;(f\;b))\to C\;(\sup a\;f)
\end{matrix}}{\Gamma\vdash \rec_\W\;e:\forall w:(\W x:\alpha.\;\beta).\;C\;w}$$
$$\frac{\Gamma\vdash a:\alpha\quad\Gamma\vdash b:\alpha}
{\Gamma\vdash a=b:\P}\qquad
\frac{\Gamma\vdash a:\alpha}{\Gamma\vdash \refl\;a:a=a}$$
$$\frac{\Gamma\vdash a:\alpha\quad 1\le\ell\quad \Gamma\vdash C:\alpha\to\U_\ell\quad\Gamma\vdash e:C\;a}{\Gamma\vdash \rec_=\;e:\forall b:\alpha.\;a=b\to C\;b}$$
$$\frac{\Gamma\vdash r:\alpha\to\alpha\to\P}{\Gamma\vdash \acc_r:\alpha\to\P}\qquad
\frac{\Gamma\vdash x:\alpha\quad\Gamma\vdash f:\forall y:\alpha.\;r\;y\;x\to\acc_r\;y}{\Gamma\vdash \intro_\acc\;x\;f:\acc_r\;x}$$
$$\frac{\begin{matrix}
1\le\ell\quad \Gamma\vdash C:\alpha\to\U_\ell\\
\Gamma\vdash e:\forall x:\alpha.\; (\forall y:\alpha.\;r\;y\;x\to\acc_r\;y)\to (\forall y:\alpha.\;r\;y\;x\to C\;y)\to C\;x
\end{matrix}}{\Gamma\vdash \rec_\acc\;e:\forall x:\alpha.\;\acc_r\;x\to C\;x}$$
All of these could have been defined as inductive types in the sense of \autoref{sec:inductive}:
\begin{align*}
\bot&:=\mu t:\P.\;0\\
\Sigma x:\alpha.\;\beta&:=\mu t:\U_{\max(\ell,\ell',1)}.\;(\mathsf{pair}:\forall x:\alpha.\;\beta\to t)\\
\alpha+\beta&:=\mu t:\U_{\max(\ell,\ell',1)}.\;(\mathsf{inl}:\alpha\to t)+(\mathsf{inr}:\beta\to t)\\
\ulift_{\ell}^{\ell'}\;\alpha&:=\mu t:\U_{\ell'}.\;(\mathsf{up}:\alpha\to t)\\
\|\alpha\|&:=\mu t:\P.\;(\intro:\alpha\to t)\\
\W x:\alpha.\;\beta&:=\mu t:\U_{\max(\ell,\ell',1)}.\;(\mathsf{sup}:\forall x:\alpha.\;(\beta\to t)\to t)\\
a=b&:=(\mu t:\alpha\to\P.\;(\refl:t\;a))\;b\\
\acc_r&:=\mu t:\alpha\to\P.\;(\intro:\forall x:\alpha.\;(\forall y:\alpha.\;r\;y\;x\to t\;y)\to t\;x)
\end{align*}
However, we are interested in taking them as primitive in this section and deriving general inductive types. All of the new operators have compatibility rules for $\equiv$ and $\Leftrightarrow$; we will not belabor this as they all look roughly the same: when all the parts are equivalent, so is the whole. For example:
$$\frac{\Gamma\vdash\alpha\equiv\alpha'\quad\Gamma,x:\alpha\vdash\beta\equiv\beta'}{\Gamma\vdash\Sigma x:\alpha.\;\beta\equiv\Sigma x:\alpha'.\;\beta'}$$

Since we will need to handle $\P$ specially in the proof of soundness, we have simplified all the large eliminating recursors to require $1\le\ell$. The general recursor can be constructed from this by using $C':=\lambda x:P.\;\ulift_{\ell}^{\max(1,\ell)}\;(C\;x)$ (for each such inductive type $P$).

In a few of the constructors, additional parameters are elided, such as $C$ in $\rec_\bot$; one should imagine that each constructor is sufficiently annotated to ensure unique typing. Following their interpretation as inductive types, they also come with the following $\iota$ rules:
\begin{align*}
\pi_1(a,b)&\equiv a\\
\pi_2(a,b)&\equiv b\\
\rec_+\;a\;b\;(\inl x)&\equiv a\;x\\
\rec_+\;a\;b\;(\inr x)&\equiv b\;x\\
{\downarrow\uparrow} x&\equiv x\\
\rec_\W\;e\;(\sup a\;f)&\equiv e\;a\;f\;(\lambda b:\beta[a/x].\;\rec_\W\;e\;(f\;b))\\
\rec_=\;e\;a\;h&\equiv e\\
\rec_\acc\;e\;x\;(\intro_\acc\;x\;f)&\equiv e\;x\;f\;(\lambda (y:\alpha)\;(h:r\;y\;x).\;\rec_\acc\;e\;y\;(f\;y\;h))
\end{align*}
which are valid in any context that typechecks everything on the LHS.

Here are a few additional type operators that can be defined from the ones given:
$${\bf 0}_\ell:=\ulift^\ell\;\bot\qquad\top:=\bot\to\bot\qquad{\bf 1}_\ell:=\ulift^\ell\;\top\qquad \alpha\times \beta:=\Sigma\_:\alpha.\;\beta$$
$$p\land q:=\|p\times q\|\qquad p\lor q:=\|p+q\|$$
$$\{x:\alpha\mid p\}:=\Sigma x:\alpha.\;p\qquad
\exists x:\alpha.\; p:=\|\{x:\alpha\mid p\}\|$$

The following additional ``$\eta$ rules'' are needed for the reduction, which are provable but not definitional equalities in Lean. Since we are going for soundness only, we will help ourselves to this modest strengthening of the system; moreover this is only for convenience -- without such $\eta$ rules we would only be able to go as far as indexed $\W$-types, which are more complex. (These rules are also required for this axiomatization since we've omitted the recursors in favor of projections for $\Sigma$ and $\ulift$.)
$${\uparrow\downarrow} x\equiv x\qquad (\pi_1\;x,\pi_2\;x)\equiv x$$
The results of \autoref{sec:unique} apply straightforwardly to this setting, with these two rules added as $\rightsquigarrow_\kappa$ reduction rules along with all the $\iota$ rules mentioned above.
%
\subsection{Translating type families}
Let us first suppose that the inductive family lives in a universe $1\le\ell$. In this case we don't have to worry about $\P$ and small elimination. The idea is to eliminate families by first erasing the indices to get a ``skeleton'' type $S$ that mixes all the different members of the family together, and then separately define a predicate $\mathsf{good}:S\to\forall x::\alpha.\P$ that carves out the members that actually belong to index $x$. The final result will be the type $\lambda x::\alpha.\;\{s:S\mid\mathsf{good}\;s\;x\}$. For example, the type
$$X=\mu t:\N\to\U_1.\;(\mathsf{one}:t\;1)+(\mathsf{double}:\forall n:\N.\;t\;n\to t\;(2n))$$
has the indices erased to get
$$S=\mu t:\U_1.\;(\mathsf{one}:t)+(\mathsf{double}:\forall n:\N.\;t\to t),$$
and then the predicate is defined by recursion on $S$:
\begin{align*}
\mathsf{good}\;\mathsf{one}\;m&:=m=1\\
\mathsf{good}\;(\mathsf{double}\;n\;x)\;m&:=m=2n\land \mathsf{good}\;x\;n
\end{align*}
Now $S$ will also be reduced to the $\W$-type:
$$S'=\W x:{\bf 1}+\N.\;\rec_+\;(\lambda\_.\;{\bf 0})\;(\lambda n.\;{\bf 1})$$
because there are two branches, one with no non-recursive arguments and one with a non-recursive argument of type $\N$ (hence ${\bf 1}+\N$), and first branch has no recursive arguments and the second has one.

So the general translation will take the form
$$P\;x\simeq\{s:\W p:A.\;B\;p\mid \rec_\W\;(\lambda (p:A)\; \_.\;G\;p)\;s\;x\},$$\vspace{-7mm}
\begin{align*}
\mbox{where}\qquad\Gamma&\vdash A:\U_\ell\\
\Gamma&\vdash B:A\to\U_\ell\\
\Gamma&\vdash G:\forall p:A.\;(B\;p\to\forall x::\alpha.\;\P)\to\forall x::\alpha.\;\P.
\end{align*}

We will construct these three terms recursively based on the derivation of the $\spec$ judgment.
$$\boxed{\Gamma;t:F\vdash K\spec\Rightarrow A;B;G}$$
$$\frac{1\le\ell\quad \Gamma\vdash x::\alpha}{\Gamma;t:\forall x::\alpha.\;\U_\ell\vdash 0\spec\Rightarrow {\bf 0};\rec_0;\rec_0}$$
$$\frac{\Gamma;t:F\vdash \beta\ctor\Rightarrow A_1;p.B_1;pgx.G_1\quad \Gamma;t:F\vdash K\spec\Rightarrow A;B;G}{\Gamma;t:F\vdash (c:\beta)+K\spec\Rightarrow A_1+A;\rec_+\;(\lambda p.B_1)\;B;\rec_+\;(\lambda pg\;(x::\alpha).\;G_1)\;G}$$
$$\boxed{\Gamma;t:F\vdash \beta\ctor\Rightarrow A;p.B;pgx.G}$$
$$\frac{\Gamma\vdash e::\alpha}{\Gamma;t:\forall x::\alpha.\;\U_\ell\vdash t\;e\ctor\Rightarrow {\bf 1}_\ell;\ p.\;{\bf 0}_\ell;\ pgx.\;x=e}$$
$$\frac{\Gamma\vdash \beta:\U_{\ell'}\quad \ell'\le\ell\quad \Gamma,y:\beta;t:\forall x::\alpha.\;\U_\ell\vdash\tau\ctor\Rightarrow A;p.B;pgx.G}{
\begin{array}{l}
\Gamma;t:\forall x::\alpha.\;\U_\ell\vdash\forall y:\beta.\;\tau\ctor\Rightarrow\Sigma y':\beta.\;A[y'/y];\\
\qquad p'.B[\pi_1\;p'/y][\pi_2\;p'/p];\ p'gx.G[\pi_1\;p'/y][\pi_2\;p'/p]
\end{array}}$$
$$\frac{\begin{matrix}
\Gamma\vdash \gamma::\U_{\ell'}\quad \Gamma,z::\gamma\vdash e::\alpha\quad \ell'_i\le\ell\\
\Gamma;t:\forall x::\alpha.\;\U_\ell\vdash\tau\ctor\Rightarrow A;p.B;pg'x.G
\end{matrix}}{
\begin{array}{l}
\Gamma;t:\forall x::\alpha.\;\U_\ell\vdash(\forall z::\gamma.\;t\;e)\to\tau\ctor\Rightarrow A;\ p.\;\Sigma(z::\gamma)+B;\\
\qquad pgx.\;G[\lambda b.\;g\;(\inr b)/g']\land \forall z::\gamma.\;g\;(\inl (z))\;e
\end{array}}$$
In the final rule, the notation $(z)$ where $z::\gamma$ means the tuple of elements of $z$ of type $\Sigma(z::\gamma)$: explicitly, $(z_1,\dots,z_n)=(z_1,(z_2,\dots,(x_n,()))):\Sigma(z::\gamma)$.
Note that in the base case of $\mathsf{ctor}$, we have $x=e$ where $x$ and $e$ are telescopes; this can be defined as $(x)=(e)$, or using heterogeneous equality $x_1=e_1\land x_2==e_2\land \dots\land x_n==e_n$, or using the equality recursor $\exists (h_1:x_1=e_1)\;(h_2:\rec_=\;x_2\;x_1\;h_1=e_2)\dots$. We will use $(x)=(e)$ since it is the least notationally burdensome of these options.

The final result is given by the following translation:
$$\frac{\Gamma;t:F\vdash K\spec\Rightarrow A;B;G}
{\Gamma\vdash\scott{\mu t:F.\;K}=\lambda x::\alpha.\;\{s:\W p:A.\;B\;p\mid \rec_\W\;(\lambda (p:A)\; \_.\;G\;p)\;s\;x\}}$$
In the case of a small eliminator, we just artificially lift the target universe above 1, translate it, and then propositionally truncate the resulting type and lift if back to the original universe $\ell$:
$$\frac{\Gamma;t:F\vdash K\spec\quad \neg(\Gamma;t:\forall x::\alpha.\;\U_\ell\vdash K\LE)}
{\Gamma\vdash\scott{\mu t:\forall x::\alpha.\;\U_\ell.\;K}=\lambda x::\alpha.\;\ulift^\ell\|\scott{\mu t:\forall x::\alpha.\;\U_{\ell'}.\;K}\;x\|},$$
where $\ell'$ is the maximum of $1$ and all the constructor arguments. The idea here is that since we have a small eliminator, it's impossible to tell that members of the inductive type are distinct, so we lose nothing in the propositional truncation.

\subsection{Translating subsingleton eliminators}
The hard case is when we have a subsingleton eliminator. In this case we must abandon $\W$-types entirely, since we have to produce a subsingleton family from the start -- propositional truncation will destroy the large elimination property, so we have to use $\acc$ instead. The zero case is easy:
$$\frac{\Gamma\vdash x::\alpha}
{\Gamma\vdash\scott{\mu t:\forall x::\alpha.\;\U_\ell.\;0}=\lambda x::\alpha.\;{\bf 0}_\ell}$$

For our purposes it will be easier to work with the following variant on $\acc$:
\begin{align*}
&\alpha:\U_\ell,\vph:\alpha\to\P,r:\alpha\to\alpha\to\P\vdash\acc^\vph_r=\mu t:\alpha\to\P.\\
&\qquad\;(\intro:\forall x:\alpha.\;\vph\;x\to(\forall y:\alpha.\;r\;y\;x\to t\;y)\to t\;x)
\end{align*}

This is just the same as $\acc_r$ but for the additional parameter $\vph$ that restricts the satisfying instances. This can be built from plain $\acc$ in our existing axiomatization as follows:

\begin{align*}
\acc^\vph_r\;x&:=\exists h:\vph\;x.\;\acc_{r'}\;(x,h)\\
\mbox{where}\qquad r'&:=\lambda x\;x':\{x:\alpha\mid \vph\;x\}.\;r\;(\pi_1\;x)\;(\pi_1\;x')
\end{align*}
Large elimination for $\acc^\vph_r$ is derivable because $\exists h:p.\;q$ has projections when $p$ is a proposition.

In the translation, we must pack up the family into a single type and then use $\acc$ for the recursive instances. Let us run an example first:
$$P=\mu t:\N\to\N\to\P.\;(\intro:\forall n:\N.\;n>2\to (\forall m.\;m<n\to t\;n\;m)\to t\;0\;n)$$
This is a large eliminating type because of the constructor's three arguments, one appears in the result ($t\;0\;n$), one is a proposition ($n>2$), and one is recursive ($\forall m.\;m<n\to t\;n\;m$).

First we pack the domain into a sigma type, in this case $\N\times\N$, and the propositional constraints go into $\vph$. The recursive arguments become the edge relation for $\acc$. Here, $(a,b)$ is accessible when there exists an $n$ such that $(a,b)=(0,n)$, $n>2$ and for all $m<n$, $(n,m)$ is accessible, so we translate this to $\vph(a,b)$ iff there exists $n$ such that $(a,b)=(0,n)$ and $n>2$, and $r\;(a',b')\;(a,b)$ iff there exists $m,n$ such that $(a,b)=(0,n)$ and $m<n$ and $(a',b')=(n,m)$.

In both clauses we introduce a variable $n$ equal to $b$ or $b'$, and this variable can be eliminated. This is true generally because of the restriction on large eliminators: every non-propositional nonrecursive argument, like $n$ here, must appear in the output type, yielding a variable-variable equality $n=b$ which can be used to eliminate $n$. However, due to potential dependencies on earlier arguments, we will delay this elimination to the recursor. So in this translation we have:
\begin{align*}
P\;x\simeq\acc_r^\vph\;(x)\qquad\mbox{where}\qquad\Gamma&\vdash \vph:=\lambda p:\Sigma(x::\alpha).\; B[p/(x)]\\
\Gamma&\vdash r:=\lambda p\;q:\Sigma(x::\alpha).\;R[p/(x')][q/(x)]\\
\Gamma,x::\alpha&\vdash B:\P\\
\Gamma,x'::\alpha,x::\alpha&\vdash R:\P
\end{align*}
where we must specify the definition of $B$ and $R$ inductively with the displayed free variables. Here the notation $B[p/(x)]$ means to replace each $x_i$ with the appropriate projection $\pi_1(\pi_2^i\;p)$ in $B$. We will also accumulate an auxiliary $\Gamma,x'::\alpha,x::\alpha\vdash S:\P$ for constructing the disjunctions in $R$.

$$\boxed{\Gamma;t:F\vdash \tau\LEctor\Rightarrow x.B;x'x.[S;R]}$$
$$\frac{}{\Gamma;t:F\vdash t\;e\LEctor\Rightarrow x.\;x=e;\ x'x.\;[x=e;\bot]}$$
$$\frac{\Gamma,t:F\vdash \beta:\U_\ell\quad \Gamma,y:\beta;t:F\vdash\tau\LEctor\Rightarrow x.B;x'x.[S;R]}
{\Gamma;t:F\vdash\forall y:\beta.\;\tau\LEctor\Rightarrow x.\exists y:\beta.\;B;x'x.[\exists y:\beta.\;S;\exists y:\beta.\;R]}$$
$$\frac{\Gamma;t:F\vdash\beta\LEctor\Rightarrow x.B;x'x.[S;R]}{\Gamma;t:F\vdash(\forall z::\gamma.\;t\;e)\to\beta\LEctor\Rightarrow x.B;x'x.[S;(S\land\exists z::\gamma.\;x'=e)\lor R]}$$
Intuitively, $S$ collects the facts that are true about the main instance argument $x$, so that in each recursive constructor we push a conjunction of $S$ with the fact $\exists z::\gamma.\;x'=e$ we need to hold for $x'$. Since we do the same thing for propositional and index arguments (just existentially generalize everything), we have collapsed both into one rule. Once we have constructed the term, we have the following rule:

$$\frac{\Gamma,t:\forall x::\alpha.\;\U_\ell\vdash \beta\LEctor\Rightarrow x.B;x'x.[S;R]}
{\Gamma\vdash\scott{\mu t:\forall x::\alpha.\;\U_\ell.\;(c:\beta)}=\lambda x::\alpha.\;\ulift^\ell(\acc^\vph_r(x))}$$
\vspace{-5mm}
\begin{align*}
\mbox{where, as before, }\vph&:=\lambda p:\Sigma(x::\alpha).\; B[p/(x)]\\
\mbox{and }r&:=\lambda p\;q:\Sigma(x::\alpha).\;R[p/(x')][q/(x)].
\end{align*}

\subsection{The remainder}
We have described the translation of a recursive type in great detail, but it still remains to define the introduction rules and the recursor, and show that the iota rule holds definitionally with these definitions. As these are more or less uniquely determined by the translated type of the inductive type itself, and it is yet more cumbersome than what has been thus far written, this will be left as future work, probably as part of a formalization of all of this. For now, we will proceed with the understanding that the eight inductive types $\bot,\Sigma,+,\ulift,\|\cdot\|,\W,=,\acc$ are indeed sufficient to cover all Lean-definable inductive types, and leave all this horrible induction behind.

\section{Soundness}
\subsection{Modeling Lean in ZFC}
Let $N\in\N$ be the number of inaccessible cardinals in $V$, or $N=\infty$ if there are infinitely many. For $0\le n<N$, let $U_{n+1}=V_\kappa$ where $\kappa$ is the $n$th inaccessible, and let $U_n=V$ for $n\ge N$. Let $U_0=\{\emptyset,\{\bullet\}\}$ where $\bullet\in U_1$ is any set (we will take $\bullet=\emptyset$ for concreteness).

Note the following properties of the $U$ hierarchy:
\begin{itemize}
\item $U_n$ is a class;
\item $U_m\subseteq U_n$ when $m\le n$;
\item For $n>0$, if $A,B\in U_n$ then $A^B\in U_n$;
\item For $n>0$, if $F:A\to U_n$ then $\bigcup_{x\in A}F(x)\in U_n$;
\item If $m<N$ and $m<n$ then $U_m\in U_n$.
\end{itemize}

Thus the hierarchy is an approximation to a sequence of Grothendieck universes where the upper stages are not necessarily in a strict order, and not necessarily sets.

Also, fix a choice function $\vep_n$ on $U_n$ for all $n<N$, that is, a function such that $\vep_n(x)\in x$ for all $x\in U_n$.

We start with the levels, which have a simple interpretation. Let $UV(\ell)$ denote the set of free universe variables in the level expression $\ell$, and similarly with $UV(e)$. (There are no universe binding operations, so all variables are free.) The expression $\scott{\ell}_v$ is defined when $v$ is a function with domain containing $FV(\ell)$ and codomain $\N$, as follows:
\begin{align*}
\scott{u}_v&=v(u)\\
\scott{0}_v&=0\\
\scott{S\ell}_v&=\scott{\ell}_v+1\\
\scott{\max(\ell,\ell')}_v&=\max(\scott{\ell}_v,\scott{\ell'}_v)\\
\scott{\imax(\ell,\ell')}_v&=\begin{cases}
0&\mbox{if }\scott{\ell'}_v=0\\
\max(\scott{\ell}_v,\scott{\ell'}_v)&\mbox{if }\scott{\ell'}_v\ne 0
\end{cases}
\end{align*}
Fix a valuation $v$ on the universe variables. We define the expressions $\scott{\Gamma}_v$ when $\Gamma$ is a well formed context, and $\scott{\Gamma\vdash e}_v$ when $\Gamma\vdash e:\alpha$ for some $\alpha$, by mutual recursion on the following measure:
\begin{itemize}
\item The size of an expression $|e|$ is the sum of all its immediate subterms plus 1.
\item The size of a context is $|{\cdot}|=1/2$, $|\Gamma,x:\alpha|=|\Gamma|+|\alpha|$.
\item The size of an expression in context is $|\Gamma\vdash e|=|\Gamma|+|e|-1/2$.
\end{itemize}
Note that $|\Gamma|<|\Gamma\vdash e|$ and $|\Gamma\vdash\alpha|<|\Gamma,x:\alpha|$. Here $\scott{\Gamma}$ will be a set of lists of types, and $\scott{\Gamma\vdash e}$ will be a (total) function on $\scott{\Gamma}$, if it is defined at all. We will denote the evaluation at $\gamma\in\scott{\Gamma}$ by $\scott{\Gamma\vdash e}_\gamma$ (or $\scott{\Gamma\vdash e}_{v,\gamma}$ when being explicit about the universe valuation as well).

Let $\prod^\ell_{x\in A}B=\begin{cases}
\prod_{x\in A}B&\mbox{if }\scott{\ell}\ne 0\\
\bigcap_{x\in A}B&\mbox{if }\scott{\ell}=0.
\end{cases}$ This is the ``modified'' product space which depends on the target universe $\ell$. Also let $\bar R=\{(a,b)\mid\bullet\in R(a)(b)\}$, which translates a DTT relation into a ZFC relation.

\begin{itemize}
\item $\scott{\cdot}=()$, the empty list. (This can be encoded as $\emptyset$.)
\item $\scott{\Gamma,x:\alpha}=\sum_{\gamma\in\scott{\Gamma}}\scott{\Gamma\vdash\alpha}_\gamma$, that is, the set of pairs $(\gamma,x)$ such that $\gamma\in\scott{\Gamma}$ and $x\in\scott{\Gamma\vdash\alpha}_\gamma$.
\item $\scott{\Gamma\vdash x}_\gamma=\pi_i(\gamma)$, where $x$ is the $i$th variable in the context.
\item $\scott{\Gamma\vdash\U_\ell}_\gamma=U_{\scott{\ell}}$ if $\scott{\ell}<N$, else undefined
\item If $\Gamma\vdash e_1:\beta$ and $\Gamma\vdash\beta:\U_\ell$, then $$\scott{\Gamma\vdash e_1\;e_2}_\gamma=\begin{cases}
\scott{\Gamma\vdash e_1}_\gamma(\scott{\Gamma\vdash e_2}_\gamma)&\mbox{if }\scott{\ell}\ne 0\\
\bullet&\mbox{if }\scott{\ell}=0.
\end{cases}$$
\item If $\Gamma,x:\alpha\vdash e:\beta$ and $\Gamma,x:\alpha\vdash\beta:\U_\ell$, then
$$\scott{\Gamma\vdash \lambda x:\alpha.\;e}_\gamma=\begin{cases}
(x\in\scott{\Gamma\vdash \alpha}_{\gamma}\mapsto\scott{\Gamma,x:\alpha\vdash e}_{(\gamma,x)})&\mbox{if }\scott{\ell}\ne 0\\
\bullet&\mbox{if }\scott{\ell}=0.
\end{cases}$$
\item If $\Gamma,x:\alpha\vdash e:\beta$ and $\Gamma,x:\alpha\vdash\beta:\U_\ell$, then\\
$\scott{\Gamma\vdash \forall x:\alpha.\;e}_\gamma=\prod^\ell_{x\in \scott{\Gamma\vdash\alpha}_\gamma}\scott{\Gamma,x:\alpha\vdash\beta}_{(\gamma,x)}$.
\item $\scott{\Gamma\vdash\bot}_\gamma=\emptyset$
\item $\scott{\Gamma\vdash\rec_\bot}_\gamma=\emptyset$ (the empty function)
\item $\scott{\Gamma\vdash\Sigma x:\alpha.\;\beta}_\gamma=\sum_{x\in \scott{\Gamma\vdash\alpha}_\gamma}\scott{\Gamma\vdash\beta}_{(\gamma,x)}$
\item $\scott{\Gamma\vdash(e_1,e_2)}_\gamma=(\scott{\Gamma\vdash e_1}_\gamma,\scott{\Gamma\vdash e_2}_\gamma)$
\item $\scott{\Gamma\vdash\pi_1\;e}_\gamma=\pi_1(\scott{\Gamma\vdash e}_\gamma)$
\item $\scott{\Gamma\vdash\pi_2\;e}_\gamma=\pi_2(\scott{\Gamma\vdash e}_\gamma)$
\item $\scott{\Gamma\vdash\alpha+\beta}_\gamma=\scott{\Gamma\vdash\alpha}_\gamma\sqcup\scott{\Gamma\vdash\beta}_\gamma$
\item $\scott{\Gamma\vdash\inl e}_\gamma=\iota_1(\scott{\Gamma\vdash\alpha}_\gamma)$
\item $\scott{\Gamma\vdash\inr e}_\gamma=\iota_2(\scott{\Gamma\vdash\beta}_\gamma)$
\item If $\Gamma\vdash C:\alpha+\beta\to\U_\ell$ where $\scott{\ell}=0$, then $\scott{\Gamma\vdash\rec_+^C\;a\;b}_\gamma=\bullet$, otherwise $\scott{\Gamma\vdash\rec_+^C\;a\;b}_\gamma$ is the function on $\scott{\Gamma\vdash\alpha}_\gamma\sqcup\scott{\Gamma\vdash\beta}_\gamma$ such that
\begin{align*}
\scott{\Gamma\vdash\rec_+^C\;a\;b}_\gamma(\iota_1(x))&=\scott{\Gamma\vdash a}_\gamma(x)&&\mbox{for }x\in \scott{\Gamma\vdash\alpha}_\gamma\\
\scott{\Gamma\vdash\rec_+^C\;a\;b}_\gamma(\iota_2(y))&=\scott{\Gamma\vdash a}_\gamma(y)&&\mbox{for }y\in \scott{\Gamma\vdash\beta}_\gamma.
\end{align*}
\item $\scott{\Gamma\vdash\ulift_\ell^{\ell'} \alpha}_\gamma=\scott{\Gamma\vdash\alpha}_\gamma$
\item $\scott{\Gamma\vdash{\uparrow}e}_\gamma=\scott{\Gamma\vdash{\downarrow}e}_\gamma=\scott{\Gamma\vdash e}_\gamma$
\item $\scott{\Gamma\vdash\|\alpha\|}_\gamma=\{x\in\{\bullet\}\mid\scott{\Gamma\vdash\alpha}_\gamma\ne \emptyset\}$
\item $\scott{\Gamma\vdash|e|}_\gamma=\bullet$
\item $\scott{\Gamma\vdash\rec_{||}^{C,\alpha}\;f}_\gamma=\bullet$
\item $\scott{\Gamma\vdash\W x:\alpha.\;\beta}_\gamma=\W_{x\in\scott{\Gamma\vdash\alpha}_\gamma}\scott{\Gamma,x:\alpha\vdash\beta}_{(\gamma,x)}$ (see below)
\item $\scott{\Gamma\vdash\sup a\;f}_\gamma=(\scott{\Gamma\vdash a}_\gamma,\scott{\Gamma\vdash f}_\gamma)$
\item If $\Gamma\vdash C:(\W x:\alpha.\;\beta)\to\U_\ell$ where $\scott{\ell}=0$, then $\scott{\Gamma\vdash\rec_\W^C\;e}_\gamma=\bullet$, otherwise $\scott{\Gamma\vdash\rec_\W^C\;e}_\gamma=\rec_\W(\scott{\Gamma\vdash\W x:\alpha.\;\beta}_\gamma,\scott{\Gamma\vdash e}_\gamma)$ (see below)
\item $\scott{\Gamma\vdash a=b}_\gamma=\{x\in\{\bullet\}\mid\scott{\Gamma\vdash a}_\gamma=\scott{\Gamma\vdash b}_\gamma\}$
\item $\scott{\Gamma\vdash \refl\;a}_\gamma=\bullet$
\item If $\Gamma\vdash C:\alpha\to\U_\ell$ where $\scott{\ell}=0$, then $\scott{\Gamma\vdash\rec_=^{C,a}\;e\;b\;h}_\gamma=\bullet$, otherwise $\scott{\Gamma\vdash\rec_=^{C,a}\;e\;b\;h}_\gamma=\scott{\Gamma\vdash e}_\gamma$
\item $\scott{\Gamma\vdash\acc_r\;x}_\gamma=\{y\in\{\bullet\}\mid\acc_{\overline{\scott{\Gamma\vdash r}_\gamma}}(x)\}$ (see below)
\item $\scott{\Gamma\vdash\intro_\acc\;x\;f}_\gamma=\bullet$
\item If $\Gamma\vdash C:\alpha\to\U_\ell$ where $\scott{\ell}=0$, then $\scott{\Gamma\vdash\rec_\acc^{C,r}\;e}_\gamma=\bullet$, otherwise $\scott{\Gamma\vdash\rec_\acc^{C,r}\;e}_\gamma=\rec_\acc(\scott{\Gamma\vdash\alpha}_\gamma,\overline{\scott{\Gamma\vdash r}_\gamma},\scott{\Gamma\vdash e}_\gamma)$ (see below)
\item If $\Gamma\vdash\alpha:\U_\ell$ where $\scott{\ell}=0$, then
\begin{itemize}
\item $\scott{\Gamma\vdash\alpha/R}_\gamma=\scott{\Gamma\vdash\alpha}_\gamma$
\item $\scott{\Gamma\vdash\mk_R\;x}_\gamma=\scott{\Gamma\vdash x}_\gamma$
\item $\scott{\Gamma\vdash\lift^u_R\;\beta\;f\;h}_{v,\gamma}=\scott{\Gamma\vdash f}_\gamma$
\end{itemize}
Otherwise, with $r:=\overline{\scott{\Gamma\vdash R}_\gamma}$ in the following clauses:
\begin{itemize}
\item $\scott{\Gamma\vdash\alpha/R}_\gamma=\scott{\Gamma\vdash\alpha}_\gamma/r$
\item $\scott{\Gamma\vdash\mk_R\;x}_\gamma=[\scott{\Gamma\vdash x}_\gamma]_r$
\item If $v(u)=0$ then $\scott{\Gamma\vdash\lift^u_R\;\beta\;f\;h}_v=\bullet$, otherwise $\scott{\Gamma\vdash\lift^u_R\;\beta\;f\;h}_\gamma$ is the function such that $\scott{\Gamma\vdash\lift^u_R\;\beta\;f\;h}_\gamma([x]_r)=\scott{\Gamma\vdash f}_\gamma(x)$
\end{itemize}
\item $\scott{\Gamma\vdash\mathsf{propext}}_\gamma=\bullet$
\item $\scott{\Gamma\vdash\mathsf{choice}_u\;\alpha\;h}_{v,\gamma}=\vep_{v(u)}(\scott{\Gamma\vdash\alpha}_\gamma)$

\end{itemize}
UNFINISHED

% \section{Strong normalization}
\subsection{Evaluation in the empty context}
In \autoref{sec:church_rosser}, we set up a reduction relation $\rightsquigarrow_\kappa$ which is plainly useless for a normalization algorithm, because of the following rule:
$$(K)\ \frac{P\mbox{ is SS inductive}}{\Gamma\vdash \rec_P(C,e,p,h)\rightsquigarrow_\kappa e\;\inv[p,h]\;v}$$
Recall that $v$ is a sequence of lambdas $v_i=\lambda x::\xi_i.\;\rec_P(C,e,\pi_i[\inv[p,h],x],u_i\;x)$. That is, we reduce any $\rec_P$ term, regardless of its arguments, to a longer expression that contains another $\rec_P$ term (assuming that the inductive type has a recursive argument). So this is guaranteed not to terminate, but it was sufficiently powerful to act as a semi-decision procedure for definitional equality where existing methods fail (by giving up early, they recover decidability at the cost of completeness).

But our examples thus far of undecidability of definitional equality all work in an inconsistent context. Is this necessary? What if the context is known to be inhabited by some terms? In this case we may as well substitute the terms in to reduce the problem to evaluation in the empty context.

That is, we wish to determine if $\vdash e\equiv e'$, hopefully by a method similar to the reduction $\rightsquigarrow_\kappa$, but without the self-destructive unfolding K rule.

% \section{Compilation}
First we translate the types:
$$\tau::=\ast\mid \obj\mid \tau\to\tau$$

$$\frac{\Gamma\vdash\alpha\Rightarrow \alpha'\quad \Gamma,x:\alpha\vdash \beta\Rightarrow \beta'}
{\Gamma\vdash\forall x:\alpha.\;\beta\Rightarrow
\begin{cases}
\ast&\mbox{if }\beta'=\ast\\
\alpha'\to\beta'&o.w.
\end{cases}}\qquad
\frac{}{\Gamma\vdash\U_\ell\Rightarrow \ast}\qquad
\frac{\Gamma\vdash\alpha\rightsquigarrow \alpha'\quad \Gamma\vdash \alpha'\Rightarrow \tau}
{\Gamma\vdash\alpha\Rightarrow\tau}$$
$$\frac{}{\Gamma\vdash(\mu t:(\forall x::\alpha.\;\P).\;K)\;e\Rightarrow \ast}\qquad
\frac{\ell\not\equiv 0}{\Gamma\vdash(\mu t:(\forall x::\alpha.\;\U_\ell).\;K)\;e\Rightarrow \obj}$$

Next we translate expressions:
$$e::=x\mid $$
$$\frac{}{\Gamma\vdash x\Rightarrow x}\qquad
\frac{}{\Gamma\vdash\U_\ell\Rightarrow \ast}\qquad
\frac{}{\Gamma\vdash\forall x:\alpha.\;\beta\Rightarrow \ast}$$
$$\qquad
\frac{\ell\not\equiv 0}{\Gamma\vdash(\mu t:(\forall x::\alpha.\;\U_\ell).\;K)\;e\Rightarrow \obj}$$

UNFINISHED


\bibliography{references}
\bibliographystyle{plain}

\end{document}
